As mentioned in \sref{ie-polynomial-chaos} and \aref{polynomial-chaos}, the coefficients of PC expansions are integrals, which, in general, should be calculated numerically. In numerical integration, an integral of a function is approximated by a summation over the function values computed at a set of prescribed points, or nodes, which are multiplied by the corresponding set of prescribed weights. Such pairs of nodes and weights are called quadrature rules \cite{press2007}. A quadrature rule is characterized by its precision $\qdprecision$, which is defined as the maximal total order of polynomials that the rule integrates exactly. A sequence of rule with increasing precision forms a family of quadrature rules, wherein the index of a particular rule in the family is called its accuracy level denoted by $\qdlevel$. In multiple dimensions, an $\vars$-variate quadrature rule is then a tensor product of one-dimensional counterparts.

It can be seen in \eref{inner-product} that the integrand can be decomposed into two parts: the weight function, $\PDF(\vz)$, and everything else. The former stays the same for all inner products; therefore, an integration rule is typically chosen in such a way that the corresponding weights take into consideration this ``constant'' part since there is no point of recomputing $\PDF(\vz)$ each time when the other part, \ie, the functions that the inner product operates on, changes. Consequently, there exist different families of quadrature rules tailored for different weight functions. Define such a quadrature-based approximation of \eref{inner-product} as
\[
  \oInner{f(\vz)}{g(\vz)} \approx \oQuad{\vars}{\qdlevel}{f(\vz) \: g(\vz)} = \sum_{i = 1}^{\qdorder} f(\qdn{\vZ}_i) \: g(\qdn{\vZ}_i) \: \qdw_i
\]
where $\qdn{\vZ}_i \in \real^\vars$ and $\qdw_i \in \real$ are the prescribed nodes and weights, respectively, $\qdorder$ is their number, and $\qdlevel$ is the accuracy level of the quadrature rule, which is $\vars$-variate. Note that, once the rule to use is identified, $\qdn{\vZ}_i$ and $\qdw_i$ are fixed, \ie, there are no any additional computational efforts.

Since in the illustrative example in \sref{illustrative-example} we need to compute inner products with respect to Gaussian measures, the Gauss-Hermite quadrature rule is of particular interest. The rule belongs to a broad class of rules known as Gaussian rules, which implies that its order $\qdorder$ and presicion $\qdprecision$ are related as $\qdprecision = 2 \qdorder - 1$; this feature makes Gaussian quadratures especially efficient \cite{heiss2008}. Consequently, we rewrite \eref{pc-coefficients} as
\[
  \pcc{\vP}_{ki} = \frac{1}{\pcn_i} \oQuad{\vars}{\qdlevel}{\vP_k(\o) \pcb_i(\vZ(\o))}
\]
where $\{ \pcn_i \}_{i = 1}^{\pcterms}$ are computed exactly either by directly using the same quadrature rule or by taking products of one-dimensional counterparts with known analytical expressions \cite{xiu2010}; the result is further tabulated. It is important to note that $\qdlevel$ should be chosen in such a way that the rule is exact for polynomials of total order at least $2 \pcorder$, \ie, twice the order of the PC expansion, which can be seen in \eref{pc-coefficients} \cite{eldred2009}. Therefore, $\qdlevel \geq \pcorder + 1$ as the quadrature is Gaussian.

There is one more and, arguably, the most crucial aspect of numerical integration that we ought to discuss: the algorithm used to construct multidimensional quadratures. In low dimensions, the construction can be easily based on the direct tensor product of one-dimensional rules. However, in high dimensions, the situation changes dramatically as the number of points produced by this approach can easily explode. For instance \cite{heiss2008}, if a one-dimensional rule has only four points, \ie, $\qdorder = 4$, then in ten dimensions, \ie, $\vars = 10$, the number of points becomes $\qdorder^{\vars} = 4^{10} = 1,048,576$, which is not affordable. Moreover, it can be shown that most of the points obtained in such a way do not contribute to the asymptotic accuracy and, therefore, are a waste of the computational time. To tackle this difficulty, we construct so-called sparse grids using one algorithm that can reasonable be call the state of the art in multidimensional integration. The algorithm was proposed by Sergey Smolyak in 1963 and is now known by his name. The Smolyak algorithm preserves the accuracy of the underlying one-dimensional rules for complete polynomials while significantly reducing the number of integration nodes. For instance, in the example given earlier, the number of points computed by the algorithm would be only $1,581$. Unfortunately, the Smolyak algorithm is out of the scope of this paper; the interested reader is referred to, \eg, \cite{eldred2009, maitre2010, heiss2008} for further details.
