We adopt the commonly used thermal model based on Fourier's heat equation, which, after a proper spacial discretization, leads to the following system of DAEs:
\begin{subnumcases}{\elabel{fourier-system-original}}
  \mC \frac{d\vTI(\t, \o)}{d\t} + \mG \: \vTI(\t, \o) = \mM \: \vP(\t, \o) \elabel{fourier-original} \\
  \vTO(\t, \o) = \mM^T \vTI(\t, \o) + \vTO_\amb
\end{subnumcases}
Here, the number of differential equations is $\nodes$ where $\nodes$ is the number of thermal nodes; $\mC, \mG \in \real^{\nodes \times \nodes}$ are a diagonal matrix of the thermal capacitance and a symmetric, positive-definite matrix of the thermal conductance, respectively; $\vTI \in \real^{\nodes}$ is the vector of the difference between the temperature of the thermal nodes and the ambient temperature; $\vP \in \real^\cores$ and $\mM \in \real^{\nodes \times \cores}$ are the input vector of power (see \eref{power-model}) and its mapping matrix to the thermal nodes; $\vTO \in \real^\cores$ is the temperature vector of the processing elements, and $\vTO_\amb \in \real^\cores$ is the vector of the ambient temperature. Without loss of generality, the last is modeled as a deterministic variable. For convenience, we perform an auxiliary transformation of \eref{fourier-system-original}  using \cite{ukhov2012}
\begin{equation*}
  \vX = \mC^\frac{1}{2} \vTI, \qquad \mA = -\mC^{-\frac{1}{2}} \mG \mC^{-\frac{1}{2}}, \qquad \mB = \mC^{-\frac{1}{2}} \mM
\end{equation*}
and obtain the system in \eref{fourier-system} where the coefficient matrix $\mA$ preserves the symmetry and positive-definiteness of $\mG$. In general, the differential part in \eref{fourier-system}, or equally in \eref{fourier-system-original}, is nonlinear due to the source term $\vP(\t, \o)$, which has an arbitrary dependency on temperature (see \eref{power-model}). Therefore, there are no closed-form solutions to the system.

The time intervals $\{ \dt_i \}_{i = 1}^{\steps}$ of the input profile $\profPdyn$ (see \sref{problem-formulation}) are assumed to be short enough such that the total power can be approximated by a constant within one interval. In this case, \eref{fourier-de} is a system of linear differential equations that can be solved analytically. The solution is
\begin{equation} \elabel{ode-solution}
  \vX(\t, \o) = \mE(\t) \: \vX(0, \o) + \mD(\t) \: \vP(0, \o)
\end{equation}
where $\t \in [0, \dt]$, $\vP(0, \o)$ is the power dissipation at the beginning of the time interval with respect to the corresponding temperature, and
\begin{align*}
  & \mE(\t) = e^{\mA \t} \in \real^{\nodes \times \nodes} \\
  & \mD(\t) = \mA^{-1} (e^{\mA \t} - \mI) \mB \in \real^{\nodes \times \cores}
\end{align*}
Repeating the procedure for all time intervals $\{ \dt_k \}_{k=1}^{\steps}$ of the power profile and letting the initial temperature be equal to the abmient temperature, which has no loss of generality, we obtain the recurrence in \eref{recurrence}.

Then, we use the technique described in \cite{ukhov2012} to compute the coefficient matrices $\mE(\t)$ and $\mD(\t)$ of the recurrence in \eref{recurrence}.
