Here we elaborate on the orthogonality property \cite{xiu2010, maitre2010} of PC expansions, which is extensively utilized at \stage{4}\ in \sref{polynomial-chaos}.
Due to the inherent complexity, uncertainty quantification problems are typically viewed as approximation problems.
More precisely, one usually constructs computationally efficient surrogates of the initial models and then studies these light representations instead.
PC expansions \cite{xiu2010} are one way to perform such approximations, in which the approximating functions are orthogonal polynomials.
A set of multivariate polynomials $\{ \pcb_i(\vz) \}$ is orthogonal if
\begin{equation} \elabel{orthogonality}
  \oInner{\pcb_i(\vz)}{\pcb_j(\vz)} = \pcn_i \delta_{ij}, \qquad \forall i, j,
\end{equation}
where $\oInner{\cdot}{\cdot}$ denotes the inner product in the Hilbert space spanned by the polynomials, $\delta_{ij}$ is the Kronecker delta function, and $\pcn_i = \oInner{\pcb_i(\vz)^2}$ is a normalization constant.
The inner product with a weight function $\fPDF(\vz)$ is defined as the following multidimensional integral:
\begin{equation} \elabel{inner-product}
  \oInner{h(\vz)}{g(\vz)} := \int h(\vz) g(\vz) \fPDF(\vz) d\vz.
\end{equation}
In the stochastic context, the parameters $\vz$ correspond to random variables, which we denote by $\vZ(\o)$, and the weight function corresponds to the \pdf\ of $\vZ(\o)$ (see \sref{uncertain-parameters}).
Some of the standard distributions and the corresponding polynomial bases are given in \tref{askey} \cite{eldred2008} and can be found in the Askey scheme \cite{xiu2010} of hypergeometric orthogonal polynomials.
It can be seen that the weight functions are equal to the \pdfs\ of the corresponding probability distributions up to constant multipliers.
Therefore, the inner product coincides with the covariance operator.
Consequently, the presence of orthogonality is equivalent to the absence of correlations.

\updated{Finally, let us note the following.
A distribution that does not have a direct correspondence with any basis from the Askey scheme can be transformed to one of those that do have such a correspondence using the technique shown in \sref{ie-uncertain-parameters}.
Another solutions is to construct a custom polynomial basis using the Gram-Schmidt procedure.
In addition, apart from continuous, PC expansions can be applied to discrete distributions.
Refer to \cite{xiu2010, maitre2010} for further discussions.}
