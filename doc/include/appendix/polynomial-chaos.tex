Due to the inherent complexity, real-world uncertainty quantification problems are typically viewed as approximation problems, wherein one first constructs light representations of the initial models and then studies the constructed surrogates as they are computationally more efficient.
PC expansions \cite{xiu2010} are one way to perform such approximations, in which the approximating functions are orthogonal polynomials.
In this section, we provide some of the basic properties \cite{xiu2010, maitre2010} of this class of polynomials necessary for the PC expansions of power and temperature derived in \sref{polynomial-chaos}.

A set of multivariate polynomials $\{ \pcb_i(\vz) \}$ is orthogonal if
\begin{equation} \elabel{orthogonality}
  \oInner{\pcb_i(\vz)}{\pcb_j(\vz)} = \pcn_i \delta_{ij}, \qquad \forall i, j,
\end{equation}
where $\oInner{\idot}{\idot}$ denotes the inner product in the Hilbert space spanned by the polynomials, $\delta_{ij}$ is the Kronecker delta function, and $\pcn_i = \oInner{\pcb_i(\vz)^2}$ is a normalization constant.
The inner product with a weight function $\PDF(\vz)$ is defined as the following multidimensional integral:
\begin{equation} \elabel{inner-product}
  \oInner{h(\vz)}{g(\vz)} = \int h(\vz) g(\vz) \PDF(\vz) d\vz.
\end{equation}
In the stochastic context, the parameters $\vz$ correspond to random variables, denoted by $\vZ(\o)$, and the weight function corresponds to the \pdf\ of $\vZ(\o)$ (see \sref{uncertain-parameters}).
Some of the standard distributions and the corresponding polynomial bases are given in \tref{askey} \cite{eldred2008}; these polynomials can be found in the Askey scheme \cite{xiu2010} of hypergeometric orthogonal polynomials.
It can be seen that the weight functions are equal the \pdfs\ of the corresponding probability distributions up to constant multipliers.
Therefore, the inner product coincides with the covariance operator.
Consequently, the presence of orthogonality is equivalent to the absence of correlations.
Note that, although we focus on continuous random variables, the PC approach and, thus, our framework work equally well for discrete probability distributions; see, \eg, \cite{xiu2010, maitre2010}.

The general formula of a truncated PC expansion applied to the power term in \eref{recurrence} is given in \sref{polynomial-chaos}, \eref{pc-expansion}.
However, we have not shown yet how to find the corresponding coefficients of the expansion, $\pcc{\vP}_{ki}$.
To this end, a spectral projection of the stochastic quantity being expanded, \ie, $\vP_k(\o)$, is to be performed onto the space spanned by $\{ \pcb_i(\vZ(\o)) \}_{i = 1}^{\pcterms}$, where $\pcterms$ is the number of polynomials in the truncated basis.
This means that we need to compute inner products of \eref{power-model} with each polynomial from the basis as
\[
  \oInner{\vP_k(\o)}{\pcb_i(\vZ(\o))} = \oInner{\sum_{j=1}^{\pcterms} \pcc{\vP}_{kj} \: \pcb_j(\vZ(\o))}{\pcb_i(\vZ(\o))}
\]
where $i = 1, \dotsc, \pcterms$ and $k = 1, \dotsc, \nsteps$.
Making use of the orthogonality in \eref{orthogonality}, we obtain
\begin{equation} \elabel{pc-coefficients}
  \pcc{\vP}_{ki} = \frac{1}{\pcn_i} \oInner{\vP_k(\o)}{\pcb_i(\vZ(\o))}.
\end{equation}
In general, the inner product in \eref{pc-coefficients}, defined by \eref{inner-product}, should be evaluated numerically.
This task is independent from the rest of the derivations in this section and, therefore, is discussed separately in \aref{gauss-quadrature}.
The total number of the coefficient, $\pcterms$, is given by the following expression, which corresponds to the total-order polynomial space \cite{eldred2008, beck2011}:
\begin{equation} \elabel{pc-terms}
  \pcterms = { \pcorder + \nvars \choose \nvars } = \frac{(\pcorder + \nvars)!}{\pcorder! \nvars!}.
\end{equation}
The overall procedure is to be perform for $k = 1, \dotsc, \nsteps$.

Finally, we need to show the derivation of the recurrent expression in \eref{pc-recurrence}, which is used together with \eref{fourier-output} to compute the PC coefficients of temperature. Using the definition in \eref{pc-expansion}, \eref{expanded-recurrence} can be explicitly written as
\[
  \sum_{i = 1}^{\pcterms} \pcc{\vX}_{ki} \: \pcb_i(\vZ(\o)) = \sum_{i = 1}^{\pcterms} \left( \mCF_k \: \pcc{\vX}_{(k - 1)i} + \mCS_k \: \pcc{\vP}_{ki} \right) \pcb_i(\vZ(\o)).
\]
Multiplying the above equation by each polynomial from the basis and making use of the orthogonality property in \eref{orthogonality}, we obtain \eref{pc-recurrence}.
It is worth being mentioned that, since $\vP(\o)$ depends on temperature (see \sref{power-model}), at each step of the iterative process in \eref{pc-recurrence}, the computation of $\pcc{\vP}_{ki}$ should be done with respect to the current PC expansion of temperature, \ie, $\vTO_{k - 1}(\o)$.
The computed coefficients can now be utilized to evaluate various statistics of the stochastic power and temperature processes.
Also, as shown in \eref{pc-moments}, PC series provide analytical expressions for probabilistic moments.
The formulae in \eref{pc-moments} are due to the fact that, by definition \cite{xiu2010}, the first polynomial $\pcb_1(\vZ(\o))$ in a polynomial basis is unity; hence, $\oExp{\pcb_1(\vZ(\o))} = 1$.
Therefore, using \eref{orthogonality}, we conclude that $\oExp{\pcb_i(\vZ(\o))} = 0$ for $i = 2, \dotsc, \pcterms$.

From the development above, it is important to note that the theory of PC expansions suffers from the so-called curse of dimensionality.
\eref{pc-coefficients} together with \eref{pc-terms} reveal this problem: when the number of stochastic dimensions increases, the number of polynomial terms as well as the complexity of the corresponding coefficients exhibit a growth, which is exponential without special treatments.
The problem does not have a general solution and is one of the central topics of many ongoing studies.
In this paper, we mitigate this issue (a) by keeping the number of stochastic dimensions low using the KL expansion (see \sref{ie-uncertain-parameters}) and (b) by utilizing efficient integration techniques as discussed in \aref{gauss-quadrature}.
\begin{table}[b]
  \vspace{-2.0em}
  \centering
  \caption{Probability distributions and polynomial bases \cite{eldred2008}}
  \vspace{-1.0em}
  \begin{tabular*}{0.95\linewidth}{L{65pt}L{75pt}c}
    \toprule
    Distribution & Polynomial basis & Weight function \\
    \midrule
    \midrule
    Gaussian & Hermite & $e^{{-\z^2}/{2}}$ \\
    Uniform & Legendre & $1$ \\
    Beta & Jacobi & $(1 - \z)^\alpha (1 + \z)^\beta$ \\
    Exponential & Laguerre & $e^{-\z}$ \\
    Gamma & Generalized Laguerre & $\z^\alpha e^{-\z}$ \\
    \bottomrule
  \end{tabular*}
  \tlabel{askey}
\end{table}

