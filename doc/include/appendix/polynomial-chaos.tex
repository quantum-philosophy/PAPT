In this section, we discuss certain properties of the PC expansions that form the core of the proposed approach and are extensively utilized at \stage{4}\ described in \sref{polynomial-chaos}.

Due to the inherent complexity, real-world uncertainty quantification problems are typically viewed as approximation problems, wherein one first constructs light representations of the initial models and then studies the constructed surrogates as they are computationally more efficient.
PC expansions \cite{xiu2010} are one way to perform such approximations, in which the approximating functions are orthogonal polynomials.
In this section, we provide some of the basic properties \cite{xiu2010, maitre2010} of this class of polynomials necessary for the PC expansions of power and temperature derived in \sref{polynomial-chaos}.

A set of multivariate polynomials $\{ \pcb_i(\vz) \}$ is orthogonal if
\begin{equation} \elabel{orthogonality}
  \oInner{\pcb_i(\vz)}{\pcb_j(\vz)} = \pcn_i \delta_{ij}, \qquad \forall i, j,
\end{equation}
where $\oInner{\idot}{\idot}$ denotes the inner product in the Hilbert space spanned by the polynomials, $\delta_{ij}$ is the Kronecker delta function, and $\pcn_i = \oInner{\pcb_i(\vz)^2}$ is a normalization constant.
The inner product with a weight function $\fPDF(\vz)$ is defined as the following multidimensional integral:
\begin{equation} \elabel{inner-product}
  \oInner{h(\vz)}{g(\vz)} := \int h(\vz) g(\vz) \fPDF(\vz) d\vz.
\end{equation}
In the stochastic context, the parameters $\vz$ correspond to random variables, denoted by $\vZ(\o)$, and the weight function corresponds to the \pdf\ of $\vZ(\o)$ (see \sref{uncertain-parameters}).
Some of the standard distributions and the corresponding polynomial bases are given in \tref{askey} \cite{eldred2008}; these polynomials can be found in the Askey scheme \cite{xiu2010} of hypergeometric orthogonal polynomials.
It can be seen that the weight functions are equal the \pdfs\ of the corresponding probability distributions up to constant multipliers.
Therefore, the inner product coincides with the covariance operator.
Consequently, the presence of orthogonality is equivalent to the absence of correlations.
Note that, although we focus on continuous random variables, the PC approach and, thus, our framework work equally well for discrete probability distributions; see, \eg, \cite{xiu2010, maitre2010}.
