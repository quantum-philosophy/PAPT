The total power dissipation of the system with $\cores$ processing elements is defined in the following form:
\begin{equation} \elabel{power-model}
  \vP(\t, \o) = \f\Big(\vP_\dyn(\t), \vTO(\t, \o), \vZ(\o)\Big)
\end{equation}
where $\f: \real^\cores \times \real^\cores \times \real^\vars \to \real^\cores$ is an arbitrary function, possibly a ``black box'', of the nominal dynamic power $\vP_\dyn(\t)$, stochastic temperature $\vTO(\t, \o)$, and uncertain parameters $\vZ(\o)$. In this work, the function is assumed to be smooth in $\vZ(\o)$ and to have a finite variance, which is applicable to most physical systems \cite{xiu2002}. The definition of $\f$ is flexible enough to account for such effects as the interdependency between the leakage current and temperature \cite{liu2007, srivastava2010}. In this case, one can split the function into dynamic $\f_\dyn(\vP_\dyn(\t), \o)$ and leakage $\f_\leak(\vTO(\t, \o), \o)$ parts and define appropriate models for the components; this partition is to be further discussed in \sref{illustrative-example}.
