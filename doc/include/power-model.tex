In this section, we introduce the power model used in conjunction with the thermal model presented in \sref{thermal-model}. The (total) power dissipation of the system with $\cores$ processing elements is given as the following sum:
\begin{equation} \elabel{power-model}
  \vP(\t, \o) = f_\dyn(\vP_\dyn(\t), \o) + f_\leak(\vTO(\t, \o), \o)
\end{equation}
where $f_\dyn, f_\leak: \real^\cores \times \outcomes \to \real^\cores$ model the dynamic and leakage power, respectively, as functions of the uncertain parameters $\vZ(\o)$. In addition, $f_\dyn$ depends on the nominal dynamic power $\vP_\dyn \in \real^\cores$ while the leakage part, $f_\leak$, is driven by the operating temperature $\vTO \in \real^\cores$. The later dependency is well-known and strictly nonlinear, cf. \cite{srivastava2010, liu2007}. $f_\dyn$ and $f_\leak$ are assumend to have finite variances, which is applicable to most physical systems \cite{xiu2002}. The functions are not required to have explicit mathematical formulations; they can be given as a ``black box'' as long as the input $\vZ(\o)$ is preserved. We do not impose any further restrictions on $f_\dyn$ and $f_\leak$, and let the user of the proposed framework decide\footnote{Power model are available in the contemporary literature, for instance, follow the reference in \cite{srivastava2010}.}.
