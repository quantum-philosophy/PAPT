The (total) power dissipation of the system with $\cores$ processing elements is given as the following sum:
\begin{equation} \elabel{power-model}
  \vP(\t, \o) = f_\dyn(\vP_\dyn(\t), \o) + f_\leak(\vTO(\t, \o), \o)
\end{equation}
where $f_\dyn, f_\leak: \real^\cores \times \outcomes \to \real^\cores$ model the dynamic and leakage power, respectively, as functions of the uncertain parameters $\vZ(\o)$. In addition, $f_\dyn$ depends on the nominal dynamic power $\vP_\dyn \in \real^\cores$, and the leakage part, $f_\leak$, depends on the operating temperature $\vTO \in \real^\cores$ due to the well-known non-linear interdependency between leakage and temperature, c.f. \cite{srivastava2010, liu2007}. We do not impose any further restrictions on $f_\dyn$ and $f_\leak$, and let the user to decide\footnote{One can easily find a dicent number of alternatives in the contemporary literature, for instance, follow the reference in \cite{srivastava2010}.}. Moreover, the functions are not required to have explicit mathematical formulations; they can be given as a `black box' as long as its inputs are in the set $\{ \vP_\dyn, \vTO, \vZ \}$.
