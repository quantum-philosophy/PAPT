The power dissipation of the processing elements of the system is defined as the following $\cores$-dimensional vector:
\begin{equation} \elabel{power-model}
  \vP(\t, \o) = \f(\vP_\dyn(\t), \vTO(\t, \o), \vU(\o))
\end{equation}
where $\f: \real^\cores \times \real^\cores \times \real^\params \to \real^\cores$ is an arbitrary function, possibly a ``black box'', of the nominal dynamic power $\vP_\dyn(\t)$, stochastic temperature $\vTO(\t, \o)$, and uncertain parameters $\vU(\o) = \oInvTransform{\vZ(\o)}$ (see \sref{uncertain-parameters}). In this work, $\f$ is assumed to be smooth in $\vZ(\o)$ and to have a finite variance, which is applicable to most physical systems \cite{xiu2002}. The definition of $\f$ is flexible enough to account for such effects as the interdependency between the leakage current and temperature \cite{srivastava2010, liu2007}, which is to be further discussed in \sref{illustrative-example} on a particular example. Note that, for brevity, we do not write explicitely the dependency of $\vP$ and $\vTO$ on $\vU(\omega)$ and use $\omega$ instead; also, since $\vU(\o) = \oInvTransform{\vZ(\o)}$, in the following, we shall use $\vZ(\o)$ instead of $\vU(\o)$.
