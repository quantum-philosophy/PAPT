Stochastic systems are often naturally parametrized with a set of independent \rvs\ \cite{xiu2010}. In general, however, the uncertain parameters $\vU(\o)$ can be correlated and should be preprocessed to extract a set of mutually independent \rvs. To this end, certain probability transformations should be undertaken. In this section, we outline a possible strategy in brief and refer the reader to \cite{xiu2010, eldred2009} for further details.

Linear correlations between \rvs\ can be removed by means of a linear transformation based on a factorization procedure of the covariance matrix of $\vU(\o)$; the technique is demonstrated in \sref{illustrative-example}. If $\vU(\o)$ are distributed normally, the absence of correlations is equal to independence. When the data are nearly normal, such approaches as the Box-Cox transformation are employed to straighten the normality assumption and apply the previous technique. In the general case, the most prominent solutions are the Rosenblatt and Nataf transformations \cite{eldred2009}. Rosenblatt's approach is suitable when the joint probability distribution function of $\vU(\o)$ is known; however, such information is rarely available. The marginal distributions and correlation matrix of $\vU(\o)$ are more likely to be given, which are sufficient to perform the Nataf transformation.

In the rest of the paper, $\vZ(\o) = \oTransform{\vU(\o)}$ denotes a vector of $\vars$ independent \rvs\ extracted from $\vU(\o)$ using an appropriate transformation $\oTransform{\idot}$. Let $(\outcomes, \sAlgebra, \pMeasure)$ be a complete probability space \cite{durrett2010} of $\vZ(\o)$ where $\outcomes$ is a set of outcomes, $\o \in \outcomes$, $\sAlgebra \subset 2^\outcomes$ is a $\sigma$-algebra on $\outcomes$, and $\pMeasure: \sAlgebra \to [0, 1]$ is a probability measure. Also, let $\PDF_\vZ(\vz)$ be the joint probability density (mass) function of $\vZ(\o)$ associated with the continuous (discrete) measure $\pMeasure$. Without loss of generality, $\oTransform{\idot}$ is assumed to center and normalize $\vZ(\o)$, i.e., $\oExp{\vZ(\o)} = \mZero$ and $\oCov{\vZ(\o)} = \mI$. Denote $\oInvTransform{\idot}$ the inverse transformation of $\oTransform{\idot}$, then $\vU(\o) = \oInvTransform{\vZ(\o)}$. In the following, we shall not write explicitly $\oInvTransform{\vZ(\o)}$ and use $\vZ(\o)$, $\vZ$, or $\o$ instead.
