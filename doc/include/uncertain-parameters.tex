Independence of the uncertain parameters is a prerequisite of the PC. Stochastic systems typically meet this requirement as they are often naturally parametrized \cite{xiu2010} with independent \rvs. In general, however, $\vU(\o)$ can be correlated and, therefore, should be preprocessed in order to fulfill the condition. To this end, an adequate probability transformation should be undertaken. Denote such a transformation by $\oTransform{\vU(\o)} = \vZ(\o): \real^\params \to \real^\vars$, which maps $\params$ correlated uncertain parameters $\vU(\o)$ onto $\vars$ independent \rvs\ $\vZ(\o)$. Let $(\outcomes, \sAlgebra, \pMeasure)$ be a complete probability space \cite{maitre2010} of $\vZ(\o)$ where $\outcomes$ is a set of outcomes, $\o \in \outcomes$, $\sAlgebra \subset 2^\outcomes$ is a $\sigma$-algebra on $\outcomes$, and $\pMeasure: \sAlgebra \to [0, 1]$ is a probability measure. Without loss of generality, $\oTransform{\idot}$ is assumed to center and normalize $\vZ(\o)$, i.e., $\oExp{\vZ(\o)} = \mZero$ and $\oCov{\vZ(\o)} = \mI$. Denote the inverse transformation to $\oTransform{\idot}$ by $\oInvTransform{\idot}$, then $\vU(\o) = \oInvTransform{\vZ(\o)}$ is the needed parametrization of the system in term of independent RVs suitable for the PC.

In this paragraph, we briefly outline a possible strategy of choosing $\oTransform{\idot}$ and refer the reader to \cite{xiu2010, eldred2009} for further details. Correlated multivariate normal \rvs\ can be transformed to independent ones via a linear transformation based on a factorization procedure of the covariance matrix of $\vU(\o)$; the technique is demonstrated in \sref{ie-uncertain-parameters}. When the data are nearly normal, such approaches as the Box-Cox transformation are employed to straighten the normality assumption and apply the previous technique. In the general case, the most prominent solutions are the Rosenblatt and Nataf transformations \cite{eldred2009}. Rosenblatt's approach is suitable when the joint probability distribution function of $\vU(\o)$ is known; however, such information is rarely available. The marginal distributions and correlation matrix of $\vU(\o)$ are more likely to be given, which are sufficient to perform the Nataf transformation.
