In the current scenario, the PC is to be founded on the Hermite polynomial basis as it is optimal for $\vZ(\o) \sim \normal{\vZero}{\mI}$ \cite{xiu2002}. To illustrate the basis, in the two-dimensional case the first Hermite polynomials $\pcb_i(\{ \Z_1(\o), \Z_2(\o) \})$ are
\[
  \begin{array}{lllllll}
  1, & \Z_1, & \Z_2, & \Z_1^2 - 1, & \Z_1 \Z_2, & \Z_2^2 - 1, & \dotsc
  \end{array}
\]
The goal now is to compute the coefficients of the expansion in \eref{pc-expansion}, which is a multidimensional integration problem. In numerical integration, an integral is typically approximated by a weighted sum of the function values computed at certain points known as nodes. A particular scheme to assign nodes and weights is called a quadrature rule \cite{press2007}. Since we need to compute expectations with respect to the Gaussian measure, the Gauss-Hermite quadrature rule is of a particular interest. The rule belongs to Gaussian quadrature rules where the order $\qdorder$, i.e., the number of points involved, and precision $\qdprecision$, i.e., the maximal total order of polynomials that the rule integrate exactly, are related as $\qdprecision = 2 \qdorder - 1$, which makes Gaussian quadratures especially efficient \cite{heiss2008}. Using the family of one-dimensional Gauss-Hermite quadrature rules, a multivariate quadrature rule is constructed. In low dimensions, the construction is based on the direct tensor produce of the one-dimensional rules; however, in high dimensions, the total number of nodes can easily explode. To overcome the problem, we employ the Smolyak algorithm, which constructs multidimensional grids in such a way that the accuracy of the underlying one-dimensional rules is preserved for complete polynomials whereas the number of nodes in the grid is significantly smaller than those of the tensor product \cite{eldred2009, maitre2010, heiss2008}. The $\vars$-variate quadrature rule of accuracy level $\qdlevel$ is defined as
\[
  \oQuad{\vars}{\qdlevel}{f} \eqdef \sum_{i = 1}^{\qdorder} f(\qdn{\vZ}_i) \qdw_i
\]
where $\qdorder$ is the order of the multivariate rule, $\qdn{\vZ}_i \in \real^\vars$ and $\qdw_i \in \real$ are the prescribed nodes and weights, respectively. The nodes are $\vars$-dimensional vectors, which corresponds to the number of uncertain parameters $\vZ(\o)$. The level of accuracy $\qdlevel$ should be chosen in such a way that the quadrature rule is exact for polynomials of order $2 \pcorder$ (twice the order of the PC expansion), which can be seen in \eref{pc-coefficients}. Therefore, $\qdlevel = \pcorder + 1$ according to the property of Gaussian quadratures discussed earlier. Consequently, \eref{pc-coefficients} is rewritten as
\[
  \pcc{\vP}_i(0) = \frac{1}{\pcn_i} \oQuad{\vars}{\qdlevel}{\vP(0, \o) \pcb_i(\vZ(\o))}
\]
where $\{ \pcn_i \}_{i = 1}^{\pcterms}$ are computed \emph{exactly} either by directly using the same quadrature rule or by taking products of one-dimensional counterparts with known analytical expressions \cite{xiu2010}; the result is further tabulated.
