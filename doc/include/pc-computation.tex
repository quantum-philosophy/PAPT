As discussed in \sref{preprocessing-rvs}, $\vZ(\o) \sim \normal{\vZero}{\mI}$, therefore, the Hermite polynomial basis is of a particular interest in this scenario. As an example, in the 2-dimensional case the first Hermite polynomials $\pcb_i(\{ \Z_1(\o), \Z_2(\o) \})$ are
\[
  \begin{array}{lllllll}
  1, & \Z_1, & \Z_2, & \Z_1^2 - 1, & \Z_1 \Z_2, & \Z_2^2 - 1, & \dotsc
  \end{array}
\]
The goal now is to compute the coefficients of the power expansion in \eref{pc-expansion}, which should be done according to \eref{inner-product-integral} since the Gaussian distribution is continuous.

In numerical integration, an integral of a function is approximated by a weighted sum of the function (integrand) values evaluated at certain points known as nodes. A particular scheme to choose nodes and weights is called a \definition{quadrature rule}. A one-dimensional quadrature rule is characterized by its order and precision. The order $\qdorder \in \natural{1}$ is the number of nodes employed in the rule whereas the precision $\qdprecision \in \natural{0}$ is the maximal order of polynomials that the rule integrates \emph{exactly}. A one-dimensional quadrature rule forms a family, which is defined as an indexed set of rules of this kind with increasing order. The index of a rule with a particular order in the family is known as its level (of accuracy) $\qdlevel \in \natural{0}$.

In multiple dimensions, multivariate quadrature rules are needed. One way to construct such rules is to use the tensor product of one-dimensional quadrature rules. In this case, the total number of nodes, which is the order of the new rule, is $\qdorder^\vars$ where $\vars$ is the number of dimensions. However, the precision $\qdprecision$ stays the same, i.e., the rule is exact for multivariate polynomials of \emph{total} order not grater than $\qdprecision$. Consequently, in the multidimensional scenario, the number of nodes can easily explode even for low precision requirements. For instance, if a one-dimensional rule requires only 3 nodes to integrate exactly 5-order polynomials, then in 20 dimensions the number of nodes grows up to $\sim 3 \cdot 10^9$. Moreover, if the integrand is a complete polynomial\footnote{A complete polynomial is a multivariate polynomial with a bounded \emph{total} order \cite{heiss2008}.}, most of the nodes in the tensor product are useless in the sense that they do not contribute to the resulting accuracy \cite{heiss2008}. Therefore, multivariate quadrature rules based on the tensor product are inefficient and often infeasible; deliberately constructed greeds of a sparse structure are desired.

One of the most successful and wide-spread technique to construct a sparse grid was proposed by Sergey Smolyak (1963) and is known now by his name. Given a family of one-dimensional quadrature rules, the Smolyak algorithm constructs a multidimensional grid in such a way that the accuracy of the underlaying rules is preserved for complete polynomials whereas the number of nodes in the grid is significantly smaller than those of the tensor product \cite{heiss2008, eldred2009}. Consequently, the algorithm mitigates the well-known ``curse of dimensionality'' mentioned earlier.

Let us now turn the theory above into practice. Since in the example under consideration, we need to compute expectations (integrals) with respect to the Gaussian measure, one quadrature rule is of a particular interest for us. The rule is known as the Gauss-Hermite quadrature rule where the nodes are the roots of the Hermite polynomials and the weight function is $e^{-\beta (x - \alpha)^2}$. The rule belongs to Gaussian quadrature rules where the order $\qdorder$ and precision $\qdprecision$ are related as $\qdprecision = 2 \qdorder - 1$ making Gaussian quadratures especially efficient \cite{heiss2008}. Using the family of one-dimensional Gauss-Hermite quadrature rules and the Smolyak algorithm, the $\vars$-variate quadrature rule of accuracy level $\qdlevel$ is constructed. Define the rule as
\[
  \oQuad{\vars}{\qdlevel}{f} \eqdef \sum_{i = 1}^{\qdorder} f(\qdn{\vZ}_i) \qdw_i
\]
where $\qdorder$ is the order of the multivariate rule, $\qdn{\vZ}_i \in \real^\vars$ and $\qdw_i \in \real$ are the prescribed nodes and weights, respectively. The nodes are $\vars$-dimensional vectors, which corresponds the number of uncertain parameters $\vZ(\o)$. The level of accuracy $\qdlevel$ should be chosen in such a way that the quadrature rule is exact for polynomials of order $2 \pcorder$ (twice the order of the gPC expansion), which can be seen in \eref{pc-coefficients}. Therefore, $\qdlevel = \pcorder + 1$ due to the nature of Gaussian quadrature discussed earlier. Consequently, \eref{pc-coefficients} is rewritten as
\[
  \pcc{\vP}_i(0) = \frac{1}{\pcn_i} \oQuad{\vars}{\qdlevel}{\vP(0, \o) \pcb_i(\vZ(\o))}
\]
where $\{ \pcn_i \}_{i = 1}^{\pcterms}$ can be \emph{exactly} computed using the same quadrature rule and further tabulated for the Hermite polynomial basis.
