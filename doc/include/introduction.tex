Over the last decade, the concept of uncertainty quantification (UQ) has become central of a wide range of research areas \cite{xiu2010}. The primary target of UQ is evaluation of the output of the systems that exhibit non-deterministic behavior due to the presence of uncertainties of some kind \cite{eldred2009}. A multiprocessor platform is a prominent example, wherein the variability originates from, for instance, the semiconductor manufacturing process and operating environment. In particular, the uncertainty impacts the power dissipation and, consequently, temperature, which are among the main concerns of multiprocessor architects. At the same time, in the literature, one can observe that the majority of the studies involving power-temperature analysis (PTA) are based on deterministic solutions, e.g., \cite{ukhov2012}; therefore, the reliability of the corresponding results can be questionable in practice due to the negligence of variations.

To overcome the outlined limitation of the deterministic PTA, a number of stochastic PTA techniques have been recently introduced. In \cite{juan2011}, a learning-based approach is presented to estimate the maximal steady-state temperature of multiprocessor platforms under the variability of the leakage power. The last is also considered in \cite{juan2012}, where a statistical model of the steady-state temperature based on the Gaussian distribution and Box-Cox transformation is derived. The aforementioned techniques are constrained by the necessity of the steady-state condition. However, when the knowledge of the transient temperature is desired, or the steady-state assumption is not satisfied, the steady-state PTA is of little use; the stochastic transient PTA is to be performed. Moreover, even though some variations of the fabrication parameters can be approximated by normal random variables (\rvs), the assumption that the leakage power follows a known distribution is debatable due to (a) the strict nonlinearities between the process parameters and leakage; (b) the nonlinear interdependency of the leakage current and temperature \cite{liu2007}. Hence, more flexible stochastic PTA techniques are needed.

A straightforward approach to analyze a stochastic system is the Monte Carlo (MC) sampling coupled with a deterministic simulator. MC-based techniques possess a unique feature: they are insensible to the number of stochastic dimensions \cite{maitre2010}, which makes them advantageous when a large number of \rvs\ are involved. However, the major problem with the MC sampling is in the low rate of convergence, e.g., the mean value converges as $\mcsamples^{-\ifrac{1}{2}}$ where $\mcsamples$ is the number of samples \cite{xiu2010, maitre2010}. A sample, in this case, is a complete realization of the system, which makes MC-based methods slow and often infeasible since the number of such simulations should be considerably high to obtain reliable estimates.

Attractive alternatives to the MC sampling are spectral methods \cite{xiu2010, maitre2010, ghanem1991} featuring a much faster convergence. One of such methods is the subject of this paper; namely, the polynomial chaos (PC) is considered, which is the current state of the art of numerical analysis of stochastic systems \cite{xiu2010}. The PC is valuable for the purpose of UQ due to its strong mathematical background and ability to construct analytical representations of uncertain quantities \cite{eldred2009}. The PC and other spectral methods have been successfully employed to quantify various stochastic systems \cite{xiu2010}. In \cite{bhardwaj2006}, the Karhunen-Lo\`{e}ve (KL) expansion, which is closely related to the PC, is used to estimate the leakage current of electrical circuits. In \cite{shen2009}, the full-chip leakage estimation is achieved by means of the PC expansion based on the Hermite polynomials. An analysis of the voltage response of large power grids is carried out in \cite{ghanta2006}, where the KL and PC expansions are jointly applied.

The contribution of this paper is in the following. We develop a PC-based framework for UQ of power and temperature traces of multiprocessor systems. The total power dissipation is modeled as an arbitrary transformation of temperature and a set of uncertain parameters. The latter are assumed to have known probability distributions and correlation structures; however, there are no a priori assumptions about the resulting distributions of power and temperature. Based on a nominal dynamic power profile, our technique produces the corresponding stochastic power and temperature profiles defined in terms of orthogonal polynomials of \rvs; the expressions are straightforward to be further analyzed. The framework is illustrated on the variations of the subthreshold leakage current \cite{srivastava2010}, which is followed by an assessment of the efficiency and accuracy of our technique using a MC sampling approach.

The reminder of the paper is organized as follows. In \sref{preliminaries}, we introduce the architecture model that we shall consider and formulate the objective of our study. The proposed framework is presented in \sref{proposed-framework}. \sref{illustrative-example} illustrates our technique on a particular example, which is further discussed in \sref{experimental-results}, where a comparison with a MC-based approach is reported. Finally, \sref{conclusion} concludes the paper.
