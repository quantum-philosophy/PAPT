The variations in the dynamic power due to the process variation are negligible whereas the leakage power is the major headache of the designer \cite{juan2012, srivastava2010}.

It is widely accepted that the leakage current has a log-normal distribution, cf. \cite{juan2012, srivastava2010}. The sub-threshold leakage is the most sensible to the process variation while the gate leakage has become less important with the introduction of high-k dielectrics \cite{juan2012}.

The most commonly used approach to perform the \STempA is the Monte Carlo sampling (MCS) coupled with a deterministic thermal simulator, such as \cite{huang2006}. The problem is in the low rate of convergence, e.g., the mean value converges as $1/\sqrt{n}$ where $n$ is the number of simulations \cite{xiu2009}. Therefore, the MCS requires an unaffordably large number of realizations of the system to be performed.

A possible way to speed up the analysis is to employ the model-order reduction (MOR) techniques \cite{benner2011}, which can dramatically decrease the number of state variables of the system. These techniques can be considered as a preprocessing step and, therefore, they are applicable to all the above-mentioned methods.
