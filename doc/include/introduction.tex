Over the last decade, the concept of uncertainty quantification (UQ) has become central for a wide range of application areas \cite{xiu2010}. The primary target of UQ is evaluation of the output of systems that exhibit non-deterministic behavior due to the presence of uncertainties of some kind \cite{eldred2009}. A multiprocessor platform is a prominent example of such a system, wherein the variability originates from, for instance, the semiconductor manufacturing process and operating environment. In particular, the uncertainty impacts the power dissipation and, consequently, temperature, which are among the main concerns of multiprocessor system designs. At the same time, one can observe that the majority of the literature involving system-level power-temperature analysis (PTA) ignores these important aspects, e.g., \cite{rao2009, liu2010, rai2011, thiele2011, ukhov2012}; therefore, the reliability of the corresponding results is questionable in practice.

To overcome the limitations of the deterministic PTA, a number of stochastic PTA techniques have been recently introduced. A solely power-targeted but temperature-aware solution is proposed in \cite{chandra2010}, which employs Monte Carlo (MC) simulations to account for process and environmental variations. A learning-based approach is presented in \cite{juan2011} to estimate the maximal steady-state temperature of multiprocessor platforms under the variability of leakage. The leakage variations are also considered in \cite{juan2012}, where a statistical model of the steady-state temperature based on the Gaussian distribution is derived. None of the aforementioned techniques attempts to perform the stochastic \emph{transient} PTA and to compute the evolving-in-time probability distribution of temperature. However, such transient curves are of practical importance. First of all, certain procedures cannot be undertaken without the knowledge of time-dependent temperature variations, e.g., the reliability optimization based on the thermal-cycling fatigue \cite{ukhov2012}. Secondly, when the constant steady-state temperature assumption, as in \cite{juan2011, juan2012}, fails---which is often the case since power profiles are not invariant in reality, and the thermal time constant of on-chip components is small enough to make temperature follow the power pattern---the steady-state PTA is of little use, and the transient counterpart is to be performed. Furthermore, normal distributions, as in \cite{juan2012}, do not cover all possible scenarios of variability; other distributions can be preferable for certain situations, e.g., the beta distribution is a typical choice for uncertainties with bounded supports \cite{maitre2010}. Apart from the limitations mentioned above, power is frequently assumed to follow a priori known probability distributions, e.g., the log-normal distribution is a popular choice for leakage \cite{srivastava2010}. However, such an assumption is debatable due to (a) the strict nonlinearities between the process parameters, power, and temperature; (b) the nonlinear interdependency of temperature and the leakage power \cite{liu2007}. To conclude, the present models of uncertainty and the stochastic PTA techniques for multiprocessor system design are restricted in use due to one or several of the following traits: based on MC simulations (potentially slow) \cite{chandra2010}, limited to power analysis \cite{chandra2010}, limited to the constant steady-state temperature \cite{juan2011, juan2012}, exclusive focus on the maximal temperature \cite{juan2011}, the normality assumption of variations \cite{juan2012}, a priori chosen distributions of power \cite{srivastava2010}. Hence, more flexible stochastic PTA techniques are needed.

A straightforward approach---which, as mentioned earlier, is utilized in \cite{chandra2010}---to analyze a stochastic system is the MC sampling coupled with a deterministic simulator. The major problem with the MC sampling is the low rate of convergence, e.g., the mean value converges as $\mcsamples^{-\ifrac{1}{2}}$ where $\mcsamples$ is the number of samples \cite{xiu2010, maitre2010}. A sample is a complete realization of the system, which makes MC-based methods slow and often infeasible since the number of needed simulations can be extremely high in order to obtain reliable estimates.

Attractive alternatives to the MC sampling are spectral methods \cite{xiu2010, maitre2010, ghanem1991} featuring a much faster convergence. One such method is applied in this paper, namely, the generalized polynomial chaos (PC), which is the current state of the art in numerical analysis of stochastic systems \cite{xiu2010}. The PC is adequate for the purpose of UQ due to its flexibility in modeling various systems and, more importantly, in ability to construct easy-to-analyze representations of uncertainty \cite{eldred2009}. In \cite{shen2009}, the PC based on the Hermite polynomials is employed to estimate the full-chip leakage. The Karhunen-Lo\`{e}ve (KL) expansion, which is closely related to the PC, is used in \cite{bhardwaj2006} to calculate the leakage current of electrical circuits. An analysis of the voltage response of large power grids is carried out in \cite{ghanta2006}, where the KL and PC expansions are jointly utilized.

Our contribution is in the following: we develop, for the first time, a framework for UQ of power and temperature traces of multiprocessor systems. The total power dissipation is modeled as an arbitrary transformation of temperature and a set of uncertain parameters. The uncertain parameters are assumed to have given probability distributions and correlation structures, and there are no a priori assumptions about the resulting distributions of power and temperature. Given a nominal dynamic power profile, our technique produces the corresponding stochastic power and temperature profiles defined in terms of polynomials of random variables (\rvs); the expressions are straightforward to be further analyzed. The framework is illustrated on one of the most important parameters affected by process variation: the leakage current. However, it is worth being noted that, in contrast to the majority of the literature mentioned above, our approach is not bounded to any particular source of variability.

The reminder of the paper is organized as follows. In \sref{preliminaries}, we introduce the architecture model that we shall consider and formulate the objective of our study. The proposed framework is presented in \sref{proposed-framework}. \sref{illustrative-example} illustrates our technique on a particular example, which is further discussed in \sref{experimental-results}, where a comparison with a MC-based approach is reported. Finally, \sref{conclusion} concludes the paper.
