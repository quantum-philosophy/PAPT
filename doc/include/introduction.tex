The variations in the dynamic power due to the process variation are negligible whereas the leakage power is the major headache of the designer \cite{juan2012, srivastava2010}.

It is widely accepted that the leakage current has a log-normal distribution, cf. \cite{juan2012, srivastava2010}. The sub-threshold leakage is the most sensible to the process variation while the gate leakage has become less important with the introduction of high-k dielectrics \cite{juan2012}.

The most commonly used approach to perform the \sta\ is Monte Carlo sampling (MCS) techniques coupled with a deterministic thermal simulator. The problem is in the low rate of convergence, e.g., the mean value converges as $1/\sqrt{n}$ where $n$ is the number of simulations \cite{xiu2009}. Therefore, the MCS requires an unaffordably large number of realizations of the system to be performed.

In this work, we use a mathematical framework called the generalized polynomial chaos (PC), which is the current state-of-the-art of numerical analysis of stochastic systems \cite{xiu2009, xiu2002}. The core of the PC is the decomposition of stochastic processes into infinite sums of orthogonal polynomials of random variables (r.v.'s) that significantly eases the subsequent analysis. As its name suggests, the generalized PC is a generalization of the original PC where only Hermite polynomials and Gaussian r.v.'s were considered \cite{ghanem1991}. In contrast, the PC employs a broad family of orthogonal polynomials from the Askey scheme, which the Hermite polynomials are a subset of, and handles different probability distributions. It can be shown that \emph{any} functional with a finite variance can be approximated with a PC expansion, and this expansion converges in the $\L{2}$ (mean-square) sense.

In \cite{shen2009}, the PC is applied to estimate the full-chip leakage current.

The output of the proposed framework constitutes the stochastic power profile of the multiprocessor system and the corresponding stochastic temperature profile.

A possible way to speed up the analysis is to employ the model-order reduction (MOR) techniques \cite{benner2011}, which can dramatically decrease the number of state variables of the system. These techniques can be considered as a preprocessing step and, therefore, they are applicable to all the above-mentioned methods.

The rest of the paper is organazed as follows.
