Over the last decade the concept of uncertainty quantification (UQ) has drawn a lot of attention in an astonishingly comprehensive range of research areas \cite{xiu2009}. The primary target of UQ techniques is the evaluation of the output of the systems that exhibit non-deterministic behaviour due to an internal/external interference of uncertainties of some kind \cite{eldred2009}. The reason for this rapid growth of interest is clear --- the capacity of idealized models, in which everything is assumed to be purely deterministic, have been entirely depleted by the ever demanding technological progress. The results of such models are no longer acceptable since they inherently lack an essential portion of reality, viz. its random nature.

Multiprocessor platforms, as regular physical systems at the first place, are not delivered from the fate of uncertainty. Often variability is conatural and unavoidable; the uncertainties due to the semiconductor manufacturing process and operating environment are the most prominent examples. Consequently, the fabrication of robust products requires a proper modeling of the immanent variations, specially at the design stage.

One of the major concerns of multiprocessor architects are temperature and its close counterpart power. In the contemporary literature, one can notice that most of the studies, which involve thermal modeling, are based on deterministic solutions where the input parameters are assumed to be constant, e.g., \cite{ukhov2012}. However, the reliability of such estimates can be questionable in practice due to the negligence of physical uncertainties. Sequentially, the overall quality of frameworks based on the deterministic temperature analysis (TA) can be severely deteriorated. The stochastic TA is highly needed.

The most straight-forward way to perform the stochastic TA is to employ a Monte Carlo (MC) sampling technique coupled with a deterministic thermal simulator. MC-based approaches possess a few unique features, which make them preferable for certain situations. First, such techniques are easy to implement since existing deterministic solves can be readily employed without any modifications. Secondly, and more importantly, the MC sampling is insensible to the number of stochastic dimenstions \cite{maitre2010}, therefore, MC techniques can be advantageous for systems that involve large numbers of random variables (r.v.'s). However, the major problem with the MC sampling is in the low rate of convergence, e.g., the mean value is known to converge as $\mcsamples^{-\ifrac{1}{2}}$ where $\mcsamples$ is the number of samples \cite{xiu2009, maitre2010}. One sample in this case is a complete realization of the whole thermal system, which makes MC-based methods slow and often infeasible to undertake since the number of such simulations should be considerably high (in the order of $10^4$) to obtain representative results.

Attractive alternatives to the MC sampling are spectral methods \cite{maitre2010}, which demonstrate much faster convergence properties, and one of these methods is the subject of this paper. Namely, we use a probabilistic framework called the polynomial chaos (PC), which constinutes the current state of the art of numerical analysis of stochastic systems \cite{xiu2009}. The PC is especially valuable for the purpose of UQ due to its strong mathemetical background and ability to construct \emph{analytical} representations of uncertain quantities \cite{eldred2009}. To elaborate, the core of the PC is the decomposition of stochastic processes into infinite sums of orthogonal polynomials of r.v.'s. It can be shown that \emph{any} functional with a finite variance can be approximated with such a PC expansion, and this expansion converges in the $\L{2}$ (mean-square) sense. Once constructed, PC series are trivial for a broad range of subsequent analyses; see \cite{eldred2009, maitre2010}. The PC employed here is known as the generalized PC \cite{xiu2002}, which is an extension of the original PC where only Hermite polynomials and Gaussian r.v.'s were considered \cite{ghanem1991}. In contrast, the generalized PC covers the whole family of hypergeometric orthogonal polynomials from the Askey scheme, which the Hermite polynomials are a subset of, and handles different probability distributions. Even though we focus on the generalized PC, the proposed framework can be easily extended to other bases; refer to \cite{maitre2010} for an overview.

Spectral methods have been successfully employed to quantify an ample range of diverse stochastic systems \cite{xiu2010}. Within the area of computer systems, much attention has been devoted to modeling the leakage current. In \cite{bhardwaj2006}, the authors use the Karhunen-Lo\`{e}ve (KL) expansion \cite{loeve1978} to estimate leakage of electrical circuits based on covariance functions of process parameters. In \cite{shen2009}, the full-chip leakage estimation is achieved by means of the Hermite PC (i.e., the original PC). The analysis of large power grids aimed to characterize the voltage response is carried out in \cite{ghanta2006} where the KL and Hermite PC expansions are used.

In this work, we utilize the PC to quantify power and the resulting temperature of stochastic multiprocessor platforms. For the demonstration purposes, we apply our approach to a particular problem where the overall process is illustrated step by step. To be exact, the leakage power subjected to the process variation is addressed. The leakage current has a complex dependency structure on the intrisict fabrication parameters, which are uncertain due to the inperfection of the semiconductor matufacturing process, cf. \cite{juan2012, srivastava2010}. Moreover, since the leakage power and temperature have a strong, nonlinear interdependency, the goal of the stochastic TA becomes even more challenging.

The rest of the paper is organazed as follows. \sref{preliminaries} overviews basic notations used here and introduces the architecture model that we shall consider. In \sref{problem-formulation}, we briefly outline the overall goal of our research. \sref{proposed-framework} is the key part where the proposed framework is presented. In \sref{illustrative-example}, we illustrate our technique on the leakage power subjected to the process variation. Numerial results in terms of computational speed and accuracy are given in \sref{experimental-results} where a MC technique is employed to assess the proposed framework for the example from \sref{illustrative-example}. Finally, \sref{conclusion} concludes the paper.
