Over the last decades the concept of uncertainty quantification (UQ) has gained much attention in an astonishingly wide range of research areas; see \cite{xiu2009}. The primary target of UQ techniques is the evaluation of the output of the systems that exhibit non-deterministic behaviour due to internal and/or external interferences of uncertainties of some kind \cite{eldred2009}. The reason for this rapidly growing attention is clear --- the capacity of idealized models, in which everything is assumed to be purely deterministic, have been entirely depleted, and the results that such models yield cannot be percieved as acceptable by the ever demanding technological progress.

Multiprocessor platforms, being regular physical systems at the first place, are not delivered from this fate of uncertainty. Often variations are inherent and unavoidable; the uncertainties due to the manufacturing process and operating environment are the most prominent examples. The only way to handle such uncertainties and fabricate a robust product is to properly model them during the design stage.

One of the major concerns of multiprocessor architects are temperature and its close counterpart power. In the contemporary literature, one can notice that most of the studies, which involve thermal modeling, are based on deterministic solutions where the input parameters are assumed to be constant, e.g., \cite{ukhov2012}. However, the reliability of such estimates can be questionable in practice due to the negligence of physical uncertainties. Consequently, the overall quality of frameworks based on the deterministic temperature analysis (TA) can be severely deteriorated.

The most straight-forward approach to perform the stochastic TA is to employ a Monte Carlo (MC) sampling technique coupled with a deterministic thermal simulator. The problem here is in the low rate of convergence, e.g., the mean value is known to converge as $1/\sqrt{n}$ where $n$ is the number of simulations \cite{xiu2009}. A simulation in this case is a complete realization of the whole thermal system, therefore, MC-based methods typically take a huge mount of computational time and often are simply infeasible to undertake.

In this work, we use a mathematical framework called the polynomial chaos (PC), which is the current state of the art of numerical analysis of stochastic systems \cite{xiu2009}. The PC is especially attractive for the purpose of UQ due to its strong mathemetical background and ability to construct \emph{analytical} representations of uncertain quantities \cite{eldred2009}. To be exact, the core of the PC is the decomposition of stochastic processes into infinite sums of orthogonal polynomials of random variables (r.v.'s). It can be shown that \emph{any} functional with a finite variance can be approximated with such a PC expansion, and this expansion converges in the $\L{2}$ (mean-square) sense. Once constructed, PC series are trivial from the perspective of a wide range of subsequent analyses of different kind \cite{eldred2009}. The PC employed here is known as the generalized PC \cite{xiu2002}, which is an extension of the original PC where only Hermite polynomials and Gaussian r.v.'s were considered \cite{ghanem1991}. In contrast, the generalized PC covers a broad family of hypergeometric orthogonal polynomials from the Askey scheme, which the Hermite polynomials are a subset of, and handles different probability distributions. Even though we specifically focus on the generalized PC, the proposed framework can be easily extended to other bases; refer to \cite{maitre2010} for an overview.

It is widely accepted that the leakage current has a log-normal distribution, cf. \cite{juan2012, srivastava2010}. The sub-threshold leakage is the most sensible to the process variation while the gate leakage has become less important with the introduction of high-k dielectrics \cite{juan2012}.

The variations in the dynamic power due to the process variation are negligible whereas the leakage power is the major headache of the designer \cite{juan2012, srivastava2010}.

In \cite{shen2009}, the PC is applied to estimate the full-chip leakage current.

The output of the proposed framework constitutes the stochastic power profile of the multiprocessor system and the corresponding stochastic temperature profile.

A possible way to speed up the analysis is to employ the model-order reduction (MOR) techniques \cite{benner2011}, which can dramatically decrease the number of state variables of the system. These techniques can be considered as a preprocessing step and, therefore, they are applicable to all the above-mentioned methods.

The rest of the paper is organazed as follows.
