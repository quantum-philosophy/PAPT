Over the last decades the concept of uncertainty quantification (UQ) has gained much attention in an astonishingly wide range of research areas; see \cite{xiu2009}. The primary target of UQ techniques is the evaluation of the output of the systems that exhibit non-deterministic behaviour due to internal and/or external interferences of uncertainties of some kind \cite{eldred2009}. The reason for this rapidly growing attention is clear --- the capacity of idealized models, in which everything is assumed to be purely deterministic, have been entirely depleted, and the results that such models yield cannot be percieved as acceptable by the ever demanding technological progress.

Multiprocessor platforms, being regular physical systems at the first place, are not delivered from this fate of uncertainty. Often variations are inherent and unavoidable; the uncertainties due to the manufacturing process and operating environment are the most prominent examples. The only way to handle such uncertainties and fabricate a robust product is to properly model them during the design stage.

One of the major concerns of multiprocessor architects are temperature and its close counterpart power. In the contemporary literature, one can notice that most of the studies, which involve thermal modeling, are based on deterministic solutions where the input parameters are assumed to be constant, e.g., \cite{ukhov2012}. However, the reliability of such estimates can be questionable in practice due to the negligence of physical uncertainties. Consequently, the overall quality of frameworks based on the deterministic temperature analysis (TA) can be severely deteriorated.

The most straight-forward way to perform the stochastic TA is to employ a Monte Carlo sampling (MCS) technique coupled with a deterministic thermal simulator. MCS-based approaches possess a few unique features, which make them preferable for certain situations. Firstly, such techniques are easy to implement since existing deterministic solves can be readily employed without any modifications. Secondly, and more importantly, the MCS is insensible to the number of stochastic dimenstions \cite{maitre2010}, therefore, MCS techniques can be advantageous for systems that involve large numbers of random variables (r.v.'s). However, the major problem with the MCS is in the low rate of convergence, e.g., the mean value is known to converge as $1/\sqrt{n}$ where $n$ is the number of simulations \cite{xiu2009, maitre2010}. One simulation in this case is a complete realization of the whole thermal system, which makes MC-based methods slow and often infeasible to undertake since the number of such simulations should be considerably high (in the order of $10^4$) to obtain representative results.

Attractive alternatives to the MCS are spectral methods \cite{maitre2010}, which demonstrate much faster convergence properties, and one of these methods is the subject of this paper. Namely, we use a probabilistic framework called the polynomial chaos (PC), which constinutes the current state of the art of numerical analysis of stochastic systems \cite{xiu2009}. The PC is especially valuable for the purpose of UQ due to its strong mathemetical background and ability to construct \emph{analytical} representations of uncertain quantities \cite{eldred2009}. To elaborate, the core of the PC is the decomposition of stochastic processes into infinite sums of orthogonal polynomials of r.v.'s. It can be shown that \emph{any} functional with a finite variance can be approximated with such a PC expansion, and this expansion converges in the $\L{2}$ (mean-square) sense. Once constructed, PC series are trivial for a broad range of subsequent analyses; see \cite{eldred2009, maitre2010}. The PC employed here is known as the generalized PC \cite{xiu2002}, which is an extension of the original PC where only Hermite polynomials and Gaussian r.v.'s were considered \cite{ghanem1991}. In contrast, the generalized PC covers the family of hypergeometric orthogonal polynomials from the Askey scheme, which the Hermite polynomials are a subset of, and handles different probability distributions. Even though we focus on the generalized PC, the proposed framework can be easily extended to other bases; refer to \cite{maitre2010} for an overview.

Spectral methods have been successfully employed to quantify an ample range of diverse stochastic systems \cite{xiu2010}. Within the area of computer systems, much attention has been devoted to modeling the leakage current. In \cite{bhardwaj2006}, the authors use the Karhunen-Lo\`{e}ve (KL) expansion \cite{loeve1978} to estimate leakage of electrical circuits based on covariance functions of process parameters. In \cite{shen2009}, the full-chip leakage estimation is achieved by means of the Hermite PC. The analysis of large power grids aimed to characterize the voltage response is carried out in \cite{ghanta2006} where the KL and Hermite PC expansions are used.

For the demonstration purposes, we apply our approach to a particular application and illustrate the overall process of the stochastic temperature and power analysis step by step. To be exact, the leakage power subjected to the process variation is addressed. The leakage current has a complex dependency structure on the intrisict fabrication parameters, which, due to the inperfection of the matufacturing process, are well-known to be uncertain, cf. \cite{juan2012, srivastava2010}. Moreover, since the leakage power and temperature have a strong, nonlinear inderdependency, the goal of the TA becomes even more severe to achieve. The output of the proposed framework constitutes the stochastic power profile of the multiprocessor system and the corresponding stochastic temperature profile.

The rest of the paper is organazed as follows. \sref{preliminaries} overviews basic notations used here and introduces the architecture model that we shall consider. In \sref{problem-formulation}, we briefly outline the overall goal of our research. \sref{proposed-framework} is the key part where the proposed framework is presented. In \sref{illustrative-example}, we illustrate our technique on a particular example, namely, the leakage power subjected to the process variation is addressed. Numerial results in terms of computational speed and accuracy are given in \sref{experimental-results} where a MCS technique is employed to assess the proposed framework for the example from \sref{illustrative-example}. Finally, \sref{conclusion} concludes the paper.
