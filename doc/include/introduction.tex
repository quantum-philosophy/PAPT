The concept of uncertainty quantification (UQ) has drawn a lot of attention in an astonishingly wide range of research areas \cite{xiu2009}. The primary target of UQ is the evaluation of the output of the systems that exhibit a non-deterministic behavior due to the presence of uncertainties of some kind \cite{eldred2009}. The reason for this rapid growth of interest is clear: the capacity of purely deterministic models have been severely depleted; the results of such models are no longer acceptable as they fail to properly reflect the underlying physical processes.

Multiprocessor platforms, as regular physical systems at the first place, are not delivered from the fate of uncertainty. One of the major concerns of multiprocessor architects are temperature and its close counterpart power. In the contemporary literature, one can notice that most of the studies that involve temperature analysis (TA), both steady-state and transient, are based on deterministic solutions, e.g., \cite{ukhov2012}. However, the reliability of such estimates can be questionable in practice due to the negligence of the unavoidable variability; the uncertainties due to the semiconductor manufacturing process and operating environment are the most prominent examples.

To overcome the outlined limitation of the deterministic TA, the stochastic TA is to be developed. A number of such techniques have been recently introduced. In \cite{juan2011}, a learning-based approach is presented to estimate the maximal steady-state temperature of 3D ICs under the variability of the leakage power. Another procedure to account for the uncertainties of leakage is proposed in \cite{juan2012} where a statistical model of the steady-state temperature based on the Gaussian distribution and Box-Cox transformation \cite{box1964} is derived. The aforementioned techniques are constrained by the necessity of the \emph{steady-state} condition. However, when the knowledge of the transient temperature is desired, or the steady-state assumption does not hold, the steady-state TA is of no use; the stochastic \emph{transient} TA is to be performed, which is a considerably more problematic task. In fact, the steady-state TA is a special case of the transient TA where power does not change, and time approaches infinity. Moreover, even though fabrication parameters are generally accepted to have normal distributions, the normality of the leakage power, even after the Box-Cox transformation, is debatable due to the presence of strict nonlinearities between the process parameters, leakage, and temperature \cite{liu2007}; the later two, in addition, form a closed loop with a positive feedback. Consequently, flexible approaches for the stochastic TA, which are able to handle the transient case and make no assumtions about the resulting distributions of power and, consequently, temperature, are highly needed.

The most straight-forward way to perform the stochastic TA of any kind is to employ a Monte Carlo (MC) sampling technique coupled with a deterministic temperature simulator. MC-based approaches possess a few unique features, which make them preferable for certain situations. First, such techniques are easy to implement since existing deterministic solves can be readily employed. Secondly, and more importantly, the MC sampling is insensible to the number of stochastic dimensions \cite{maitre2010}; therefore, MC techniques can be advantageous for systems that involve large numbers of random variables. However, the major problem with the MC sampling is in the low rate of convergence, e.g., the mean value is known to converge as $\mcsamples^{-\ifrac{1}{2}}$ where $\mcsamples$ is the number of samples \cite{xiu2009, maitre2010}. One sample in this case is a complete realization of the whole thermal system, which makes MC-based methods slow and often infeasible since the number of such simulations should be considerably high to obtain reliable results.

Attractive alternatives to the MC sampling are spectral methods \cite{maitre2010}, which demonstrate a much faster convergence. One of these methods is a probabilistic framework called the polynomial chaos (PC), which constitutes the current state of the art of numerical analysis of stochastic systems \cite{xiu2009}. The PC is especially valuable for the purpose of UQ due to its strong mathematical background and ability to construct \emph{analytical} representations of uncertain quantities \cite{eldred2009}. The constructed expansions are essentially easy for the subsequent analyses.

The PC and other spectral methods have been successfully employed to quantify an ample range of diverse stochastic systems \cite{xiu2010}. In \cite{bhardwaj2006}, the authors use the Karhunen-Lo\`{e}ve (KL) expansion \cite{loeve1978}, which is closely related to the PC, to estimate leakage of electrical circuits based on covariance functions of process parameters. In \cite{shen2009}, the full-chip leakage estimation is achieved by means of the PC expansion with the Hermite polynomials. The analysis of large power grids aimed to characterize the voltage response is carried out in \cite{ghanta2006} where the KL and Hermite PC expansions are used.

In this work, we utilize the PC to develop a framework to quantify temperature and power of stochastic multiprocessor systems. The framework is demonstrated on a particular example, in which the variability of temperature affected by the uncertain leakage power is modeled. In this case, the stochastic TA is complicated by the following facts. First, the leakage current has a nonlinear dependency structure on the intrinsic fabrication parameters, which are uncertain due to the imperfection of the semiconductor manufacturing process \cite{juan2011, juan2012, srivastava2010}. Secondly, the leakage power and temperature have a strong, nonlinear interdependency \cite{liu2007}, which makes the stochastic TA even more challenging. Nevertheless, the proposed framework shows efficiency and accuracy in solving such problems as it is verified by a MC sampling technique.

The rest of the paper is organized as follows. \sref{preliminaries} overviews the basic notations and introduces the architecture model that we shall consider. In \sref{problem-formulation}, we briefly outline the objective of our study. The proposed framework is presented in \sref{proposed-framework}. \sref{illustrative-example} illustrates our technique on a particular example, which is discussed in figures in \sref{experimental-results}. Finally, \sref{conclusion} concludes the paper.
