Over the last decade, the concept of uncertainty quantification (UQ) has become central for a wide range of application areas \cite{xiu2010}. The primary target of UQ is characterization of the output of systems that exhibit non-deterministic behavior due to the presence of uncertainties of some kind \cite{eldred2009}. A multiprocessor platform is a prominent example of such a system, wherein the variability originates from, for instance, the semiconductor manufacturing process and operating environment. In particular, the uncertainty impacts power and, consequently, temperature, which are among the main concerns of multiprocessor system designs. At the same time, one can observe that the majority of the literature involving system-level power-temperature analysis (PTA) ignores these important aspects, \eg, \cite{rao2009, rai2011, thiele2011, ukhov2012}; therefore, the reliability of the corresponding results is questionable in practice.

To overcome the limitations of the deterministic PTA, a number of stochastic PTA techniques have been recently introduced. A solely power-targeted but temperature-aware solution is proposed in \cite{chandra2010}, which employs Monte Carlo (MC) simulations to account for process and environmental variations of multiprocessor systems. A learning-based approach is presented in \cite{juan2011} to estimate the maximal temperature under the steady-state condition and variability of the leakage current. Leakage is also considered in \cite{juan2012}, where a statistical model of the steady-state temperature based on Gaussian distributions is derived. None of the aforementioned techniques attempts to perform the stochastic \emph{transient} PTA and to compute the evolving-in-time probability distribution of temperature. However, such transient curves are of practical importance. First of all, certain procedures cannot be undertaken without the knowledge of time-dependent temperature variations, \eg, the reliability optimization based on the thermal-cycling fatigue \cite{ukhov2012}. Secondly, when the steady-state assumption, considered, \eg, in \cite{juan2011, juan2012}, fails---which is often the case since power profiles are not invariant in reality, and the thermal time constant of on-chip components is small enough to make temperature follow the power pattern---the steady-state PTA is of little use, and the transient counterpart is to be performed. Furthermore, the frequently made assumption that power and/or temperature follow \apriori\ known probability distributions---for instance, Gaussian and log-normal distributions are popular choices, as in \cite{juan2012, srivastava2010}---is debatable due to (a) the strict nonlinearities between the process parameters, power, and temperature; (b) the nonlinear interdependency of temperature and the leakage power \cite{liu2007}. To conclude, the present models of uncertainty and the stochastic PTA techniques for multiprocessor system design are restricted in use due to one or several of the following traits: based on MC simulations (potentially slow) \cite{chandra2010}, limited to power analysis \cite{chandra2010}, limited to the assumption of the constant steady-state temperature \cite{juan2011, juan2012}, exclusive focus on the maximal temperature \cite{juan2011}, \apriori\ chosen distributions of power and temperature \cite{juan2012, srivastava2010}. Consequently, there is a lack of flexible stochastic PTA techniques.

A straightforward approach, which is mentioned earlier in the context of \cite{chandra2010}, to analyze a stochastic system is the MC sampling coupled with a deterministic simulator. The major problem with the MC sampling is the low rate of convergence, \eg, the mean value converges as $\mcsamples^{-\ifrac{1}{2}}$ where $\mcsamples$ is the number of samples \cite{xiu2010, maitre2010}. A sample is a complete realization of the system, which makes MC-based methods slow and often infeasible since the number of needed simulations can be extremely high in order to obtain reliable estimates \cite{diaz-emparanza2002}.

Attractive alternatives to the MC sampling are spectral methods \cite{xiu2010, maitre2010, ghanem1991} featuring much faster convergence properties. One of such method is applied in this paper, namely, the generalized polynomial chaos (PC), which is the current state of the art in numerical analysis of stochastic systems \cite{xiu2010}. The popularity of PC is dictated by its applicability to a wide range of UQ problems and, more importantly, by its ability to construct easy-to-analyze representations of model responses to stochastic inputs \cite{eldred2009}. In other words, given a computationally intensive model subjected to uncertainty, PC produces a light surrogate, which possesses a number of pleasant features from the postprocessing perspective. To name few, the cummulative probability function (\cdf), probability density function (\pdf) can be estimated by sampling a trivial polynomial expression, which should be contrasted with the direct, expensive MC sampling of the original model; the expected value, variance, and other moments of higher orders can be determited analytically without any sampling. PC is commonly accompained by another spectral method called the Karhunen-Lo\`{e}ve (KL) expansion, which we also consider. KL acts at the preprocessing stage and parametrizes the model in terms of a finite---and typically much smaller yet sufficient---set of uncorrelated random variables (\rvs); the technique is especially useful if the presence of uncertainty is characterized by a random field, \ie, an infinite collection of \rvs. As mentioned before, such spectral methods are widespread. For instance, in \cite{shen2009}, the PC expansion based on the Hermite polynomials is employed to estimate the full-chip leakage; the KL expansion is used in \cite{bhardwaj2006} to calculate the leakage current of electrical circuits; an analysis of the voltage response of large power grids is carried out in \cite{ghanta2006}, where the PC and KL expansions are jointly utilized.

The contribution of the paper is in the following: we develop, for the first time, a framework for UQ of transient power and temperature variations of multiprocessor systems. The total power dissipation is modeled as an arbitrary transformation of temperature and a set of uncertain parameters; therefore, the framework is capable of capturing arbitrary joint effects of the mentioned components on the system. The uncertain parameters are assumed to have a given probability distribution, and there are no \apriori\ assumptions about the resulting distributions of power and temperature. Given a nominal dynamic power profile, our technique produces the corresponding stochastic power and temperature profiles defined in terms of polynomials of \rvs; the expressions are straightforward to be further analyzed. The framework is illustrated on one of the most important parameters affected by process variation: the subthreshold leakage current. However, it is worth being noted that our approach is not binded to any particular source of variability and, apart from the process-related variations, can be applied to, for instance, the variations induced by environment.

The reminder of the paper is organized as follows. In \sref{preliminaries}, we introduce the architecture model, which we shall consider, and formulate the objective of our study. The proposed framework is presented in \sref{proposed-framework}, where we pursue generality. A particular application of our technique is given in \sref{illustrative-example}, where the chosen illustrative example undergoes the theoretical development of the fremework step by step, and in \sref{experimental-results}, where a comparison with a MC-based approach is reported. \sref{conclusion} concludes the paper. The work also contains supplimentary materials with in-depth discussions on certain aspects of our approach; the materials are located in the appendix and are referrenced to through out the article by the capital letter ``S'', \eg, \aref{thermal-model}.
