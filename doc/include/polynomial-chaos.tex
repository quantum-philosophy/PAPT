At this point, the goal is to transform \eref{power-model} in such a way that the nonlinear correlations in \eref{recurrence} are eliminated. To this end, we employ a mathematical framework called the generalized polynomial chaos (PC) \cite{xiu2002}, which deconstructs stochastic processes into infinite sequences of hypergeometric orthogonal polynomials, defined by the Askey scheme, of \rvs. Orthogonality of a set of polynomials $\{ \pcb_i(\vz) \}$ is defined as
\begin{equation} \elabel{orthogonality}
  \oInner{\pcb_i(\vz)}{\pcb_j(\vz)} \eqdef \pcn_i \delta_{ij}
\end{equation}
where the operator $\oInner{\idot}{\idot}$ denotes the inner product in the Hilbert space spanned by the polynomials, $\delta_{ij}$ is the Kronecker delta function, and $\pcn_i = \oInner{\pcb_i(\vz)^2}$ are normalization constants. The inner product with a weight function $\PDF(\vz)$ is defined as the following multidimensional integral:
\begin{equation} \elabel{inner-product-integral}
  \oInner{f(\vz)}{g(\vz)} \eqdef \int f(\vz) g(\vz) \PDF(\vz) d\vz
\end{equation}
when $\PDF(\vz)$ is continuous, or as the following summation:
\begin{equation} \elabel{inner-product-sum}
  \oInner{f(\vz)}{g(\vz)} \eqdef \sum f(\vz) g(\vz) \PDF(\vz)
\end{equation}
when $\PDF(\vz)$ is discrete. In our case, the weight function corresponds to the probability density (or mass) function $\PDF_\vZ(\vz)$ of $\vZ(\o)$, and the inner product coincides with the expectation operator. It can be seen that orthogonality is equivalent to the absence of correlations, and this is what we seek for.

The first step towards a PC expansion is the choice of a suitable polynomial basis $\{ \pcb_i(\vZ(\o)) \}$ according to the Askey scheme \cite{xiu2002}. The step is crucial as the rate of convergence essentially depends on it. There are no strict rules that guarantee the optimal choice \cite{maitre2010, knio2006}; however, in general, one should be guided by the underlying \rvs. For instance, the Hermite polynomials are suitable for Gaussian \rvs\ while the Jacobi polynomials are preferable for beta variations; refer to \cite{xiu2010, xiu2002} for the other choices of polynomial bases and probability distributions as well as the case of arbitrary distributions.

Having an appropriate basis chosen, we first apply the PC procedure to power in \eref{ode-solution}. The resulting infinite series should be truncated to become feasible for practical computations. In this case, the expansion is defined as
\begin{equation} \elabel{pc-expansion}
  \oPC{\vars}{\pcorder}{\vP(0, \o)} \eqdef \sum_{i = 1}^{\pcterms} \pcc{\vP}_i \; \pcb_i(\vZ(\o))
\end{equation}
where $\pcc{\vP}_i \in \real^\cores$ are the coefficients of the expansion, and $\{ \pcb_i(\vZ(\o)) \}_{i = 1}^{\pcterms}$ is the truncated basis with $\pcterms$ polynomials. $\pcorder$ is the order of the expansion that determines the maximal degree of the $\vars$-variate polynomials involved in the expansion; hence, $\pcorder$ also defines the level of accuracy. When the total-order polynomial space is used \cite{beck2011}, $\pcterms$ is
\begin{equation} \elabel{pc-terms}
  \pcterms = {(\pcorder + \vars)!} \; / \; {(\pcorder! \; \vars!)}
\end{equation}

Finally, we need to determine the coefficients of the expansion, $\pcc{\vP}_i$. To this end, a spectral projection of the stochastic process in \eref{power-model} onto the space spanned by $\{ \pcb_i(\vZ(\o)) \}_{i = 1}^{\pcterms}$ is performed. Consequently, one needs to compute the inner product of \eref{power-model}, taken at the beginning of the corresponding time interval, with each polynomial from the basis as
\[
  \oInner{\sum_{i=1}^{\pcterms} \pcc{\vP}_i \pcb_i(\vZ(\o))}{\pcb_j(\vZ(\o))} = \oInner{\vP(0, \o)}{\pcb_j(\vZ(\o))}
\]
where $j \in \{ 1, \dotsc, \pcterms \}$. Making use of \eref{orthogonality}, we obtain
\begin{equation} \elabel{pc-coefficients}
  \pcc{\vP}_j = \frac{1}{\pcn_j} \oInner{\vP(0, \o)}{\pcb_j(\vZ(\o))}
\end{equation}
for $j \in \{ 1, \dotsc, \pcterms \}$. In general, the evaluation of the inner product in \eref{pc-coefficients}, which is defined by \eref{inner-product-integral} and \eref{inner-product-sum}, should be done numerically. \eref{pc-terms} together with \eref{pc-coefficients} reveal the major difficulty with polynomial expansions: the so-called curse of dimensionality, which results in a large number of terms in expansions and in the growing complexity of their coefficients when the number of stochastic dimensions is high. This issue is to be addressed in \sref{ie-uncertain-parameters} and \sref{ie-polynomial-chaos}.

It can be seen in \eref{recurrence} that, due to the linearity of the operations involved in the recurrence, $\vX_k(\t, \o)$ retains the same polynomial structure as $\vP_k(0, \o)$. Therefore, using \eref{pc-expansion}, \eref{recurrence} is rewritten as
\begin{align*}
  \oPC{\vars}{\pcorder}{\vX_k(\t, \o)} & = \mE(\t) \; \oPC{\vars}{\pcorder}{\vX_k(0, \o)} \\
  & {} + \mD(\t) \; \oPC{\vars}{\pcorder}{\vP_k(0, \o)}
\end{align*}
where $k \in \{ 1, \dotsc, \steps \}$ and $\t \in [0, \dt_k]$. Thus, there are two PC expansions for two concurrent stochastic processes with the same basis, but different coefficients. The above equation can be explicitly written as
\[
  \sum_{i = 1}^{\pcterms} \pcc{\vX}_{ki}(\t) \pcb_i(\vZ(\o)) = \sum_{i = 1}^{\pcterms} \left( \mE(\t) \pcc{\vX}_{ki} + \mD(\t) \pcc{\vP}_{ki} \right) \pcb_i(\vZ(\o))
\]
Multiplying the above equation by each polynomial from the basis and making use of \eref{orthogonality}, we have
\begin{equation} \elabel{pc-recurrence}
  \pcc{\vX}_{ki}(\t) = \mE(\t) \pcc{\vX}_{ki} + \mD(\t) \pcc{\vP}_{ki}
\end{equation}
where $k \in \{ 1, \dots, \steps \}$ and $i \in \{ 1, \dotsc, \pcterms \}$. Finally, \eref{pc-recurrence} is substituted into \eref{fourier-output} to compute the coefficients of the PC expansion of the temperature vector $\vTO(\t, \o)$. It should be noted that, since $\vP(0, \o)$ depends on temperature (see \sref{power-model}), at each step of the iterative process in \eref{pc-recurrence}, the computation of $\pcc{\vP}_{ki}$ should be undertaken with respect to the current PC expansion of $\vTO_k(0, \o)$.

It can be seen that the PC has replaced the recurrence in \eref{recurrence}---wherein an arbitrary functional $\vP_k(0, \omega)$ (see \sref{power-model}) of the uncertain parameters and stochastic temperature is involved, which, moreover, needs to be summed up in the presence of correlations with another random vector, $\vX_k(0, \omega)$---with a deterministic recurrence in \eref{pc-recurrence}, where only linear operations are used.
