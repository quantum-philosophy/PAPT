% \begin{table}
%   \centering
%   \tlabel{askey}
%   \begin{tabular}{|l|l|l|}
%     \hline
%     & Orthogonal Polynomials & Probability Distribution \\
%     \hline
%     \hline
%     \multirow{4}{*}{Continuous} & Hermite & Gaussian and Log-normal \\
%     & Laguerre & Gamma \\
%     & Jacobi & Beta \\
%     & Legendre & Uniform \\
%     \hline
%     \multirow{4}{*}{Discrete} & Charlier & Poisson \\
%     & Krawtchouk & Binomial \\
%     & Meixner & Negative binomial \\
%     & Hahn & Hypergeometric \\
%     \hline
%   \end{tabular}
%   \caption{The Askey scheme of orthogonal polynomials and the corresponding probability distributions \cite{xiu2002}.}
% \end{table}
At this point, the goal is to transfrom \eref{power-model} in such a way that the non-linear correlations in the recurrence given by \eref{recurrence} are eliminated. To this end, we empoly the generalize polynomial chaos (gPC) \cite{xiu2002}, which deconstructs stochastic processes into infinite sequences of orthogonal polynomials of r.v.'s. Withing the thermal model given by \eref{fourier-system}, the power dissipation and, consequently, temperature are such stochastic processes. The former is the source of uncertainties, therefore, we apply the gPC methodology directly to it.

According to the gPC methodology, the first step is to choose a suitable polynomial basis $\{ \pcb_i(\vIRV(\o)) \}$ from the Askey scheme of orthogonal polynomials \cite{xiu2002}. The step is crucial due to the following reason. The infinite PC/gPC expansion is guaranteed to converge in $\L{2}$ but the rate of convergence essentially depends on the chosen basis. Since, the infinite expansion should necessarily be truncated to become feasible for the practical computations, an inappropriate basis can lead either to a low accuracy or the need to keep a lot of terms in the expansion ruining the performance. There are no strict rules that can guarantee the optimal choice of the polynomial basis, however, there are best practices saying that the choice should be governed by the underlying r.v.'s., which drive the stochastic process. For instance, the Hermite polynomials are suitable for normally distributed r.v.'s while the Lagendre polynomials are preferable for the uniform variations\footnote{The distribution should not be necessarily continuous; the gPC works with discrete distributions as well \cite{xiu2002}.}. It is worth being mentioned that the major sources of uncertainties due to the process variation, such as the channel length, threshold voltage, gate oxide thickness, etc., are generally accepted to have a Gaussian distribution \cite{srivastava2010, liu2007, juan2012}, therefore, the Hermite basis is of a particular interest.

Once an appropriate basis has been chosen, the next step is to actually perform the gPC expansion of \eref{power-model} defined as
\begin{equation}
  \oPC{\vars}{\pcorder}{\vP(\t, \vTO(\t, \o), \o)} \eqdef \sum_{i = 1}^{\pcterms} \pcc{\vP}_i(\t) \pcb_i(\vIRV(\o))
\end{equation}
where $\pcc{\vP}_i(\t) \in \real^\cores$ are the coefficients of the expansion. $\pcorder \in \natural{0}$ is the order of the expansion, which determines the maximal (total) degree of the $\vars$-variate polynomials involved in the expansion and, consequently, the level of accuracy of the resulting model. $\pcterms$ is the number of terms in the expansion, which can be computed as $\pcterms = {\pcorder + \vars \choose \pcorder}$.

Finally, we need to determine the coefficients of the expansion $\pcc{\vP}_i(\t)$. To this end, the standard procedure of the projection of the original stochastic process (\eref{power-model}) onto the space spanned on the orthogonal polynomials $\{ \pcb_i(\vIRV(\o)) \}_{i = 1}^{\pcterms}$ is performed as follows.

Since the basis is orthogonal, the following equality involving the $\vars$-dimensional integration holds:
\begin{equation} \elabel{orthogonality}
  \oExp{\pcb_i(\vIRV) \pcb_j(\vIRV)} = \oExp{\pcb_i(\vIRV)^2} \delta_{ij}, \sep i,j \in \natural{0}
\end{equation}
where $\delta_{ij}$ is the Kronecker delta function, and the expectation is taken with respect to the probability measure of the chosen basis.

\begin{equation} \elabel{pc-coefficients}
  \pcc{f}_k = \frac{\innerp{f(x), \pcb_k}}{\innerp{\pcb_k^2}}, \forall k = 1, \dotsc, \pcorder
\end{equation}

\begin{equation} \elabel{pc-integral}
  \innerp{f(\vIRV), \pcb_k} = \int_{} f(\virv) \pdf_\vIRV(\virv) d\virv
\end{equation}
