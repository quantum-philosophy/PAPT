Given a thermal specification $\system$ of the platform (see \sref{architecture-model}), an equivalent thermal RC circuit with $\nodes$ thermal nodes is constructed \cite{kreith2000}. The structure of the circuit depends on the intended level of granularity, which defines the resulting accuracy. For clarity, we assume that each processing element is mapped onto one corresponding node, and the thermal package is represented as a set of additional nodes (see also \sref{ie-thermal-model}). The thermal behavior of the circuit is modeled with the following system of differential-algebraic equations:
\begin{subnumcases}{\elabel{fourier-system-original}}
  \mC \frac{d\vTI(\t, \o)}{d\t} + \mG \vTI(\t, \o) = \mM \vP(\t, \o) \elabel{fourier-original} \\
  \vTO(\t, \o) = \mM^T \vTI(\t, \o) + \vTO_\amb
\end{subnumcases}
where $\mC, \mG \in \real^{\nodes \times \nodes}$ are a diagonal matrix of the thermal capacitance and a symmetric, positive-definite matrix of the thermal conductance, respectively; $\vTI \in \real^\nodes$ is the vector of the difference between the temperature of the thermal nodes and the ambient temperature; $\vP \in \real^\cores$ and $\mM \in \real^{\nodes \times \cores}$ are the input vector of power (see \eref{power-model}) and its mapping matrix to the thermal nodes; $\vTO \in \real^\cores$ is the temperature vector of the processing elements, and $\vTO_\amb \in \real^\cores$ is the vector of the ambient temperature. Without loss of generality, the last is modeled as a deterministic variable. For convenience, we perform an auxiliary transformation of \eref{fourier-system-original} \cite{ukhov2012}. Let $\vX = \mC^\ifrac{1}{2} \vTI$, $\mA = -\mC^{-\ifrac{1}{2}} \mG \mC^{-\ifrac{1}{2}}$, and $\mB = \mC^{-\ifrac{1}{2}} \mM$, then
\begin{subnumcases}{\elabel{fourier-system}}
  \frac{d\vX(\t, \o)}{d\t} = \mA \vX(\t, \o) + \mB \vP(\t, \o) \elabel{fourier-de} \\
  \vTO(\t, \o) = \mB^T \vX(\t, \o) + \vTO_\amb \elabel{fourier-output}
\end{subnumcases}
where the coefficient matrix $\mA$ preserves the symmetry and positive-definiteness of $\mG$. In general, \eref{fourier-de} is a system of nonlinear differential equations due to $\vP(\t, \o)$, which has an arbitrary dependency on temperature (see \eref{power-model}). Hence, there are no closed-form solutions to \eref{fourier-de}.

The time intervals $\{ \dt_i \}_{i = 1}^{\steps}$ of the input profile $\profPdyn$ (see \sref{problem-formulation}) are assumed to be short enough such that the total power can be approximated by a constant within one interval. In this case, \eref{fourier-de} is a system of linear differential equations that can be solved analytically. The solution is
\begin{equation} \elabel{ode-solution}
  \vX(\t, \o) = \mE(\t) \vX(0, \o) + \mD(\t) \vP(0, \o)
\end{equation}
where $\t \in [0, \dt]$, $\vP(0, \o)$ is the power dissipation at the beginning of the time interval with respect to the corresponding temperature, and
\begin{align*}
  & \mE(\t) = e^{\mA \t} \in \real^{\nodes \times \nodes} \\
  & \mD(\t) = \mA^{-1} (e^{\mA \t} - \mI) \mB \in \real^{\nodes \times \cores}
\end{align*}
Taking into consideration all $\steps$ time intervals of $\profPdyn$, we obtain the following recurrent expression:
\begin{align}
  & \vX_k(\t, \o) = \mE(\t) \vX_k(0, \o) + \mD(\t) \vP_k(0, \o) \elabel{recurrence} \\
  & \vX_0(0, \o) = \vZero, \hspace{1em} \vX_k(0, \o) = \vX_{k - 1}(\dt_{k - 1}, \o) \nonumber
\end{align}
where $k \in \{ 1, \dotsc, \steps \}$. Note that, with a slight abuse of notation, we let $\t$ start from zero at each interval, i.e., $\t \in [ 0, \dt_k ]$. Also, without loss of generality, we let the initial temperature be equal to the ambient temperature. In the deterministic case, the recurrence in \eref{recurrence} can be readily employed to perform the deterministic transient PTA \cite{thiele2011, ukhov2012}. In the stochastic case, however, the analysis of \eref{recurrence} is substantially different since $\vX_k(\t, \o)$, $\vTO_k(\t, \o)$, and $\vP_k(\t, \o)$ are probabilistic quantities. The situation is complicated by fact that, at each step of the iterative process, (a) $\vP_k(\t, \o)$ is an arbitrary (typically nonlinear) transformation of the uncertain parameters $\vU(\o)$ and stochastic temperature $\vTO_k(\t, \o)$ (see \sref{power-model}), which results in a multivariate \rv\ with an unknown probability distribution; (b) $\vU(\o)$, $\vX_k(\t, \o)$, $\vTO_k(\t, \o)$, and $\vP_k(\t, \o)$ are dependent \rvs\ as the latter three are functions of $\vU(\o)$. Therefore, the operations involved in \eref{recurrence} are to be performed on dependent \rvs\ with arbitrary probability distributions, which, in general, have no closed-form solutions. To tackle this difficulty, we utilize the theory of polynomial chaos expansions mentioned in \sref{introduction} and further discussed in the next section.
