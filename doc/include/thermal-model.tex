Given the platform description $\system$ (see \sref{architecture-model}), an equivalent thermal RC circuit with $\nodes$ \definition{thermal nodes} is constructed \cite{kreith2000}. The structure of the circuit depends on the intended level of details defining the accuracy of the final model. The thermal behaviour of the circuit is modeled with the following system of differential-algebraic equations (DAEs) with the state-space dimension equal to $\nodes$:
\begin{numcases}{}
  \mC \frac{d\vTI(\t, \o)}{d\t} + \mG \vTI(\t, \o) = \mM \vP(\t, \vTO(\t, \o), \o) \nonumber \\
  \vTO(\t, \o) = \mM^T \vTI(\t, \o) + \vTO_\amb \nonumber
\end{numcases}
$\mC, \mG \in \real^{\nodes \times \nodes}$ are the matrices of the thermal capacitance and conductance, respectively. $\mC$ is a diagonal matrix with positive elements, and $\mG$ is a symmetric, positive-definite matrix. $\vTI \in \real^\nodes$ is the state vector of the system, which corresponds to the difference between the temperature of the thermal nodes and the ambient temperature. $\vP \in \real^\cores$ is the input vector of the power dissipation of the processing elements (see \eref{power-model}), and $\mM \in \real^{\nodes \times \cores}$ is its mapping matrix to the thermal nodes. $\vTO \in \real^\cores$ is the output vector of the system, which is the temperature of the processing elements. Finally, $\vTO_\amb \in \real^\cores$ is the vector of the ambient temperature, which, for clarity of presentation and without loss of generality, we model as a deterministic variable.

For convenience, we perform an auxiliary transformation \cite{ukhov2012} of the original thermal model given above. Let $\vX = \mC^\ifrac{1}{2} \vTI$, $\mA = -\mC^\ifrac{1}{2} \mG \mC^\ifrac{1}{2}$, and $\mB = \mC^\ifrac{1}{2} \mM$, then\footnote{Since $\mC$ is diagonal, the transformation is cheat to undertake.}
\begin{subnumcases}{}
  \frac{d\vX(\t, \o)}{d\t} = \mA \vX(\t, \o) + \mB \vP(\t, \vTO(\t, \o), \o) \elabel{fourier-de} \\
  \vTO(\t, \o) = \mB^T \vX(\t, \o) + \vTO_\amb \elabel{fourier-output}
\end{subnumcases}
where $\vX$ is the new state vector. It should be noted that the coefficient matrix $\mA$ preserves the  symmetry and positive-definiteness of $\mG$. In general, \eref{fourier-de} is a \emph{non-linear} system of differential equations due to the power term, which has a non-linear dependency on temperature (see \eref{power-model}). Hence, there is no a closed-form solution to \eref{fourier-de}.

Since the input power profile $\prof{\mP_\dyn}$ is discrete in time by definition (\sref{system-profiles}), it is natural to assume that the dynamic power stays constant between neighboor samples. (Otherwise, if power undergoes dramatical changes, one needs to capture a finer profile.) Also, since the time intervals $\{ \dt_i \}_{i = 1}^{\steps}$ are short, the leakage power can be safely approximated by a constant within one interval. Therefore, if we focus only on $\t \in [0, \dt]$, \eref{fourier-de} can be rewritten as
\begin{equation} \elabel{fourier-ode}
  \frac{d\vX(\t, \o)}{d\t} = \mA \vX(\t, \o) + \mB \vP(0, \vTO(0, \o), \o)
\end{equation}
where $\vP(0, \vTO(0, \o), \o)$ is the power dissipation evaluated at the beginning of the time interval. \eref{fourier-ode} is a system of ordinary differential equations (ODEs), which can be solved analyticaly. The solution is
\begin{equation} \elabel{ode-solution}
  \vX(\t, \o) = \mE(\t) \vX(0, \o) + \mD(\t) \vP(0, \vTO(0, \o), \o)
\end{equation}
where $\t \in [0, \dt]$ and
\begin{align}
  & \mE(\t) = e^{\mA \t} \in \real^{\nodes \times \nodes} \elabel{e} \\
  & \mD(\t) = \mA^{-1} (e^{\mA \t} - \mI) \mB \in \real^{\nodes \times \cores} \elabel{d}
\end{align}
which are matrix-valued functions of $\t$.

Taking into consideration all $\steps$ time intervals of $\prof{\mP_\dyn}$, we obtain a recurrent expression for $i \in \{ 1, \dotsc, \steps \}$
\begin{equation} \elabel{recurrence}
  \vX_i(\t, \o) = \mE(\t) \vX_i(0, \o) + \mD(\t) \vP_i(0, \vTO_i(0, \o), \o)
\end{equation}
where $\t \in [ 0, \dt_i ]$ and
\begin{align*}
  & \vX_0(0, \o) = 0 & \vX_i(0, \o) = \vX_{i - 1}(\dt_{i - 1}, \o) \\
  & \vTO_0(0, \o) = \vTO_\amb & \vTO_i(0, \o) = \vTO_{i - 1}(\dt_{i - 1}, \o)
\end{align*}
Here, without loss of generality, we let the initial temperature to be equal to the ambient one.

In the deterministic case, the analytical solution, which leads to the recurrence given in \eref{recurrence}, can readily be employed to perform the (deterministic) transient temperature analysis, and it was shown to be more efficient than numerical integrators, cf. \cite{ukhov2012}. In the stochastic case, however, where $\vX_i$, $\vTO_i$, and $\vP_i$ are probabilistic quantities, a stright-forward analysis of \eref{recurrence} is difficult to undertake due to the existance of unavaidable non-linear correlations between successive steps.
