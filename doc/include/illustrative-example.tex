In this section, we demonstate the proposed framework to perfrom the stochastic temperature analysis of multiprocessor platforms under uncertain dissipation of power. To be exact, we address the process variation as it is one of the major concerns in the multiprocessor system design.

Since the influence of the process variation on the dynamic power is negligible \cite{juan2012, srivastava2010}, we let $\vP_\dyn(\t, \o) \equiv \vP_\dyn(\t)$. Therefore, in \eref{power-model}
\[
  f_\dyn(\vP_\dyn(\t), \o) = \vP_\dyn(\t)
\]
Concerning the leakage part, we focus on the variations in the subthreshold leakage current due to the variations in the effective channel length $\Leff{}$, as in \cite{juan2012}. The model is based on BSIM4 \cite{bsim4} and can be written in the following simplified form:
\[
f_{\leak, i}(\Theta_i(\t, \o), \o) = \alpha_i \; \frac{\Theta_i(\t, \o)^2}{\Leff{, i}(\o)} \text{exp}\left(-\frac{\beta \; \Leff{, i}(\o)}{\Theta_i(\t, \o)}\right)
\]
where the fitting coefficients $\alpha_i$ and $\beta_i$, $i \in \{ 1, \dotsc, \cores \}$, are specific to each of the processing elements in a heterogenuous multiprocessor system. The r.v.'s $\Leff{, i}(\o)$ are modeled as a sum of global (inter-die) and local (intra-die) variations, cf. \cite{juan2012, srivastava2010, shen2009}:
\[
  \Leff{, i}(\o) = \nLeff + \gLeff(\o) + \lLeff{, i}(\o)
\]
where $\nLeff$ is the nominal channel length, $\gLeff(\o)$ is a zero-mean r.v. that represents global variations of $\nLeff$, and $\lLeff{, i}(\o)$ is a zero-mean r.v. for local variations. The local variations are known to be highly correlated. Without loss of generality, we use the exponential model of correlations, as in \cite{shen2009}, given by
\[
  \rho(r) = e^{-r^2/\eta^2}
\]

It is generally accepted that the uncertainties due to the process variation (including the effective channel length $\Leff{}$, threshold voltage, gate oxide thickness, etc.) have Gaussian distributions \cite{srivastava2010, liu2007, juan2012}. Therefore, we assume that the Hermite polynomial basis is of a particular interest in this case.

\subsection{Preprocessing of Random Variables} \slabel{pca}
The covariance matrix is necessarily a real, symmetric matrix. Therefore, it admits the eigenvalue factorization \cite{press2007} in the form $\mCov_\mrv{X} = \m{V}_{\mCov_\mrv{X}} \m{\Lambda}_{\mCov_\mrv{X}} \m{V}_{\mCov_\mrv{X}}^T$ where $\m{V}_{\mCov_\mrv{X}}$ and $\m{\Lambda}_{\mCov_\mrv{X}}$ are an orthogonal matrix of the eigenvectors and a diagonal matrix of the eigenvalues of $\mCov_\mrv{X}$, respectively. Denote $\oEigen{\mCov_\mrv{X}} = \m{V}_{\mCov_\mrv{X}} {\m{\Lambda}_{\mCov_\mrv{X}}}^\ifrac{1}{2}$. In this notation, $\mCov_\mrv{X} = \oEigen{\mCov_\mrv{X}} \oEigen{\mCov_\mrv{X}}^T$. Consequently, $\mrv{X}$ can be normalized as $\mrv{X} = \oEigen{\mCov_\mrv{X}} \mrv{y} + \vExp_\mrv{X}$ where $\oExp{\mrv{y}} = \mZero$ and $\oCov{\mrv{y}} = \mI$.

\subsection{Computation of the gPC Coefficients} \slabel{integration}
The family of Gaussian quadrature rules is superior when it comes to the one-dimensional integration of smooth functions, viz. those that are well-approximated by a polynomial \cite{press2007}. In this case, the integral of interest is replaced by a weighted sum over $n$ points, which necessary gives the \emph{exact} solution when the integrand is a $(2n - 1)$-degree (or less) polynomial. The rules differ by the choice of abscissae and weights for the summation. For instance, the abscissae of the Gauss-Hermite quadrature rule are the roots of the Hermite polynomials, and the weight function is $e^{-x^2}$, which makes the rule suitable for the integration with respect to the Gaussian measure.

In the multi-dimensional case, which we have in \eref{pc-coefficients}, one usually employs so-called cubature rules that are constructed from one-dimensional quadrature rules. The rules are characterized by the level of accuracy $\cblevel \in \natural{1}$. A $\cblevel$-level cubature rule for the $\vars$-dimensional integration is defined as
\[
  \oCub{\vars}{\cblevel}{f} \eqdef \sum_{i = 1}^{\cbpoints} f(\cbp{\vZ}_i) w_i
\]
where $\cbpoints$ is the number of summation points, $\cbp{\vZ}_i \in \real^\vars$ and $w_i \in \real$ are the prescribed abscissae and weights, respectively. The abscissae are $\vars$-dimensional vectors, which corresponds the number of uncertain parameters $\vZ(\o)$.

In order to tackle dependent r.v.'s for a non-Gaussian distribution, the approach proposed in \cite{babuska2010} can be applied.
