Here we explain a particular application of the developed framework. We assume that the dynamic power is deterministic while the leakage power is drastically influenced by the process variation as it is motivated in \sref{introduction}.

\subsection{Preprocessing of Random Variables} \slabel{pca}
The covariance matrix is necessarily a real, symmetric matrix. Therefore, it admits the eigenvalue factorization \cite{press2007} in the form $\mCov_\mrv{X} = \m{V}_{\mCov_\mrv{X}} \m{\Lambda}_{\mCov_\mrv{X}} \m{V}_{\mCov_\mrv{X}}^T$ where $\m{V}_{\mCov_\mrv{X}}$ and $\m{\Lambda}_{\mCov_\mrv{X}}$ are an orthogonal matrix of the eigenvectors and a diagonal matrix of the eigenvalues of $\mCov_\mrv{X}$, respectively. Denote $\oEigen{\mCov_\mrv{X}} = \m{V}_{\mCov_\mrv{X}} {\m{\Lambda}_{\mCov_\mrv{X}}}^\ifrac{1}{2}$. In this notation, $\mCov_\mrv{X} = \oEigen{\mCov_\mrv{X}} \oEigen{\mCov_\mrv{X}}^T$. Consequently, $\mrv{X}$ can be normalized as $\mrv{X} = \oEigen{\mCov_\mrv{X}} \mrv{y} + \vExp_\mrv{X}$ where $\oExp{\mrv{y}} = \mZero$ and $\oCov{\mrv{y}} = \mI$.

\subsection{Computation of the gPC Coefficients} \slabel{integration}
The family of Gaussian quadrature rules is superior when it comes to the one-dimensional integration of smooth functions, viz. those that are well-approximated by a polynomial \cite{press2007}. In this case, the integral of interest is replaced by a weighted sum over $n$ points, which necessary gives the \emph{exact} solution when the integrand is a $(2n - 1)$-degree (or less) polynomial. The rules differ by the choice of abscissae and weights for the summation. For instance, the abscissae of the Gauss-Hermite quadrature rule are the roots of the Hermite polynomials, and the weight function is $e^{-x^2}$, which makes the rule suitable for the integration with respect to the Gaussian measure.

In the multi-dimensional case, which we have in \eref{pc-coefficients}, one usually employs so-called cubature rules that are constructed from one-dimensional quadrature rules. The rules are characterized by the level of accuracy $\cblevel \in \natural{1}$. A $\cblevel$-level cubature rule for the $\vars$-dimensional integration is defined as
\[
  \oCub{\vars}{\cblevel}{f} \eqdef \sum_{i = 1}^{\cbpoints} f(\cbp{\vZ}_i) w_i
\]
where $\cbpoints$ is the number of summation points, $\cbp{\vZ}_i \in \real^\vars$ and $w_i \in \real$ are the prescribed abscissae and weights, respectively. The abscissae are $\vars$-dimensional vectors, which corresponds the number of uncertain parameters $\vZ(\o)$.

In order to tackle dependent r.v.'s for a non-Gaussian distribution, the approach proposed in \cite{babuska2010} can be applied.
