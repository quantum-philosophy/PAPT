At Stage~2 in \fref{algorithm}, we need to decide on the power model with the nominal dynamic power, identified uncertain parameters, and temperature as inputs. To this end, \eref{power-model} is decomposed into the sum of dynamic and leakage components denoted by $\f_\dyn(\vP_\dyn(\t), \vU(\o))$ and $\f_\leak(\vTO(\t, \o), \vU(\o))$, respectively. As motivated in the previous subsection, we let $\f_\dyn$ be $\vP_\dyn(\t)$. The development of the leakage part, $\f_\leak$, is based on a set of SPICE simulations of reference electrical circuits, wherein the Predictive Technology Model \cite{ptm} is employed. The data from the simulations are then fed to the curve fitting toolbox provided by MATLAB \cite{matlab}, and the following exponential model of the leakage current is acquired:
\begin{equation} \elabel{leakage-current}
  I_\leak(\T, \Leff) = \exp \left( \alpha_0 + \alpha_1 \Leff + \alpha_2 \T + \alpha_3 \Leff \T + \alpha_4 \T^2 \right)
\end{equation}
where $\alpha_i$ are fitting coefficients. The reference leakage current $I_\leak(\T, \Leff)$ is further scaled up to the power level of each of the processing elements, \ie, for $i = 1, \dotsc, \cores$, we have
\[
  \f_{\leak, i}(\T_i(\t, \o), \o) = \beta_i I_{\leak, i}(\T_i(\t, \o), \Leff_i(\o))
\]
where $\beta_i$ are scaling coefficients, and $\Leff_i(\o)$ is the sum of $\Leff$, $\gLeff(\o)$, and $\lLeff(r_i, \o)$ as discussed in \sref{ie-uncertain-parameters}.

At Stage~3, the equivalent thermal RC circuits---specifically, the coefficient matrices in \eref{fourier-system}---are constructed by means of HotSpot v5.02 \cite{hotspot}. Thermal packages are modeled with three layers, and the relation between the number of processing elements and the number of thermal nodes is given by $\nodes = 4 \cores + 12$. An example of such a circuit is depicted in \fref{circuit}, and additional details are given in \aref{thermal-model}.
