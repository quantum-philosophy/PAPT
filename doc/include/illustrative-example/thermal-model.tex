We move on to \stage{3}\ where the thermal model of the multiprocessor system is to be established.
Given the thermal specification $\system$ of the considered platform (the floorplan of the die, the configuration of the thermal package, \etc), we employ HotSpot (v5.02) \cite{hotspot} in order to construct an equivalent thermal RC circuits of the system.
Specifically, we are interested in the coefficient matrices $\mCF(\t)$ and $\mCS(\t)$ in \eref{recurrence} (see also \fref{algorithm}), which HotSpot helps us to compute by providing the corresponding capacitance and conductance matrices of the system as described in \aref{thermal-model}.
In this case, thermal packages are modeled with three layers, and the relation between the number of processing elements and the number of thermal nodes is given by $\nnodes = 4 \nprocs + 12$ (where $\nprocs$ is the number of active nodes, \ie, processing elements).
An example of such a circuit for a dual-core platform is depicted in \fref{circuit}.

To conclude, the power and thermal models of the platform are now acquired, and we are ready to construct the corresponding surrogate model \via\ PC expansions, which is the topic for the discussion in the following subsection.
