The total dissipation of power is composed of two major parts: dynamic and leakage. The influence of process variation on the dynamic power is known to be negligibly small \cite{juan2011, juan2012, srivastava2010}; on the other hand, the variability of leakage is substantial, wherein the subthreshold current contributes the most \cite{juan2011, juan2012}. Hence, we focus on the subthreshold leakage and, more specifically, on the effective channel length denoted by $\Leff$. The variations of $\Leff$ are split into global (inter-die) and local (intra-die) parts \cite{chandra2010, juan2011, juan2012, srivastava2010, shen2009}. The global contribution is shared among all the processing elements and, therefore, is modeled as a single \rv\ $\gLeff(\o)$; furthermore, $\gLeff(\o)$ is typically assumed to be independent with respect to the local variations. The latter, however, are known to have high spatial correlations, which, due to particularities of the manufacturing process, often bear radial structures \cite{friedberg2005, cheng2011}. Therefore, as in \cite{ghanta2006}, we model the intra-die variations as a continuous-space stochastic process $\lLeff(r, \o)$ with the following radial covariance function \cite{ghanem1991}:
\begin{equation} \elabel{covariance-function}
  \fCov{\lLeff}{r_i, r_j} = e^{-|r_i - r_j|/\corrLength}
\end{equation}
where $r_i$ is the distance from the center of the $i$th processing element to the center of the die, and $\corrLength$ is the correlation length. In words, the structure implies that the smaller the radial distance between two processing elements on the die, the more likely they are to have similar deviations of the effective channel length. Finally, it is generally accepted that the variations of the channel length have Gaussian distributions \cite{juan2011, juan2012, srivastava2010}. Therefore, we assume that the \rv\ $\gLeff(\o)$ as well as the process $\lLeff(r, \o)$ are Gaussian.

$\gLeff(\o)$ and $\lLeff(r, \o)$ form the input set of uncertain parameters $\vU(r, \o)$, which is to be processed in order to extract a finite set of mutually independent \rvs\ $\vZ(\o)$. Following the guidelines in \sref{uncertain-parameters}, Stage~1, the KL expansion suits the best in such a situation. A thorough elaboration on KL is out of the scope of this paper; however, the interested reader is referred to \aref{uncertain-parameters} where KL is applied to \eref{covariance-function}.
