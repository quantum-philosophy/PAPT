The power dissipation is composed of two major parts: dynamic and leakage. The influence of process variation on the dynamic power is known to be negligibly small \cite{juan2011, juan2012, srivastava2010}; on the other hand, the variability of leakage is substantial, in which the subthreshold current contributes the most \cite{juan2011, juan2012}. Hence, we focus on the subthreshold leakage and, more specifically, on the effective channel length $\Leff$, which has no loss of generality. The variations of $\Leff$ are split into global (inter-die) and local (intra-die) parts \cite{chandra2010, juan2011, juan2012, srivastava2010, shen2009}. The global offset is shared among all the processing elements, \ie, it is a single \rv, whereas there are $\cores$ distinct \rvs\ for the local deviations. Therefore, the channel length within the $i$th processing element is given as
\[
  \Leff_i(\o) = \nLeff + \gLeff(\o) + \lLeff_i(\o)
\]
where $\nLeff$ is the nominal channel length, $\gLeff(\o)$ is a zero-mean \rv\ that represents the global variation, and $\lLeff_i(\o)$ is a zero-mean \rv\ for the local component. Consequently, the uncertain parameters of the current model are
\[
  \vU(\o) = \vec{\lLeff_1(\o), \dotsc, \lLeff_\cores(\o), \gLeff(\o) }.
\]
The global \rv\ is typically assumed to be uncorrelated with respect to the local ones. The latter, however, are known to have high spatial correlations, which, due to particularities of the manufacturing process, often bear radial structures \cite{friedberg2005, cheng2011}. Therefore, without loss of generality, we employ the following radial model of correlations for $\lLeff_i(\o)$ \cite{ghanem1991, ghanta2006}:
\begin{equation} \elabel{covariance-function}
  \oCorr{\lLeff_i(\o), \lLeff_j(\o)} = e^{-|r_i - r_j|/\corrLength}
\end{equation}
where $r_i$ is the distance between the $i$th processing element and the center of the die, and $\corrLength$ is the correlation length. For convenience, the resulting correlation matrix is extended by one dimension to pack $\gLeff(\o)$ and $\lLeff_i(\o)$ together. In this case, the matrix obtains one additional non-zero element on the diagonal. Taking into account the variances of the \rvs, the covariance matrix denoted by $\mCov_\vU$ is formed.
