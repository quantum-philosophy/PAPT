At \stage{4}, the uncertain parameters, power model, and thermal model developed in the previous sections are to be fused together under the desired workload, $\profilePdyn$, to produce the corresponding stochastic power $\profileP{\o}$ and temperature $\profileT{\o}$ profiles.
The construction of PC expansions, in the current scenario, is based on the Jacobi polynomial basis (see \tref{askey} in the appendix) as it is preferable in situations involving beta-distributed parameters \cite{xiu2010}.
To give an example, for a dual-core platform (\ie, $\nprocs = 2$) with two stochastic dimensions (\ie, $\nvars = 2$), the second-order PC expansion (\ie, $\pcorder = 2$) of temperature at the $k$th moment of time has the following form:
\begin{align}
  & \oPC{2}{2}{\vTO_k(\o)} = \pcc{\vTO}_{k1} \pcb_1(\vZ(\o)) + \pcc{\vTO}_{k2} \pcb_2(\vZ(\o)) + \pcc{\vTO}_{k3} \pcb_3(\vZ(\o)) \nonumber \\
  & {} \; \; + \pcc{\vTO}_{k4} \pcb_4(\vZ(\o)) + \pcc{\vTO}_{k5} \pcb_5(\vZ(\o)) + \pcc{\vTO}_{k6} \pcb_6(\vZ(\o)) \elabel{pc-example}
\end{align}
where the coefficients $\pcc{\vTO}_{ki}$ are vectors with two elements corresponding to the two processing elements,
\begin{align*}
  & \pcb_1(\vz) = 1, \hspace{1em} \pcb_2(\vz) = 2 \z_1, \hspace{1em} \pcb_3(\vz) = 2 \z_2, \hspace{1em} \pcb_4(\vz) = 4 \z_1 \z_2 \\
  & \pcb_5(\vz) = \frac{15}{4} \z_1^2 - \frac{3}{4}, \hspace{1em} \text{and} \hspace{1em} \pcb_6(\vz) = \frac{15}{4} \z_2^2 - \frac{3}{4}.
\end{align*}
The expansion for power has the same structure but different coefficients.
Such a series might be shorter or longer depending on the accuracy requirements defined by $\pcorder$.

Once the basis has been chosen, we need to compute the corresponding coefficients, specifically, $\pcc{\vP}_{ki}$ in \eref{pc-expansion}, which will yield $\pcc{\vTO}_{ki}$.
As shown in \aref{polynomial-chaos}, the computation of $\pcc{\vP}_i$ involves multidimensional integration with respect to the \pdf\ of $\vZ(\o)$.
In numerical analysis, this task is typically accomplished by virtue of a quadrature rule \cite{press2007}, which, loosely speaking, is a weighted summation over the integrand values computed at a set of prescribed points.
These points along with the corresponding weights are generally precomputed and tabulated; for this purpose, we use the library of MATLAB codes available at \cite{burkardt2013}.
A natural choice of a quadrature rule when beta distributions are concerned is the Gauss-Jacobi quadrature.
Additional details are given in \aref{gauss-quadrature}.

To summarize, we have completed four out of five stages of the proposed framework depicted in \fref{algorithm}.
The result is a light surrogate for the model in \eref{fourier-system}.
At each moment of time, the surrogate is composed of two $\nprocs$-valued polynomials, one for power and one for temperature, that are defined in terms of $\nvars$ mutually independent random variables; an example of such a polynomial is given in \eref{pc-example}.
