The construction of PC expansions is based on the Hermite basis (see \tref{askey}) as it was found to be optimal in situations involving Gaussian parameters \cite{xiu2010}. A one-dimensional example of the basis is given in \fref{hermite} where the first six Hermite polynomials $\{ \pcb_i(\z) \}_{i = 1}^6$ are displayed.

Once the basis has been chosen, we need to compute the corresponding coefficients, specifically, $\pcc{\vP}_i$ in \eref{pc-expansion}. As shown in \aref{polynomial-chaos}, the computation of $\pcc{\vP}_i$ involves multidimensional integration with respect to the \pdf\ of the \rvs\ $\vZ(\o)$. In numerical analysis, this task is typically accomplished by virtue of a quadrature rule \cite{press2007}, which, loosely speaking, is a weighted summation over the integrand values computed at prescribed points. A natural choice of a quadrature rule when the weight function is a Gaussian \pdf\ is the Gauss-Hermite quadrature. Further details are given in \aref{gauss-quadrature}.

To summarize, we have completed four out of five stages of the proposed UQ framework depicted in \fref{algorithm}. The result is a light surrogate of the model in \eref{fourier-system}. At each moment of time, the surrogate is composed of two $\nprocs$-valued polynomials, one for power and one for temperature, that are defined in terms of $\nvars$ mutually independent \rvs; an example of such a polynomial is given in \eref{pc-k}. The constructed representation can be trivially analyzed to retrieve various statistics of the system in \eref{fourier-system}, and this is \stage{5}\ in \fref{algorithm}, which will be illustrated as a part of the next section.
