At \stage{4}, the uncertain parameters, power model, and thermal model developed in the previous sections are to be fused together under the desired workload $\profilePdyn$ to produce the corresponding stochastic power $\profileP$ and temperature $\profileT$ profiles.
The construction of PC expansions, in the current scenario, is based on the Jacobi polynomial basis as it is preferable in situations involving beta-distributed parameters \cite{xiu2010}.
To give an example, for a dual-core platform (\ie, $\nprocs = 2$) with two stochastic dimensions (\ie, $\nvars = 2$), the second-order PC expansion (\ie, $\pcorder = 2$) of temperature at the $k$th time moment is as follows:\footnote{The Jacobi polynomials have two parameters \cite{xiu2010}, and the shown $\{ \pcb_i \}_{i = 1}^6$ correspond to the case where both parameters are equal to two.}
\begin{align}
  \oPC{2}{2}{\vTO_k} &= \pcc{\vTO}_{k1} \, \pcb_1(\vZ) + \pcc{\vTO}_{k2} \, \pcb_2(\vZ) + \pcc{\vTO}_{k3} \, \pcb_3(\vZ) \nonumber \\
  & {} + \pcc{\vTO}_{k4} \, \pcb_4(\vZ) + \pcc{\vTO}_{k5} \, \pcb_5(\vZ) + \pcc{\vTO}_{k6} \, \pcb_6(\vZ) \elabel{pc-example}
\end{align}
where the coefficients $\pcc{\vTO}_{ki}$ are vectors with two elements corresponding to the two processing elements,
\begin{align*}
  & \pcb_1(\vX) = 1, \hspace{0.7em} \pcb_2(\vX) = 2 \x_1, \hspace{0.7em} \pcb_3(\vX) = 2 \x_2, \hspace{0.7em} \pcb_4(\vX) = 4 \x_1 \x_2 \\
  & \pcb_5(\vX) = \frac{15}{4} \x_1^2 - \frac{3}{4}, \hspace{0.7em} \text{and} \hspace{0.7em} \pcb_6(\vX) = \frac{15}{4} \x_2^2 - \frac{3}{4}.
\end{align*}
The expansion for power has the same structure but different coefficients.
Such a series might be shorter or longer depending on the accuracy requirements defined by $\pcorder$.

Once the basis has been chosen, we need to compute the corresponding coefficients, specifically, $\{ \pcc{\vP}_{ki} \}_{i = 1}^\pcterms$ in \eref{pc-expansion}, which yield $\{ \pcc{\vTO}_{ki} \}_{i = 1}^\pcterms$.
As shown in \aref{polynomial-chaos}, these computations involve multidimensional integration with respect to the \pdf\ of $\vZ$, which should be done numerically using a quadrature rule; recall \sref{pc-coefficients}.
When beta distributions are concerned, a natural choice of such a rule is the Gauss-Jacobi quadrature.
Additional details are given in \aref{gauss-quadrature}.

To summarize, we have completed four out of five stages of the proposed framework depicted in \fref{algorithm}.
The result is a light surrogate for the model in \eref{fourier-system}.
At each moment of time, the surrogate is composed of two $\nprocs$-valued polynomials, one for power and one for temperature, which are defined in terms of $\nvars$ mutually independent random variables; an example of such a polynomial is given in \eref{pc-example}.
