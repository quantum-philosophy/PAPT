\begin{reviewer}
This work developed a useful probabilistic framework to estimate the transient power and temperature variations of electronic system designs. The description is clear, and the organization is well.
\end{reviewer}
\begin{authors}
Thank you.
\end{authors}

\begin{reviewer}
A few of comments and suggestions are listed as follows.

1).The computational complexity of the proposed framework should be analyzed.
\end{reviewer}
\begin{authors}
The analysis of the time complexity of the framework is available.
It was not include it in the manuscript as we tried to avoid those things that could not be explained clearly within the given space limit.
A clear explanation requires an in-depth discussion on the algorithmic structure of the proposed framework since the particularities of our implementation are important for the performance of the proposed approach.
In particular, a thorough attention is needed to the parts of the algorithm that stay unchanged when the input changes.
These parts can and should be precomputed and stored such that their future use implies no additional computational costs (this is partially related to the second comment of this reviewer regarding Sec.~VI-D).
Such an in-depth explanation was planned to be given as a part of another work.
Taking the received feedback into account, we have reconsidered our decision and have included in the manuscript a concise version of the analysis of the time complexity given in what follows.

All the statements below regarding certain aspects of PC expansions should be understood only in the context of our implementation.
Recall that $\nprocs$, $\nnodes$, $\nsteps$, $\pcterms$, and $\qdorder$ are the numbers of processing elements, thermal nodes, time steps, polynomial terms, and quadrature points, respectively.
The construction of PC expansions is based on so-called non-intrusive spectral projections [M. Eldred \etal, 2008] using quadrature rules.
The needed quadrature rule is determined solely by the intended PC expansion.
The structure of the intended PC expansion (\ie, everything except for the coefficients) is determined solely by the desired accuracy, \ie, $\pcorder$, and the parametrization of the problem in terms of independent variables, \ie, $\vZ(\o)$.
Therefore, any two problems that share $\vZ(\o)$ and $\pcorder$ have identical structures of PC expansions and identical quadrature rules.
To give an extreme example, suppose a dual-core platform has three uncertain parameters $\vU'(\o)$, and a 32-core platform has 33 uncertain parameters $\vU''(\o)$.
Then, if the model order reduction procedure leaves two beta variables $\vZ(\o)$ in both cases, the two problems are identical from the perspective of, say, the fourth-order PC, regardless of the input dynamic power profile.
Consequently, the preliminary work for a fix combination of $\vZ(\o)$ and $\pcorder$ should be done only once.
In particular, this means that, for a given architecture and a given accuracy level, an unlimited number of dynamic power profiles can be analyzed at no additional costs.
This is especially advantageous for various optimization procedures, as we note in Sec.~VII.

Going further in detail, for each problem, we compute and store two matrices.
The first one is an $\pcterms \times \qdorder$ matrix, referred to as the projection matrix, which is used to compute the coefficient of a PC expansion by one matrix multiplication.
The second one is an $\pcterms \times \qdorder$ matrix, referred to as the evaluation matrix, which is used to evaluate a PC expansion at the quadrature nodes by one matrix multiplication.
Having said all above, we can now calculate the time complexity of the PC expansions computed inside the proposed framework.
As usual, the costs associated with other preliminary computations (the construction of equivalent thermal RC circuits, \etc), which are constant for a particular problem/platform, are not included.

The iterative procedure in Eq.~7 traverses $\nsteps$ time steps.
Each step yields an $\nnodes \times \pcterms$ matrix that contains the coefficients of a PC expansion for temperature.
This matrix is the sum of two matrices corresponding to the two terms in Eq.~7.
The first matrix is based on the result from the previous iteration and is obtained by one multiplication of an $\nnodes \times \nnodes$ matrix with an $\nnodes \times \pcterms$ matrix.
The second matrix is computed by one multiplication of an $\nnodes \times \nprocs$ matrix with an $\nprocs \times \pcterms$ matrix, representing the coefficients of the current PC expansion for power.
So, the complexity of one step up to this point is $O(\nprocs^2 \, \pcterms)$ since $\nnodes$ is typically proportional to $\nprocs$ (in the experimental results, $\nnodes = 4 \times \nprocs + 12$).
The calculation of the current PC expansion for power is composed of two stages: evaluation and projection.
At the first stages, the current (\ie, from the previous step) PC expansion for temperature is evaluated at the quadrature nodes.
As mentioned earlier, this operation is one multiplication of an $\nprocs \times \pcterms$ matrix with the evaluation matrix; so, this operation is $O(\nprocs \, \pcterms \, \qdorder)$.
Then, the evaluated temperature, \ie, an $\nprocs \times \qdorder$ matrix, and the assignments of the uncertain parameters with respect to the quadrature nodes, which is another $\nprocs \times \qdorder$ matrix, are plugged in into the power model.
The result is an $\nprocs \times \qdorder$ matrix of power.
Denote the complexity of the computations associated with the power model by $O(\Pi(\nprocs, \qdorder))$.
At the projection stage, the coefficients of the current PC expansion for power are calculated by one multiplication of the output of the power model, \ie, an $\nprocs \times \qdorder$ matrix, with the projection matrix; so, this operations is $O(\nprocs \, \qdorder \, \pcterms)$, which is the same as the complexity of the evaluation stage.
The time complexity of one time step is then
\[
  O(\nprocs^2 \, \pcterms + \nprocs \, \qdorder \, \pcterms + \Pi(\nprocs, \qdorder)).
\]
The first term can be dropped as $\nprocs$ is typically much smaller than $\qdorder$.
Taking into consideration all the time steps, we have
\[
  O(\nsteps \, \nprocs \, \qdorder \, \pcterms + \nsteps \, \Pi(\nprocs, \qdorder)).
\]
The expression can be detailed even further by expanding $\pcterms$ and $\qdorder$.
For total-order polynomial spaces, the exact formula for $\pcterms$ is given in Eq.~5.
The limiting behavior of $\pcterms$ with respect to $\nvars$ is $O(\nvars^\pcorder / \pcorder!)$.
For sparse grids, $\qdorder$ does not have such a closed-form formula and depends on the family of the quadrature rule utilized.
The limiting behavior of $\qdorder$ with respect to $\nvars$, however, has been extensively studied.
For example, for Gaussian quadratures, $\log(\qdorder)$ is $O(\log(\nvars))$, which means that the dependency of $\qdorder$ on $\nvars$ is polynomial [F. Heiss \etal, 2008].
In contrast, for full-tensor-product grids, this dependency is exponential.

\done{The time complexity has been analyzed in Sec.~V-D4.}
\end{authors}

\begin{reviewer}
The authors also should compare the estimated stochastic power profiles provided by the proposed framework with those obtained by the MC simulations.
\end{reviewer}
\begin{authors}
The following discussion is partially included in our answer to the third comment of Reviewer 2.

Despite the fact that the proposed framework covers both power and temperature, we initiated this work aiming to the quantification of temperature, not power \perse\ as it had already been addressed by other researchers.
In other words, our main focus was always on temperature, and power is a byproduct on the way to temperature if you will.
Therefore, the experimental results are primarily dedicated to temperature analysis, which makes the main contribution of our research.

Furthermore, power is an intermediate step towards temperature and, thus, resides closer to the sources of uncertainty in the chain of various transformations.
Therefore, the quantification of power is expected to be easier than the quantification of temperature, and any accuracy problems with respect to power are expected to eventually propagate to temperature.
Consequently, the assessment of temperature allows us to draw reasonable conclusions with respect to power.

\done{The absence of comparisons for power has been motivated in Sec.~VII.}

\done{The power/temperature relation with respect to accuracy has been given in Sec.~VII.}
\end{authors}

\begin{reviewer}
2).In section VI.D, the authors mentioned that ``these points along with the corresponding weights are generally precomputed and tabulated;\ldots''. Could the authors clearly explain it?
\end{reviewer}
\begin{authors}
The answer to the first comment partially covers the answer to this one.

A multidimensional quadrature rule, based on a particular construction mechanism such as the Smolyak algorithm, is characterized by its family, accuracy level, and dimensionality.
There is no dependency on anything else.
Once the rule has been chosen, it stays the same regardless of what we would like to integrate.
Consequently, it is a common practice to have the needed rules precomputed and stored in tables for the future use.
One can compute such tables him or herself or use one of the many libraries available online.
In our work, we rely on the library by J. Burkardt from Florida State University; the corresponding reference is included in the manuscript.

\done{The explanations in Sec.~VI-D and App.~D have been improved.}
\end{authors}

\begin{reviewer}
3).Is it fair to demonstrate the efficiency of the proposed framework by using 10\^{}4 samples of MC simulations? The authors might perform the MC simulations until the result is converged, and use this result as the reference solution. Then, the MC simulation is re-performed until it achieves the same accuracy level (compared with the reference solution) as the developed framework.
\end{reviewer}
\begin{authors}
Due to the resource demand of MC sampling, we could not go far beyond $10^5$ samples and consider it to be a sufficiently large number.
As we write in the paper, this range is reasonable according to the experience from other studies and the formulae given in [10].
[S. Chandra \etal, 2010] uses $10^4$.
[L. Cheng \etal, 2011] uses $10^5$.
For example (taken from an earlier publication of the author of [10]), in order to approach the 0.05-probability tail of a probability distribution with confidence 99\% and accuracy 0.005, the estimated number of samples is roughly 12550.
However, the main reason of making $10^4$ MC samples be the etalon for the comparison in Sec. VII is our own experience and observations from the presented tables.
Even though $\eVar$ exhibits a prominent decrease for $10^5$ samples, as we discuss below in response to the fifth question of the reviewer, the other two metrics show that $10^4$ samples are enough.
Besides, we tried to avoid any bias towards the proposed technique, which could happen if we required the MC-based approach to have $10^5$ samples.
\end{authors}

\begin{reviewer}
4).According to Tables III and IV, a few of results show that e\_f might increase as n\_po increases compared with 10\^{}5 samples of MC simulations. Table V shows that as \\eta=1, n\_v at n\_p=16 is larger than that at n\_p=32. Could the authors explain or discuss them?
\end{reviewer}
\begin{authors}
There are three cases where $\ePDF$ increases for $10^5$: 1.50/1.51, 1.59/1.62, and 1.13/1.24.
Our explanation is the following.
As described in the paper, PC expansions provide analytical formulae for the expected values and variances while PDFs can be estimated by sampling these expansions.
Thus, $\ePDF$ differs from $\eExp$ and $\eVar$ by the fact that it is entirely based on sampling.
Hence, the observed marginal differences can be ascribed to the ``two-sided randomness'' of the third error metric, $\ePDF$.
One can also note that there are two more instantiations of this concern for $\eVar$: 66.13/66.70 and 1.56/5.03.
In this case, the results were obtained for a first- and second-order PC expansions, respectively, which are too low to be trustworthy.
Nevertheless, in our opinion, the tables do their main job very well: they clearly capture the overall trend and allow one to draw sound conclusions.
Now, let us turn to Tab. V.
The decrease of $\nvars$ from eight to 11 for 16 and 32 processing elements, respectively, also drew our attention, and we deeply investigated this issue.
These figures are a result of the model order reduction procedure described in App. B.
One component that influences the number of the preserved random variables, $\nvars$, is the placement of the processing element on the die.
As it is described at the beginning of Sec. VII, our floorplans are regular grids of processing elements.
For example, 16 dies is a 4-by-4 grid, a perfect square, while 32 dies is a 8-by-4 grid, a rectangle.
Taking into consideration the assumed correlation function and its radial component, one can note that the first floorplan is more favorable for reduction as more processing elements are located at the same distance from the center of the die.
All the floorplans used in our experiments are available online.
\end{authors}

\begin{reviewer}
5).Why the reported error of 5.71\% for variance is likely to be overestimated? Even though the MC simulation with 10\^{}5 is not reliable (good enough), it does not mean that the report error of the developed framework is overestimated.
\end{reviewer}
\begin{authors}
Yes, the reviewer is right that there is no direct implication here, and the word ``likely''  serves exactly the purpose of emphasizing a degree of our uncertainty.
However, we do believe that the sentence has its own right to be a part of our discussion in the experimental results for the following reason.
To draw an adequate conclusion about the accuracy of our framework, the MC-based approach has no simplifications inside: there is no model order reduction, and the original system of differential equations is being solved directly using traditional numerical techniques.
The quantities estimated by MC sampling converge to the true values almost surely.
Thus, we have strong reasons to believe MC sampling with large numbers of samples.
In Tab. III, we observe a drop of the error for variance from 5.71\% to 1.47\% for $10^4$ and $10^5$ samples, respectively.
First, we note that the estimate with $10^5$ samples is \emph{likely} to be closer to the truth than the one with $10^4$ samples.
Second, the rather large gap between the two values suggests that it is \emph{likely} that the next error change would not be negligible if we could draw, say, $10^6$ samples.
Consequently, it allows us to speculate that the reported error for variance according to our etalon, that is, to $10^4$ MC samples, is \emph{likely} to be overestimated, that is, the actual error is \emph{likely} to be smaller than the reported one of 5.71\%.
\end{authors}

\begin{reviewer}
6).Could the authors explain how to decide the mapping matrix \textbackslash{}tilde\{B\} and why in Appendix A?
\end{reviewer}
\begin{authors}
Yes, we have done this.
\end{authors}

\begin{reviewer}
7).The authors tried to use  ``(...)'' to help the readers to understand the work. However, it is getting kind of annoying if too many of them are used. A typo: ``A description of the latter can also found in the supplementary materials, App. B'' (lines 26\~{}28, page 5).
\end{reviewer}
\begin{authors}
We apologize for this inconvenience, but, as the reviewer said, we were trying to do our best to make the paper transparent to as broad an audience as possible.
The typo found by the reviewer has been fixed.
\end{authors}
