\begin{reviewer}
This paper proposed a novel approach to model transient power-temperature behavior considering process variation. This is a missing piece in statistical power/temperature analysis due to many reasons and technical difficulties. I think the proposed solution may still have some limitations and assumptions, but it has big contribution to the area, and one step closer to practical usage. Experiments are sufficient to validate the proposed framework. The paper is well written and organized as well.
\end{reviewer}
\begin{authors}
Thank you for the kind words.
\end{authors}

\begin{reviewer}
This is an interesting topic, so some of my comments are more like discussions:

1) Power consumption has two components: dynamic power and leakage power. Process variation mainly affects leakage power. I think that’s why the authors focus on the model for statistical leakage power. However, temperature and total power (dynamic and leakage) are highly correlated. Since this paper is focus on modeling transient power-temperature behavior, the accuracy of total power model is important. Dynamic power is related to the run-time operations. In Sec. VII, the dynamic power profiles are obtained by simulations of randomly generated via TGFF, applications defined by DAGs of tasks. Usually the simulation has a size limit, it will be nice to provide some details about how large scale of the chip can be.
\end{reviewer}
\begin{authors}
Let us emphasize first that the workload to be analyzed is an input to the proposed framework without any further restrictions: it is arbitrary and can come from an arbitrary source.
Therefore, it would not make much contribution to the paper if we performed, \eg, cycle-accurate simulations of some applications and used the output of these simulation as an input to our analysis.
Instead, as mentioned by the reviewer, our applications are generated randomly \via\ TGFF.
Apart from a DAG, TGFF generates a set of tables that specify the execution time and dynamic power of each task for each processing element individually.
Then the simulation that we perform consists of a scheduling procedure of the DAG, which determines when and where each task should be executed (we employ a list scheduler for this purpose).
Once this has been decided, the corresponding dynamic power profile is constructed based on the tables provided by TGFF.\footnote{The output files of TGFF used in our experiments are available online.}

We agree with the reviewer that, in reality, the size of the problem at hand can be a concern since power profiles, of course, should be properly captured using adequate simulators of the target platform capable of modeling the power dissipation.
However, to the best of our knowledge, contemporary software packages can handle rather large configurations.
For example, the following interrelated tools can be utilized for this purpose: SimpleScalar, PTscalar, gem5, McPAT, and Wattch.
Also, it is worth noting that the corresponding simulation is undertaken prior to our analysis and does not affect the scaling of the proposed framework.

\begin{actions}
  \action{The absence of requirements to input workloads has been noted in Sec.~IV.}
  \action{Practical aspects of input workloads have been outlined in Sec.~VII.}
\end{actions}
\end{authors}

\begin{reviewer}
2) Question about the thermal model: The authors mentioned the following in Sec. IV-C: ``Given the thermal specification S of the system at hand (see Sec. IV), \ldots'' I’m interested in the modeling of thermal specification as well. But it seems that the authors are referring to the wrong Section (or should provide more details about sub-session?).
\end{reviewer}
\begin{authors}
We use $\system$ as a shortcut to refer to all the information needed for the construction of an equivalent thermal RC circuit of the target electronic system.
The content of $\system$ is described in the second paragraph of Sec.~IV.
We agree with the reviewer that $\system$ is rather abstract; however, it is due to the fact that $\system$ is specific to the way the thermal RC circuit is obtained.
For example, in our experiments, we utilize the block model of HotSpot v5.02 for this purpose.
Then $\system$ reflects the configuration of HotSpot, which includes more than 60 parameters such as the chip thickness, silicon thermal conductivity, silicon specific heat, \etc\ Please refer to ``Temperature-aware Microarchitecture: Modeling and Implementation'' by K. Skadron \etal\ for further details.

\begin{actions}
  \action{The reference to $\system$ from Sec.~V-C has been made clear.}
  \action{The content of $\system$ has been detailed further, Sec.~IV.}
  \action{The references [Kreith, 2000] and [HotSpot] have been replaced by [Skadron, 2004], mentioned above, which is more concrete and well suited for the discussions in the paper.}
\end{actions}
\end{authors}

\begin{reviewer}
In Sec. III, the authors mentioned that this framework works for electronic systems with temperature profiles. Is that chip level? Architecture level? If it’s chip level, the cooling system is not settled yet, so it’s hard to have a final temperature profile;
\end{reviewer}
\begin{authors}
The thermal modeling considered in this paper is at the architecture/system level.
The cooling system is assumed to be decided in order to construct an adequate thermal RC circuit.
Therefore, the thermal package is included in the thermal model, as mentioned in the second paragraph of Sec.~IV.
For example, the thermal package in the experimental results consists of three layers: the thermal interface material, heat spreader, and heat sink.

\begin{actions}
  \action{The system-level scope of the paper has been emphasized in Sec.~IV.}
\end{actions}
\end{authors}

\begin{reviewer}
If it’s architecture level, there might be multiple chips involved with different statistical power behaviors, I’m curious about how the proposed method works in this situation.
\end{reviewer}
\begin{authors}
Let us discuss the comment from two perspectives.

The first potential concern of the comment is that the electronic system at hand can be composed of a number of subsystems with different statistical behaviors.
The proposed framework does not specify any particular way the processing elements are identified at the system level.
Therefore, if the user wishes to analyze a compound system, each subsystem can be treated as one processing element, for example.
If a finer granularity is needed, however, each subsystem can be represented by several processing elements.
Our approach can also be applied to each subsystem individually if it suffices the purpose of the analysis.

The second potential concern of the comment is that multiple fabricated chips, instantiating the same system, can have different statistical behaviors.
The source of uncertainty considered in this work is process variation.
At the design stage, the process parameters of a yet-to-be-fabricated chip are random variables.
Once the chip has been fabricated, these parameters take particular values and remain fixed thereafter, \ie, the process parameters of this particular chip become pure deterministic.
Thus, a fabricated chip alone does not exhibit any statistical behavior; the behavior is deterministic.
However, since this behavior is unknown at the design stage, the power/temperature profiles are stochastic for the designer, and our framework provides the tools to characterizes all possible outcomes of these uncertain profiles.

\begin{actions}
  \action{The description of the considered electronic system has been improved in Sec.~IV.}
\end{actions}
\end{authors}

\begin{reviewer}
3) On Fig 1, it will be nice to use different line shapes as well. When the paper is printed black/white, it’s hard to tell the color difference.
\end{reviewer}
\begin{authors}
Thank you for drawing our attention.
We should have done it right from the very beginning.

\begin{actions}
  \action{Fig.~1 have been adapted for printing in grayscale.}
  \action{The same issue has been addressed with respect to Fig.~3, Fig.~4, Fig.~5, and Fig.~6.}
\end{actions}
\end{authors}
