\begin{reviewer}
This paper proposed a novel approach to model transient power-temperature behavior considering process variation. This is a missing piece in statistical power/temperature analysis due to many reasons and technical difficulties. I think the proposed solution may still have some limitations and assumptions, but it has big contribution to the area, and one step closer to practical usage. Experiments are sufficient to validate the proposed framework. The paper is well written and organized as well.
\end{reviewer}
\begin{authors}
Thank you for the kind words.
\end{authors}

\begin{reviewer}
This is an interesting topic, so some of my comments are more like discussions:

1) Power consumption has two components: dynamic power and leakage power. Process variation mainly affects leakage power. I think that’s why the authors focus on the model for statistical leakage power. However, temperature and total power (dynamic and leakage) are highly correlated. Since this paper is focus on modeling transient power-temperature behavior, the accuracy of total power model is important. Dynamic power is related to the run-time operations. In Sec. VII, the dynamic power profiles are obtained by simulations of randomly generated via TGFF, applications defined by DAGs of tasks. Usually the simulation has a size limit, it will be nice to provide some details about how large scale of the chip can be.
\end{reviewer}
\begin{authors}
A dynamic power profile is an input to our framework without any further restrictions: it is arbitrary and can come from an arbitrary source.
Therefore, it would not make much contribution to the paper if we performed cycle-accurate simulations of some real-life applications, for example.
Instead, our applications are generated randomly via TGFF.
Apart from a DAG, TGFF generates a set of tables that specify the execution time and dynamic power of each task for each processing element individually.
Then the simulation that we perform consists of a scheduling procedure of the DAG, which determines when and where each task should be executed (we employ a list scheduler for this purpose).
Once this has been decided, the corresponding power profile is constructed based on the tables provided by TGFF.
The number of tasks rangers from 20, for the dual-core setup, to around 640, for the 32-core setup.
As mentioned in the paper, the output files of TGFF used in our experiments are available online.
\end{authors}

\begin{reviewer}
2) Question about the thermal model: The authors mentioned the following in Sec. IV-C: “Given the thermal specification S of the system at hand (see Sec. IV), …” I’m interested in the modeling of thermal specification as well. But it seems that the authors are referring to the wrong Section (or should provide more details about sub-session?). In Sec. III, the authors mentioned that this framework works for electronic systems with temperature profiles. Is that chip level? Architecture level? If it’s chip level, the cooling system is not settled yet, so it’s hard to have a final temperature profile; If it’s architecture level, there might be multiple chips involved with different statistical power behaviors, I’m curious about how the proposed method works in this situation.
\end{reviewer}
\begin{authors}
(a) What we mean by a thermal specification of the system is described in the second paragraph of Sec. IV.
$\system$ is a shortcut for us to refer to all the information needed for the construction of an equivalent RC thermal circuit of the system.
One of the most important components of $\system$ is the floorplan of the die.
Our floorplans, with the number of processing elements from two to 32, are constructed as it is described in Sec. VII and are available online.
(b) It is the architecture level.
The thermal package is included in our thermal model.
More precisely, we use the block model of HotSpot v5.02 where the package consists of three layers: the thermal interface material, heat spreader, and heat sink.
It is true that different chips have different power behaviors due to process variation, and this is exactly what we address in this paper: we deliver a probabilistic model that characterizes the whole space of all possible outcomes of the fabrication process.
\end{authors}

\begin{reviewer}
3) On Fig 1, it will be nice to use different line shapes as well. When the paper is printed black/white, it’s hard to tell the color difference.
\end{reviewer}
\begin{authors}
As the reviewer suggested, we have corrected all the figures to make them easily readable when the paper is printed in black and white.
\end{authors}
