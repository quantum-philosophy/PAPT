\begin{reviewer}
The author  replies to the previously proposed questions are satisfactory in general.
\end{reviewer}
\begin{authors}
Thank you.
\end{authors}

\begin{reviewer}
\clabel{5}{1}
However, the reviewr still has the concerns with the answers to the speedup and related equation (3) in the paper.

Foundamentally, the author used different method to solve the thermal equations. For the MC, the author use the traditional stable appraoch such as Runge-Kutta method, which allows any time step used. While  in the proposed statisitcal method using PC, it used the explicit iterative method as shown in (3). This explicit Forward Euler intergattion method will of course be much much faster than the Runge-Kutta method as it will not need to solve (performing LU decompostion) the matrix at all.  But this explicit integration method will only converge for very small time steps (the time step allowed actually depends on the finite difference discretization steps).

Since the explicit iteration formulat (3) is important for the proposed method to consider power and temperature interplay, the proposed method thus has severe limitations due to the time step restirctions.  The author should be provide more explanation on this issue.
\end{reviewer}
\begin{authors}
It is indeed a highly important concern, and we would like to respond with a detailed answer.
We assume that the reviewer is referring to the standard explicit/forward Euler method.
The reviewer is absolutely right that this method has poor stability properties; however, we do not use Euler's method in our work.
As shown in App.~A, the thermal system is solved using an exponential integrator method [Hochbruck, 2010],\footnote{This reference has been added in the revised manuscript.} which we shall demonstrate shortly.

Let us first digress from numerical integration schemes and investigate some aspects of our problem.
Combining Eq.~(1) and Eq.~(2) and dropping the details that are irrelevant for the present discussion, the system of ordinary differential equations under consideration is
\begin{equation}
  \frac{d\vX(\t)}{d\t} = \mA \: \vX(\t) + \mB \: \vP(\t, \vX(\t)) \tag{$\ast$}
\end{equation}
where $\vP$ is a nonlinear function of $\vX$.
A detailed description of the thermal system can be found in App.~A (see also [Ukhov, 2012]); for our purposes, it is sufficient to recall that
\[
  \mA = -\mC^{-\frac{1}{2}} \mG \mC^{-\frac{1}{2}}
\]
where $\mC$ and $\mG$ are the capacitance and conductance matrices, respectively.
By the construction of thermal RC circuits imposed by the nature of the modeled phenomenon [Skadron, 2004], both matrices are real, symmetric, and positive definite,\footnote{It is also explained in the source code of HotSpot (v5.02), namely, in \texttt{temperature\char`_block.c}, lines 278--289, which can be downloaded at \url{http://lava.cs.virginia.edu/HotSpot/}.} and $\mC$ is also diagonal.
It follows that $\mA$ is also real and symmetric as
\[
  \mA^T = (-\mC^{-\frac{1}{2}} \: \mG \: \mC^{-\frac{1}{2}})^T = -(\mC^{-\frac{1}{2}})^T \: \mG^T \: (\mC^{-\frac{1}{2}})^T = \mA.
\]
Since $\mG$ and $\mA$ are real and symmetric, both admit the eigenvalue decomposition [Press, 2007], and the corresponding eigenvectors are orthogonal.
Let this decomposition for $\mA$ be
\[
    \mA = \m{U} \m{\Lambda} \m{U}^T
\]
where $\m{U}$ and $\m{\Lambda}$ are a matrix of the eigenvectors and a diagonal matrix of the eigenvalues $\lambda_i$ of $\mA$, respectively; a similar representation can be obtained for $\mG$.
Since $\mG$ is positive definite, all its eigenvalues are negative by definition.
Due to the relation between $\mG$ and $\mA$ given earlier, the matrices share the same eigenvalues but of opposite signs.
Therefore,
\[
  \lambda_i < 0, \qquad \forall i,
\]
meaning that $\mA$ is a negative-definite matrix.

Now, denote by $\m{J}(\t, \vX(\t))$ the Jacobian matrix of $\vP(\t, \vX(\t))$ with respect to $\vX(\t)$.
It can be shown [Olver, 2006]\footnote{P. Olver and C. Shakiban. \emph{Applied Linear Algebra}. Pearson Prentice Hall, 2006. This book is not cited in the manuscript. Extra chapters of this book---which are also relevant to the present discussion, in particular, Chapters 19 and 20---are available online at \url{http://www.math.umn.edu/~olver/appl.html}.} that the system in Eq.~($\ast$) is asymptotically stable if
\[
  \zeta_i < 0, \qquad \forall i,
\]
where $\zeta_i$ are the eigenvalues of
\begin{equation}
  \mA + \mB \: \m{J}(\t, \vX(\t)). \tag{$\ast\ast$}
\end{equation}
Since the considered problem is assumed to be well posed [Gustafsson, 2008],\footnote{B. Gustafsson. \emph{High Order Difference Methods for Time Dependent PDE}. Springer Series in Computational Mathematics, 2008. This book not cited in the manuscript.} this condition should hold.
In other words, the perturbation of the eigenvalues $\lambda_i$ of $\mA$---which are all negative as it was shown earlier---caused by the nonlinear term results in negative eigenvalues $\zeta_i$.

Let us return to numerical solutions of Eq.~($\ast$).
If the explicit Euler method was utilized, the iterative procedure would be as follows:\footnote{For the clarity of our answer, we shall use a different indexing of the power term than the one used in the manuscript, which has no effect on the generality of the corresponding conclusions.}
\begin{align*}
  \vX_k & = \vX_{k - 1} + \Delta \t \: (\mA \: \vX_{k - 1} + \mB \: \vP_{k - 1}) \\
  & = (\mI + \Delta \t \: \mA) \vX_{k - 1} + \Delta \t \: \mB \: \vP_{k - 1} \\
  & = \hat{\mCF} \vX_{k - 1} + \hat{\mCS} \vP_{k - 1}
\end{align*}
where $\vX_k = \vX(\t_k)$, $\vP_k = \vP(\t_k, \vX(\t_k))$, and $\Delta \t$ is the time step of integration.
It can be shown that $\Delta \t$ needed for the above iterative procedure to be asymptotically stable is dictated by the matrix in Eq.~($\ast\ast$).
More precisely, the following inequality should be satisfied for each eigenvalue $\zeta_i$ of that matrix:
\[
  |1 + \Delta \t \: \zeta_i | < 1.
\]
As it can be seen, and as it was mentioned by the reviewer, there is a strict limitation on $\Delta \t$ due to the narrow stability region inherent to the explicit Euler method:
\[
  \Delta \t < \frac{2}{\max_i |\zeta_i|}.
\]

In our work, we do not use the integration method described above.
Instead, we proceed as follows [Ukhov, 2012].
Multiplying both sides of Eq.~($\ast$) by $e^{- \mA \t}$ and noting that
\[
  e^{-\mA \t} \frac{d\vX(\t)}{d\t} = \frac{d\, e^{-\mA \t} \vX(\t)}{d\t} + \mA e^{-\mA \t} \vX(\t) = \frac{d\, e^{-\mA \t} \vX(\t)}{d\t} + e^{-\mA \t} \mA \vX(\t),
\]
we obtain the exact solution of Eq.~($\ast$) over a time interval $\Delta \t = \t_k - \t_{k - 1}$:
\[
  \vX(\t_k) = e^{\mA \Delta \t} \vX(\t_{k - 1}) + \int_0^{\Delta \t} e^{\mA (\Delta t - \tau)} \mB \: \vP(\t_k + \tau, \vX(\t_k + \tau)) \: d\tau.
\]
Next, the integral on the right-hand side is approximated by assuming that, within $\Delta \t$, the power dissipation does not change and is equal to the power dissipation at $\t_k$, which we write in App.~A.
Then, the approximated solution is
\[
  \vX(\t_k) = e^{\mA \Delta \t} \vX(\t_{k - 1}) + \mA^{-1}(e^{\mA \Delta \t} - \mI) \mB \: \vP(\t_{k - 1}, \vX(\t_{k - 1})),
\]
which leads to our recurrence
\[
  \vX_k = \mCF \: \vX_{k - 1} + \mCS \: \vP_{k - 1}
\]
where
\[
  \mCF = e^{\mA \Delta t} \hspace{1em} \text{and} \hspace{1em} \mCS = \mA^{-1} (e^{\mA \Delta \t} - \mI) \: \mB.
\]
The matrix exponential and matrix inverse needed for the recurrence are efficiently computed using the eigenvalue decomposition of $\mA$ as described in [Ukhov, 2012].
Such a decomposition is always possible since $\mA$ is real and symmetric, which was shown earlier, and it should be computed only once for a particular thermal circuit.

The solution technique that we use in the proposed framework belongs to the family of exponential integrators [Hochbruck, 2010].
This particular member of that family is a one-step method based on the function value from the previous iteration.
This makes the method appear similar to the explicit Euler method, which was also noted by the reviewer.\footnote{Due to this similarity, the method is sometimes referred to as the exponential Euler method; in order to avoid any confusion, we do use this terminology here.}
However, the stability properties of the two approaches are considerably different, which we shall discuss next.

In contrast to Euler's method, the utilized exponential integrator evaluates the linear part of the thermal system exactly, which eliminates potential issues from this side.
In order to investigate the behavior of the nonlinear force term, let us unfold both recurrences, Euler's and ours, and compare the cumulative influences of $\hat{\mCF}$ and $\mCF$:
\[
  \vX_k = \sum_{i = 1}^k \tilde{\mCF}^{k - i} \: \mCS \: \vP_{i - 1}
\]
where $\vX_0 = \mZero$ is taken into account, and $\tilde{\mCF}$ stands for either $\hat{\mCF}$ or $\mCF$.
Due to the negative-definiteness of $\mA$ discussed earlier, $\mCF = e^{\mA \Delta \t}$ is a convergent matrix [Olver, 2006] for any positive $\Delta \t$.
Therefore, under the assumption that the problem is well posed, as our iterative procedure advances, the increasing powers of $\mCF$ prevents the numerical solution from an unbounded growth [Hochbruck, 2010], which is expected from $\hat{\mCF} = \mI + \Delta \t \mA$ with an inadequately chosen time step.

To conclude, the time step used in the recurrence given by Eq.~(3) does not impose limitations on the proposed framework.

\begin{actions}
  \action{The method used to solve the thermal model has been clarified in App.~A.}
  \action{The corresponding reference to [Hochbruck, 2010] has been included in the bibliography.}
\end{actions}
\end{authors}

\begin{reviewer}
\clabel{5}{2}
Actually, going back the speedup issue, the review thinks that a more fair comparision is to use (3) for MC run as well (If PC based method can use it with stable results, then it should work for tranisent anlysis by MC).
\end{reviewer}
\begin{authors}
The exponential iterator method employed in our work is a low-order method as its local error depends linearly on the time step of integration [Hochbruck, 2010].
The applicability of this method is based on our assumption (see App.~A) that, within each step of the input power profiles, the total dissipation of power can be approximated by a constant.
In reality, however, the total power is not constant within $\Delta \t$; in particular, it is due to the interdependency between the leakage power and temperature.
Consequently, our assumption can introduce errors in the power/temperature computations, which we are willing to accept.
The accuracy of the recurrence in Eq.~(3) can be improved using more advanced exponential iterators, for example, linearized multistep exponential methods; however, we did not explore this direction.

At the same time, we did not want to compromise the assessment of the proposed framework presented in the manuscript.
Therefore, we tried to eliminate any extra assumptions from the solution process based on MC sampling.
There are two main aspects in this regard.
First of all, the model order reduction procedure given in App.~B is not performed inside the MC-based approach: MC sampling preserves 100\% of the variance of the problem and, thus, operates on all the random variables.
Second, the MC-based approach solves the problem using a reliable general-purpose solver available in MATLAB.
This solver is based on the fourth- and fifth-order Runge-Kutta formulas and generally delivers sufficiently accurate results, which we let be the etalon for the comparison in Sec.~VII.

As we replied to \cref{5}{9} received in the first revision (that comment was concerned with alternative MC sampling techniques such as quasi-MC sampling), MC sampling was not the focal point of our research, and, therefore, we put all our efforts into the development of the proposed framework rather than into the improvement of the MC-based approach.

\begin{actions}
  \action{In addition to the solution technique used by the MC approach, it has been noted in Sec.~VII that MC sampling preserves the whole variance of the problem.}
\end{actions}
\end{authors}

\begin{reviewer}
\clabel{5}{3}
The author should give the time step used for the transient simulation using~(3).
\end{reviewer}
\begin{authors}
The time step used in the experimental results is given in the last but one paragraph of Sec.~VII, and it is equal to 1$\,\text{ms}$.
The reviewer is right that we did not mention explicitly that it was the time step of the iterative procedure in Eq.~(3); it has been fixed in the revised manuscript.

\begin{actions}
  \action{It has been clarified in Sec.~VII that the reported time step of power and temperature traces is the time step of the iterative procedure in Eq.~(3).}
\end{actions}
\end{authors}

\begin{reviewer}
\clabel{5}{4}
In summary, if (3) is essential for the proposed method, the author should mention this limitation and its impacts on the mentioned contribuations in this paepr.
\end{reviewer}
\begin{authors}
We hope that our answers to the previous comments and the corresponding modifications of the manuscript clarify the important concerns, regarding the limitations of the utilized solution technique, brought up by the reviewer.
Thank you for the thorough revision of the paper.
\end{authors}
