\begin{reviewer}
Using polynomial chaos (PC) approach to capture process variation caused performance issues in VLSI design was first pointed out and developed by authors from [5][17][18]. And it really works well ! Most theoretical part of this paper followed similar derivations as these previous papers. However, comparing with Monte Carlo based approaches, PC may end up doing  lengthy  coefficients solving procedure. No surprise that the authors need additional surrogate modeling and multi-processors to deal with the speedup.

This paper is well written overall. My major concerns are in the experimental results part.

1) The authors used 45nm technology node. But the variation is only 5\%. Is this too low ? Of course variations depend on fab. Different companies may provide different number. But isnt this should be around 30\% to 50\% on average.
\end{reviewer}
\begin{authors}
Our framework is not constrained by any particular range of standard deviations of the variability of the effective channel length due to process variation.
The value of 5\% is an example, which was taken from the literature cited in the experimental results, Sec. VII ([11] and [12]).
\end{authors}

\begin{reviewer}
2) It is not very clear how big the size of the circuits or designs we are looking at.
As far as i known, most papers referenced do have fairly big size examples and circuits. Matlab code unfortunately may not be able to handle such big examples.
But without large size of examples, the scaling up numbers in the tables in this paper are more or less like speculations.
\end{reviewer}
\begin{authors}
The temperature modeling considered in the paper is on the system level, not on the circuit level.
We use the block model of HotSpot v5.02 wherein each processing element is assigned one power-dissipating block of the thermal model (see Sec. V-C).
The number of processing elements in our examples is from two to 32, which translates into 20 to 140 thermal nodes (the relation is $4 \times n + 12$ as described in Sec. VI-C).
\end{authors}

\begin{reviewer}
3) If thermal and power are the key concerns of this paper, it probably makes sense to compare this work with existing papers (i.e. [5][17][18]).
\end{reviewer}
\begin{authors}
The thermal model itself (namely, the construction of an equivalent RC thermal circuit) is a well-established and widely used model implemented in the temperature simulator HotSpot.
Numerous justifications of the model can be found in the corresponding literature by the authors of HotSpot.
The accuracy of the analytical solution described in App. A has also been assessed not only by us [8], but also by other groups of researchers such as L. Thiele et al. [7].
The leakage power modeling is based on SPICE simulations, which is also a rather commonly encountered approach.
However, it is true that we do not have a comparison with any other variation-aware technique.
The reason for this is the following.
Although our approach covers both power and temperature, the main focus of our work is still on temperature; power is a byproduct if you will.
The most relevant, from our perspective, temperature-related works and their limitations are discussed in the paper, Sec. II.
Since we address the transient scenario while others address either the maximal or steady-state temperature, a one-to-one comparison is not straightforward.
\end{authors}
