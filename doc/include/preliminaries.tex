\subsection{Conventional Notations}
Throughout this article, we use the following notations. $\real$ is the set of real number and $\natural{n}$ is the set of integers greater or equal to $n$. Boldface letters denote vectors and matrices, e.g., $\m{M} = \mtx{m_{ij}} \in \real^{n \times m}$ denotes a $n \times m$ real matrix, and $\v{V} = \vec{v_i} \in \real^n$ denotes and an $n$-dimensional real column vector. $\m{M}^T$ and $e^\m{M}$ are the transpose and matrix exponential of $\m{M}$, respectively. $\diag{m_i} \in \real^{n \times n}$ denotes a $n \times n$ diagonal matrix. $\mI_n$ is the $n \times n$ identity matrix, and $\mZero_{n \times m}$ is the $n \times m$ zero matrix; we shall omit these indexes when the actual dimensions of $\mI$ and $\mZero$ are clear from the context.

\subsection{Elements of Probability Theory}
Let $(\outcomes, \sAlgebra, \pMeasure)$ be a complete probability space \cite{durrett2010} where $\outcomes$ is a set of outcomes, $\sAlgebra \subset 2^\outcomes$ is a $\sigma$-algebra on $\outcomes$, and $\pMeasure: \sAlgebra \to [0, 1]$ is a probability measure induced on the measurable space $(\outcomes, \sAlgebra)$. A $\sAlgebra$-measurable function $\rv{X}(\o): \outcomes \to \real$ is called a \definition{random variable} (r.v.). Denote $\exp_\rv{X} = \oExp{\rv{X}(\o)}$ the expectation of $\rv{X}(\o)$ and $\var_\rv{X} = \oVar{\rv{X}(\o)}$ its variance. A $\pMeasure$-measurable, vector-valued function $\mrv{X}(\o): \outcomes \to \real^n$ is called a multivariate random variable (m.r.v.) with $\vExp_\mrv{X} = \oExp{\mrv{X}(\o)}$ and $\mCov_\mrv{X} = \oCov{\mrv{X}(\o)}$ being its expectation vector and covariance matrix, respectively. A \definition{stochastic process} is a parametrized collection of r.v.'s $\{ \rv{X}(\t, \o) \}_{\t \in \sTime}$ where $\sTime$ is a parameter space, which is usually assumed to be the real half line $[0, \infty)$ meaning time. $\{ \mrv{X}(\t, \o) \}_{\t \in \sTime}$ denotes a $n$-dimensional stochastic process. We shall omit the argument $\o \in \outcomes$ when the stochastic nature of the quantity under consideration is clear from the context.

\subsection{Architecture Model} \slabel{architecture-model}
Consider a heterogeneous multiprocessor platform that consists of $\cores$ processing element of any kind and is equipped with a thermal package. We treat the platform as a \emph{thermal system} where the processing elements dissipate power and, consequently, cause temperature changes. Let $\system$ be a \definition{high-level description} of the system that includes the following information related to thermal issues:
\begin{itemize}
  \item The floorplan of the die (the location and dimensions of the processing elements). In the case of 3D ICs, the floorplans of each of the stacked chips.
  \item The configuration of the thermal package (the dimensions of each of the layers).
  \item The thermal parameters of the materials that the die and package are made of (the thermal conductivity, specific heat, convection capacitance, and convection resistance).
\end{itemize}

In this work, all the parameters in $\system$ are assumed to be deterministic; uncertain parameters are introduced in the following subsection.

\subsection{Uncertain Parameters} \slabel{uncertain-parameters}
Suppose the given thermal system depends on a number of \definition{uncertain parameters} that manifest themselves in the deviation of the actual power dissipation from the nominal value and, consequently, in the variations of temperature. We assume that the parameters have been properly analyzed and a set of $\vars$ \emph{mutually independent} r.v.'s has been extracted\footnote{In \sref{pca} we demonstrate this procedure for Gaussian r.v.'s and refer the reader to \cite{xiu2010} for a more general discussion.}. Let $(\outcomes, \sAlgebra, \pMeasure)$ be the corresponding probability space and $\vZ(\o): \outcomes \to \real^\vars$ be the extracted r.v.'s. Denote $\pdf_\vZ(\vz)$ the joint probability density (mass) function of $\vZ(\vz)$ associated with the continuous (discrete) measure $\pMeasure$. Also, without loss of generality, assume that the vector $\vZ(\o)$ is normalized, i.e., $\oExp{\vZ(\o)} = \mZero$ and $\oCov{\vZ(\o)} = \mI$. For brevity, we shall not write explicitly the dependency on $\vZ(\o)$ and use $\vZ$ or $\o$ instead, e.g., $f(\vZ)$ and $f(\o)$ should be understood as $f(\vZ(\o))$.

\subsection{System Profiles} \slabel{system-profiles}
A \definition{system profile} $\prof{\m{Q}}$ of a quantity $Q$ over a time interval $\period$ is defined as a tuple $(\part, \m{Q})$ where $\part = \{ 0 = \t_1 < \dotsc < \t_{\steps + 1} = \period \}$ is a partition of $\period$ into $\steps$ subintervals $\{ \dt_i = \t_{i+1} - \t_i \}_{i = 1}^{\steps}$, and $\m{Q} \in \real^{\cores \times \steps}$ is a matrix that captures the values of $Q$ for all $\cores$ processing elements at all $\steps$ time intervals. In particular, we are interested in the power and temperature profiles denoted by $\prof{\mP}$ and $\prof{\mTO}$, respectively (hereafter, $P$ stands for power and $\T$ for temperature). Since the system under consideration is stochastic, we shall distinguish between deterministic (nominal) and stochastic power and temperature profiles. In the stochastic case, $\prof{\mP(\o)}$ and $\prof{\mTO(\o)}$ contain r.v.'s.
