\subsection{Architecture Model}
Consider a multiprocessor system that comprises a number of processing elements of any kind and is equipped with a thermal package. Let $\system$ be a set of parameters that entirely describes the system and is called the \definition{system configuration}. $\system$ includes the following information:
\begin{itemize}
  \item The floorplan of the die (the location and dimensions of the processing elements).
  \item The configuration of the thermal package (the dimensions of each of the layers).
  \item The thermal parameters of the materials that the die and package are made of (the thermal conductivity, specific heat, convection capacitance, and convection resistance).
\end{itemize}

\subsection{Deterministic Power Model} \slabel{deterministic-power-model}
The total power dissipation of the system with $\cores$ processing elements at time $\t$ is modeled as the following:
\[
  \vP(\vTO(\t), \t) = \vP_\dyn(\t) + \vP_\leak(\vTO(\t))
\]
where $\vP(\t)_\dyn, \vP(\t)_\leak \in \real^\cores$ are the dynamic and leakage power, respectively, and $\vTO(\t) \in \real^\cores$ is temperature. The leakage power is modeled as a function of temperature due to the well-known, strong interdependency between them.

\subsection{Deterministic Thermal Model} \slabel{deterministic-thermal-model}
Given a system configuration $\system$, an equivalent thermal RC circuit with $\nodes$ \definition{thermal nodes} is constructed \cite{kreith2000}. The structure of the circuit depends on the intended level of details defining the accuracy of the final model.

The thermal behaviour of the circuit is modeled with the following system of differential-algebraic equations (DAEs) with the state-space dimension equal to $\nodes$:
\begin{align}
  & \mC \frac{d\vTI(\t)}{d\t} + \mG \vTI(\t) = \mM \vP(\vTO(\t), \t) \elabel{fourier} \\
  & \vTO(\t) = \mM^T \vTI(\t) + \vTO_\amb \nonumber
\end{align}
$\mC, \mG \in \real^{\nodes \times \nodes}$ are the matrices of the thermal capacitance and conductance, respectively. $\vTI \in \real^\nodes$ is the state vector of the system, which corresponds to the difference between the temperature of the thermal nodes and the ambient temperature. $\vP \in \real^\cores$ is the input vector of the power dissipation of the processing elements, and $\mM \in \real^{\nodes \times \cores}$ is its mapping matrix to the thermal nodes. $\vTO \in \real^\cores$ is the output vector of the system, which is the temperature of the processing elements. Finally, $\vTO_\amb \in \real^\cores$ is the vector of the ambient temperature.

\subsection{Deterministic System Profiles} \slabel{deterministic-system-profiles}
A \definition{power profile} of the system over a time interval $\period$ is defined as a tuple $\pP$. $\pTime = \{ 0 = \t_0 < \dots < \t_{\steps} = \period \}$ is a partition of $\period$ into $\steps$ subintervals $\dt_i = \t_{i+1} - \t_i$. $\mP \in \real^{\cores \times \steps}$ is a matrix of the corresponding power dissipation where the $i$th column vector, $\vP_i \in \real^\cores$, represents the power consumption of the processing elements at the \emph{beginning} of the $i$th time interval, $\dt_i$.

A \definition{temperature profile} of the system with respect to $\pP$ is defined as a tuple $\pT$. $\pTime$ remains the same as for the power profile, and $\mTO \in \real^{\cores \times \steps}$ is a matrix of the corresponding temperature where the $i$th column vector, $\vTO_i \in \real^\cores$, represents the temperature of the processing elements at the \emph{end} of the $i$th time interval, $\dt_i$.

\subsection{Elements of Probability Theory} \slabel{probability-theory}
Let $(\outcomes, \sAlgebra, \pMeasure)$ be a complete probability space \cite{durrett2010} where $\outcomes$ is a set of outcomes, $\sAlgebra$ is a $\sigma$-algebra on $\outcomes$, and $\pMeasure$ is a probability measure induced on the measurable space $(\outcomes, \sAlgebra)$. A $\sAlgebra$-measurable function $\rv{X}: \outcomes \to \real$ is called a \definition{random variable} (r.v.). Denote $\exp_\rv{X}$ the expectation of $\rv{X}$ and $\var_\rv{X}$ its variance. A $\pMeasure$-measurable, vector-valued function $\mrv{X}: \outcomes \to \real^n$ is called a multivariate random variable (m.r.v.) with $\vExp_\mrv{X} = \oExp{\mrv{X}}$ and $\mCov_\mrv{X} = \oCov{\mrv{X}}$ being its expectation vector and covariance matrix, respectively.

A \definition{stochastic process} is a parametrized collection of r.v.'s $\{ \rv{X}(\t) \}_{\t \in \sTime} \subset \real$ where $\sTime$ is the parameter space, which usually represents time. $\{ \mrv{X}(\t) \}_{\t \in \sTime} \subset \real^n$ denotes a $n$-dimensional stochastic process.

The covariance matrix $\mCov_\mrv{X}$ is necessarily a real symmetric matrix. Therefore, it can be decomposed into the product $\mCov_\mrv{X} = \m{U}_{\mCov_\mrv{X}} \m{V}_{\mCov_\mrv{X}} \m{U}_{\mCov_\mrv{X}}^T$ using the eigenvalue decomposition \cite{press2007} where $\m{U}_{\mCov_\mrv{X}}$ and $\m{V}_{\mCov_\mrv{X}}$ are an orthogonal matrix of the eigenvectors and a diagonal matrix of the eigenvalues of $\mCov_\mrv{X}$, respectively. Define the operator $\oEigen{\cd} = \m{U}_{(\cd)} {\m{V}_{(\cd)}}^{1/2}: \real^n \to \real^n$. Consequently, a m.r.v. $\mrv{X}$ can be normalized as $\mrv{X} = \oEigen{\mCov_\mrv{X}} \mrv{Y} + \oExp{\mrv{X}}$ where $\oExp{\mrv{Y}} = \mZ$ and $\mCov_\mrv{Y} = \mI$.
