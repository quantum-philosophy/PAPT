\subsection{Conventional Notations}
Throughout this article, we use the following notations. $\real$ and $\natural$ are the sets of real numbers and nonnegative integers, respectively. Boldface letters denote vectors and matrices, e.g., $\m{M} = \mtx{m_{ij}} \in \real^{n \times m}$ denotes a $n \times m$ real matrix, and $\v{V} = \vec{v_i} \in \real^n$ denotes and an $n$-dimensional real column vector. $\m{M}^T$ and $e^\m{M}$ are the transpose and matrix exponential of $\m{M}$, respectively. $\diag{m_i} \in \real^{n \times n}$ denotes a $n \times n$ diagonal matrix. $\mI_n$ is the $n \times n$ identity matrix, and $\mZ_{n \times m}$ is the $n \times m$ zero matrix; we shall omit these indexes when the actual dimensions of $\mI$ and $\mZ$ are clear from the context. Finaly, $\sI_n = \{ 0, \dots, n - 1 \} \subset \natural^n$ denotes the set of first $n$ natural numbers.

\subsection{Elements of Probability Theory}
Let $(\outcomes, \sAlgebra, \pMeasure)$ be a complete probability space \cite{durrett2010} where $\outcomes$ is a set of outcomes, $\sAlgebra \subset 2^\outcomes$ is a $\sigma$-algebra on $\outcomes$, and $\pMeasure: \sAlgebra \to [0, 1]$ is a probability measure induced on the measurable space $(\outcomes, \sAlgebra)$. A $\sAlgebra$-measurable function $\rv{x}(\o): \outcomes \to \real$ is called a \definition{random variable} (r.v.). Denote $\exp_\rv{x} = \oExp{\rv{x}(\o)}$ the expectation of $\rv{x}(\o)$ and $\var_\rv{x} = \oVar{\rv{x}(\o)}$ its variance. A $\pMeasure$-measurable, vector-valued function $\mrv{x}(\o): \outcomes \to \real^n$ is called a multivariate random variable (m.r.v.) with $\vExp_\mrv{x} = \oExp{\mrv{x}(\o)}$ and $\mCov_\mrv{x} = \oCov{\mrv{x}(\o)}$ being its expectation vector and covariance matrix, respectively. A \definition{stochastic process} is a parametrized collection of r.v.'s $\{ \rv{x}(\o, \t) \}_{\t \in \sTime}$ where $\sTime$ is a parameter space, which is usually assumed to be the real half line $[0, +\infty)$ meaning time. $\{ \mrv{x}(\o, \t) \}_{\t \in \sTime}$ denotes a $n$-dimensional stochastic process. We shall omit the argument $\o \in \outcomes$ when the stochastic nature of the quantity under consideration is clear from the context.

The covariance matrix is necessarily a real, symmetric matrix. Therefore, it admits the eigenvalue decomposition \cite{press2007} in the form $\mCov_\mrv{x} = \m{V}_{\mCov_\mrv{x}} \m{\Lambda}_{\mCov_\mrv{x}} \m{V}_{\mCov_\mrv{x}}^T$ where $\m{V}_{\mCov_\mrv{x}}$ and $\m{\Lambda}_{\mCov_\mrv{x}}$ are an orthogonal matrix of the eigenvectors and a diagonal matrix of the eigenvalues of $\mCov_\mrv{x}$, respectively. Define the operator $\oEigen{\cd} = \m{V}_{(\cd)} {\m{\Lambda}_{(\cd)}}^{1/2}$. In this notation, $\mCov_\mrv{x} = \oEigen{\mCov_\mrv{x}} \oEigen{\mCov_\mrv{x}}^T$. Consequently, $\mrv{x}$ can be normalized as $\mrv{x} = \oEigen{\mCov_\mrv{x}} \mrv{y} + \vExp_\mrv{x}$ where $\oExp{\mrv{y}} = \mZ$ and $\oCov{\mrv{y}} = \mI$.

\subsection{Architecture Model} \slabel{architecture-model}
Consider a multiprocessor platform that comprises $\cores$ processing elements of any kind and is equipped with a thermal package. Let $\system$ be a \definition{high-level description} of the platform that includes the following information:
\begin{itemize}
  \item The floorplan of the die (the location and dimensions of the processing elements). In the case of 3D ICs, the floorplans of each of the stacked chips.
  \item The configuration of the thermal package (the dimensions of each of the layers).
  \item The thermal parameters of the materials that the die and package are made of (the thermal conductivity, specific heat, convection capacitance, and convection resistance).
\end{itemize}

Suppose the given system depends on a number of \definition{uncertain parameters}, i.e., r.v.'s that are possibly correlated. Let $(\outcomes, \sAlgebra, \pMeasure)$ be the corresponding probability space and $\vU(\o) = \vec{\param_i(\o)} \in \real^\params$, $\o \in \outcomes$, be a $\params$-dimensional vector of these parameters. In this work, we assume that the uncertainties $\vU(\o)$, $\o \in \outcomes$, are a result of the process variation and manifest themselves in the deviation of the actual power dissipation of the system from the nominal value (see \sref{power-model}). In this case, $\outcomes$ represents the outcomes of the manufacturing process, which the given multiprocessor system is a product of.

\subsection{System Profiles} \slabel{system-profiles}
Let $\prof{\m{Q}}$ be a \definition{system profile} of a quantity $Q$ over a time interval $\period$ where $\part = \{ 0 = \t_0 < \dots < \t_{\steps} = \period \}$ is a partition of $\period$ into $\steps$ subintervals $\dt_i = \t_{i+1} - \t_i$, and $\m{Q} \in \real^{\cores \times \steps}$ is a matrix that captures the values of $Q$ for all $\cores$ processing elements at all $\steps$ time intervals. In particular, we are interested in the power and temperature profiles denoted by $\prof{\mP}$ and $\prof{\mTO}$, respectively (hereafter, $P$ stands for power and $\Theta$ for temperature). As mentioned in \sref{architecture-model}, the system under consideration is affected by the process variation, therefore, we make a distinction between deterministic (nominal) or stochastic power and temperature profiles. In the stochastic case, each column of a system profiles is a m.r.v. with a known multivariate probability distribution.
