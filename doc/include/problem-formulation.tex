Consider a heterogeneous multiprocessor system that consists of $\cores$ processing elements and is equipped with a thermal package. The processing elements are the active components of the system identified at the intended level of granularity (processors, ALUs, caches, registers, \etc). The system is characterized by its power and temperature profiles, which we define as follows. A power profile $\profP$ over a time interval $\period$ is a tuple composed of a time partition $\part = \{ 0 = \t_0 < \dotsc < \t_{\steps} = \period \}$ with $\steps$ subintervals $\{ \dt_i = \t_i - \t_{i - 1} \}_{i = 1}^{\steps}$ and a matrix $\mP \in \real^{\cores \times \steps}$ that captures the power dissipation of all $\cores$ processing elements at the \emph{beginning} of all $\steps$ time intervals. The definition of a temperature profile $\profT$ is similar to the one for power except the fact that the temperature is tracked at the \emph{end} of the time intervals. This distinction is made merely for convenience and can be thought as: the power is applied first, and the temperature follows it with a time delay due to inertia.

Let $\system$ be a thermal specification of the system defined as a collection of temperature-related information: (a) the floorplans of the active layers of the chip; (b) the geometry of the thermal package; (c) the thermal parameters of the materials that the chip and package are made of. In this work, $\system$ is deterministic. In addition, the system depends on a set of uncertain parameters $\vU(\o)$---hereafter, $\o$ hints randomness behind the quantity and will be formally defined later on---which manifest themselves in deviation of the actual power dissipation from nominal values and, consequently, in deviation of temperature from the one corresponding to the nominal power consumption. Therefore, we shall distinguish between deterministic and stochastic profiles. In the latter case, the power and temperature profiles are denoted by $\profP{\o}$ and $\profT{\o}$, respectively.

In this work, we aim to develop an UQ framework for power-temperature analysis (PTA) of multiprocessor systems where the actual power dissipation and temperature are unknown due to their dependency on the set of uncertain parameters $\vU(\o)$. The user of the framework is required (a) to provide a thermal specification of the platform $\system$; (b) to have some knowledge (or belief) on the probability distribution of $\vU(\o)$, \eg, the joint distribution or a set of marginals with a correlation matrix (discussed in \sref{uncertain-parameters}); (c) to be able to evaluate the actual power dissipation given a particular assignment of $\vU(\o) = \vU$ (discussed in \sref{power-model}). The framework is then should provide the user with a technique to analyze an arbitrary nominal dynamic power profile $\profPdyn$ of the system and obtain the corresponding stochastic power $\profP{\o}$ and temperature $\profT{\o}$ profiles with the desired level of accuracy and a low computational cost.

Here, we would like to point out that the last requirement for the framework, \ie, low costs, is essential since a proper analysis of such a stochastic system is always possible to attain via the MC sampling: it is guaranteed by the central limit theorem \cite{durrett2010}. However, what makes the difference is the cost to pay as MC-based approaches are computationally expensive, often infeasible, in practice due to large numbers of simulations needed to obtain reliable estimates \cite{xiu2010, maitre2010, diaz-emparanza2002}.
