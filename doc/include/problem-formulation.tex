The probability space that we shall reside in is defined as a triple $(\outcomes, \sAlgebra, \pMeasure)$ where $\outcomes$ is a set of outcomes, $\sAlgebra \subseteq 2^\outcomes$ is a $\sigma$-algebra on $\outcomes$, and $\pMeasure: \sAlgebra \to [0, 1]$ is a probability measure \cite{durrett2010}. Loosely speaking, an $n$-dimensional \rv\ is then a mapping $\v{X}: \o \in \outcomes \mapsto \v{X}(\o) \in \real^n$. In what follows, the probability space $(\outcomes, \sAlgebra, \pMeasure)$ is always implied.

Consider a heterogeneous multiprocessor system that consists of $\cores$ processing elements and is equipped with a thermal package. The processing elements are the active components of the system identified at the intended level of granularity (processors, ALUs, caches, registers, \etc). The system is characterized by its power and temperature profiles, which we define as follows. A power profile $\profP$ over a time interval $\period$ is a tuple composed of a time partition $\part = \{ 0 = \t_0 < \dotsc < \t_{\steps} = \period \}$ with $\steps$ subintervals $\{ \dt_i = \t_i - \t_{i - 1} \}_{i = 1}^{\steps}$ and a matrix $\mP \in \real^{\cores \times \steps}$ that captures the power dissipation of all $\cores$ processing elements at the \emph{beginning} of all $\steps$ time intervals. The definition of a temperature profile $\profT$ is the same as for power except that the temperature is tracked at the \emph{end} of the time intervals.

Let $\system$ be a thermal specification of the system defined as a collection of temperature-related information: (a) the floorplans of the active layers of the chip; (b) the geometry of the thermal package; (c) the thermal parameters of the materials that the chip and package are made of. In this work, $\system$ is deterministic. In addition, the system depends on a set of uncertain parameters, denoted by a random vector $\vU(\o)$, $\o \in \outcomes$, which manifest themselves in deviations of the actual power dissipation from nominal values and, consequently, in deviations of temperature from the one corresponding to the nominal power consumption. Therefore, we shall distinguish between deterministic and stochastic profiles. In the latter case, the power and temperature profiles are denoted by $\profP{\o}$ and $\profT{\o}$, respectively.

In this work, we aim to develop an UQ framework for power-temperature analysis (PTA) of multiprocessor systems where the actual power dissipation and temperature are unknown due to their dependency on the set of uncertain parameters $\vU(\o)$. The user of the framework is required (a) to provide a thermal specification of the platform $\system$; (b) to have some knowledge (or belief) on the probability distribution of the uncertain parameters (discussed in \sref{uncertain-parameters}); (c) to specify a power model, in which $\vU(\o)$ is an input (discussed in \sref{power-model}). The framework provides the user with the tools to analyze the system under an arbitrary workload and obtain the corresponding stochastic power $\profP{\o}$ and temperature $\profT{\o}$ profiles with a desired level of accuracy and at low computational cost.
