The major steps of our technique are depicted in \fref{algorithm}:

\stage{1}{Uncertain Parameters (\sref{uncertain-parameters}).} The PC expansions operate on mutually independent \rvs. The uncertain parameters $\vU(\o)$ might not satisfy this requirement and, therefore, should be preprocessed; we shall denote the corresponding independent set of \rvs\ by $\vZ(\o)$.

\stage{2}{Power Model (\sref{power-model}).} The user specifies the power model of the system via a ``black-box'' functional $\f$, which outputs the total power $\vP(\t, \o)$ for a particular temperature $\vTO(\t, \o)$ and an outcome of the parameters $\vU(\o)$.

\stage{3}{Thermal Model (\sref{thermal-model}).} With respect to the thermal specification $\system$ defined in \sref{problem-formulation}, a mathematical model of the thermal system is constructed. The thermal model closely interacts with the power model from \stage{2}\ and produces the corresponding temperature profile.

\stage{4}{Surrogate Model (\sref{polynomial-chaos}).} The surrogate model is attained by traversing the desired time span and gradually constructing polynomial expansions, in terms of the processed uncertain parameters $\vZ(\o)$ from \stage{1}, for the stochastic power and temperature profiles. The output is essentially a substitute for the model produced at \stage{3}\ with respect to the power model determined at \stage{2}.

\stage{5}{Post-processing (\sref{output-processing}).} The computed PC expansions are analyzed in order to obtain the desired characteristics of the system, \eg, \cdfs, \pdfs, moments.
