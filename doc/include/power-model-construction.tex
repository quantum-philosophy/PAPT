Since the influence of the process variation on the dynamic power is negligible \cite{juan2012, srivastava2010}, we let $\vP_\dyn(\t, \o) \equiv \vP_\dyn(\t)$. Therefore, in \eref{power-model}
\[
  f_\dyn(\vP_\dyn(\t), \o) = \vP_\dyn(\t)
\]
Concerning the leakage part, we focus on the subthreshold leakage where the source of uncertainty is the effective channel length $\Leff$, as in \cite{juan2012}. The variations of $\Leff$ are split into global (inter-die) and local (intra-die) parts, cf. \cite{juan2012, srivastava2010, shen2009}. The global offset by definition is shared among all the processing elements, i.e., it is a single r.v., whereas there are $\cores$ distinct r.v.'s for the local deviations. Therefore, the channel length within the $i$th processing element is given as
\[
  \Leff_i(\o) = \nLeff + \gLeff(\o) + \lLeff_i(\o)
\]
where $\nLeff$ is the nominal channel length, $\gLeff(\o)$ is a zero-mean r.v. that represents the global variation, and $\lLeff_i(\o)$ is a zero-mean r.v. for the local part. Consequently, the uncertain parameters of the current model are $\vU(\o) = \vec{\lLeff_1(\o), \dotsc, \lLeff_\cores(\o), \gLeff(\o) }$. The local variations are known to be highly correlated, hence, we should account for this as well. To this end, without loss of generality, we use the exponential model of correlations, as in \cite{shen2009}, given by
\begin{equation} \elabel{covariance}
  \oCov{\lLeff_i(\o), \lLeff_j(\o)} \propto e^{-r^2/\eta^2}
\end{equation}
where $r$ is the distance between the centers of the $i$th and $j$th processing elements, and $\eta$ is the correlation length.

Having intrisic sources of variations identified, we need to construct a leakage model with these sources as inputs. One additional and, moreover, essential parameter should be temperature. Such a model can be obtained by performing a set of SPICE-type simulations of reference electrical circuits for each of the processing elements with various combinations of the channel length and temperature. Then, surface fitting procedures are carried out in order to obtain analytical expressions for the nominal leakage currents $\{ I_{\leak, i}(\Leff, \T) \}_{i = 1}^\cores$, which are further scaled up to the power level\footnote{Details of the implementation are given in \sref{experimental-results}.}. Therefore, in \eref{power-model}
\[
  f_{\leak, i}(\T_i(\t, \o), \o) = \alpha_i I_{\leak, i}(\Leff_i(\o), \T_i(\t, \o))
\]
for $i \in \{ 1, \dotsc, \cores \}$ where $\alpha_i$ are scaling coefficients.
