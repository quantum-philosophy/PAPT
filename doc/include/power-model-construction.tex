\eref{power-model} is decomposed into a sum of dynamic and leakage components denoted respectively by $\f_\dyn(\vP_\dyn(\t), \o)$ and $\f_\leak(\vTO(\t, \o), \o)$. The influence of the process variation on the dynamic power is known to be negligibly small \cite{juan2011, juan2012, srivastava2010}; therefore, we let $\f_\dyn(\vP_\dyn(\t), \o) = \vP_\dyn(\t)$. On the other hand, the variability of the leakage power is significant, in which the subthreshold current contributes the most \cite{juan2011, juan2012}. Hence, we focus on the subthreshold leakage and, more specifically, on the effective channel length $\Leff$. The variations of $\Leff$ are split into global (inter-die) and local (intra-die) parts \cite{juan2011, juan2012, shen2009, srivastava2010}. The global offset is shared among all the processing elements, i.e., it is a single r.v., whereas there are $\cores$ distinct r.v.'s for the local deviations. Therefore, the channel length within the $i$th processing element is given as
\[
  \Leff_i(\o) = \nLeff + \gLeff(\o) + \lLeff_i(\o)
\]
where $\nLeff$ is the nominal channel length, $\gLeff(\o)$ is a zero-mean r.v. that represents the global variation, and $\lLeff_i(\o)$ is a zero-mean r.v. for the local part. Consequently, the uncertain parameters of the current model are $\vU(\o) = \vec{\lLeff_1(\o), \dotsc, \lLeff_\cores(\o), \gLeff(\o) }$. $\gLeff(\o)$ is typically assumed to be uncorrelated with respect to $\lLeff_i(\o)$ while the later are known to be highly correlated among each other. Without loss of generality, we employ the following radial model of correlation for $\lLeff_i(\o)$ \cite{ghanem1991, cheng2011}:
\begin{equation} \elabel{correlation-matrix}
  \oCorr{\lLeff_i(\o), \lLeff_j(\o)} = e^{-|r_i - r_j|/\corrLength}
\end{equation}
where $r_i$ is the distance between the $i$th processing element and the center of the die, and $\corrLength$ is the correlation length. For convenience, the resulting correlation matrix is extended by one dimension to pack $\gLeff(\o)$ and $\lLeff_i(\o)$ together. In this case, the matrix obtains one additional non-zero element on the diagonal. Taking into account the variances of the r.v.'s, the covariance matrix denoted by $\mCov_\vU$ is formed.

Having intrinsic sources of variations identified, we need to construct a leakage model with these sources and temperature as inputs. Such a model can be obtained by, e.g., performing a set of SPICE-type simulations of reference electrical circuits. In our example, we employ a circuit based on the Predictive Technology Model \cite{ptm}. A surface fitting procedure is then carried out on the output data, and the following exponential model is acquired:
\begin{equation} \elabel{leakage-current}
  I_\leak(\Leff, \T) = \text{exp} \big\{ \alpha_0 + \alpha_1 \Leff + \alpha_2 \T + \alpha_3 \Leff \T + \alpha_4 \T^2 \big\}
\end{equation}
where $\alpha_i$ are fitting coefficients. The reference leakage current $I_\leak(\Leff, \T)$ is further scaled up to the power level of each of the processing elements. Finally, in \eref{power-model} we have
\[
  \f_{\leak, i}(\T_i(\t, \o), \o) = \beta_i I_{\leak, i}(\Leff_i(\o), \T_i(\t, \o))
\]
for $i \in \{ 1, \dotsc, \cores \}$ where $\beta_i$ are scaling coefficients.
