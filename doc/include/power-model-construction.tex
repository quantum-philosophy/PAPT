The influence of the process variation on the dynamic power is known to be negligibly small \cite{juan2012, srivastava2010}. Taking this into consideration, we let $\vP_\dyn(\t, \o) \equiv \vP_\dyn(\t)$. Hence, in \eref{power-model} we have
\[
  f_\dyn(\vP_\dyn(\t), \o) = \vP_\dyn(\t)
\]
On the other hand, the variability of the leakage part is significant and should be properly analyzed and taken into account. The leakage current consists of two major parts: the subthreshold and gate currents. The former is known to be highly sensible to the process variation while the later recently became less important \cite{juan2012}. Consequently, we focus on the subthreshold leakage and, more specifically, on the effective channel length $\Leff$. The variations of $\Leff$ are split into global (inter-die) and local (intra-die) parts, cf. \cite{juan2012, srivastava2010, shen2009}. The global offset is by definition shared among all the processing elements, i.e., it is a single r.v., whereas there are $\cores$ distinct r.v.'s for the local deviations. Therefore, the channel length within the $i$th processing element is given as
\[
  \Leff_i(\o) = \nLeff + \gLeff(\o) + \lLeff_i(\o)
\]
where $\nLeff$ is the nominal channel length, $\gLeff(\o)$ is a zero-mean r.v. that represents the global variation, and $\lLeff_i(\o)$ is a zero-mean r.v. for the local part. Consequently, the uncertain parameters of the current model are $\vU(\o) = \vec{\lLeff_1(\o), \dotsc, \lLeff_\cores(\o), \gLeff(\o) }$. The local variations are known to be highly correlated, hence, we should account for this as well. To this end, without loss of generality, we use the following radial model of correlations \cite{ghanem1991} given by
\begin{equation} \elabel{correlation-matrix}
  \oCorr{\lLeff_i(\o), \lLeff_j(\o)} = e^{-|r_i - r_j|/\corrLength}
\end{equation}
where $r_i$ is the distance between the $i$th processing element and the center of the die, and $\corrLength$ is the correlation length.

Having intrisic sources of variations identified, we need to construct a leakage model with these sources as inputs. One additional and, moreover, essential parameter should be temperature. Such a model can be obtained by performing a set of SPICE-type simulations of reference electrical circuits for each of the processing elements with various combinations of the channel length and temperature. In our example, we employ a circuit base on the Berkeley MOSFET transistor model BSIM4 \cite{bsim4}, which is callibrated according to the Predictive Technology Model (PTM HP) \cite{ptm}. Then, a surface fitting procedure is carried out on the output data from the simulations, and the following exponential model is acquired:
\begin{align} \elabel{leakage-current}
  I_\leak(\Leff, \T) = \text{exp} \big\{ a_{00} & + a_{10} \Leff + a_{01} \T \\
  & {} + a_{20} \Leff^2 + a_{11} \Leff \T + a_{02} \T^2 \big\} \nonumber
\end{align}
where $a_{ij}$ are fitting coefficients. The reference leakage current $I_\leak(\Leff, \T)$ is further scaled up to the power level of each of the processing elements. Finally, in \eref{power-model} we have
\[
  f_{\leak, i}(\T_i(\t, \o), \o) = \alpha_i I_{\leak, i}(\Leff_i(\o), \T_i(\t, \o))
\]
for $i \in \{ 1, \dotsc, \cores \}$ where $\alpha_i$ are scaling coefficients.
