\subsection{Power Model} \slabel{power-model}
In this section, we introduce the power model used in conjunction with the thermal model presented in \sref{thermal-model}. The total power dissipation of the system with $\cores$ processing elements is defined in the following abstract form:
\begin{equation} \elabel{power-model}
  \vP(\t, \o) = \f\Big(\vP_\dyn(\t), \vTO(\t, \o), \vZ(\o)\Big)
\end{equation}
where $\f: \real^\cores \times \real^\cores \times \real^\vars \to \real^\cores$ is an arbitrary function, possibly a ``black box'', of the nominal dynamic power $\vP_\dyn(\t)$, stochastic temperature $\vTO(\t, \o)$, and uncertain parameters $\vZ(\o)$. In this work, the function is assumed to be smooth in $\vZ(\o)$ and to have a finite variance, which is applicable to most physical systems \cite{xiu2002}. It can be seen that the definition of $\f$ provides a great flexibility to account for such effects as the well-known, strictly nonlinear interdependency between the leakage current and temperature \cite{srivastava2010, liu2007}. In this case, one can split the function into dynamic $\f_\dyn(\vP_\dyn(\t), \o)$ and leakage $\f_\leak(\vTO(\t, \o), \o)$ parts and define appropriate models for the components; this partition is to be further discussed in \sref{illustrative-example}.


\subsection{Thermal Model} \slabel{thermal-model}
We move on to \stage{3}\ where the thermal model of the multiprocessor system is to be established.
Given the thermal specification $\system$ of the considered platform (the floorplan of the die, the configuration of the thermal package, \etc), we employ HotSpot (v5.02) \cite{hotspot} in order to construct the equivalent thermal RC circuits of the system.
Specifically, we are interested in the coefficient matrices $\mCF(\t)$ and $\mCS(\t)$ in \eref{fourier-system} (see also \fref{algorithm}), which HotSpot helps us to compute by providing the corresponding capacitance and conductance matrices of the system as described in \aref{thermal-model}.
In this case, thermal packages are modeled with three layers, and the relation between the number of processing elements and the number of thermal nodes is given by $\nnodes = 4 \nprocs + 12$.
An example of such a circuit for a dual-core platform is depicted in \fref{circuit}.

To conclude, the power and thermal models of the platform are now acquired, and we are ready to construct the corresponding surrogate model \via\ PC expansions.


\subsection{Expansion in Orthogonal Polynomials}
The goal is to transform the ``problematic'' term in \eref{recurrence}, \ie, the power term defined by \eref{power-model}, in such a way that the recurrence in \eref{recurrence} becomes computationally tractable. Our solution is the construction of a surrogate model for the power model in \eref{power-model}, which we further propagate through \eref{recurrence} to obtain an approximation for temperature. We employ the generalized polynomial chaos (PC) \cite{xiu2010}, which decomposes stochastic quantities into infinite series of orthogonal polynomials of \rvs. Such a series is especially attractive from the post-processing perspective as it is nothing more than a polynomial, hence, easy to interpret and evaluate. An introduction to orthogonal polynomials is given in \aref{orthogonal-polynomials}.

The first step towards a PC expansion is the choice of a suitable polynomial basis $\{ \pcb_i(\vz) \}$, which is typically picked from the Askey scheme of orthogonal polynomials \cite{xiu2010}. The step is crucial as the rate of convergence of PC expansions closely depends on it. There are no strict rules that guarantee the optimal choice \cite{maitre2010, knio2006}; however, there are best practices which say that one should be guided by the probability distributions of the \rvs\ that drive the stochastic system (see \aref{polynomial-chaos}). \tref{askey} in the appendix displays several examples of such paired probability distributions and polynomial bases. For instance, when the \rvs\ $\vZ(\o)$ follow a beta distribution, the Jacobi basis is worth being tried first. On the other hand, the Hermite basis (\fref{hermite}) is preferable for Gaussian \rvs.

Having an appropriate basis chosen, we apply the PC procedure to power in \eref{recurrence} and truncate the resulting infinite series in order to make it feasible for practical implementations. Such an expansion is formally defined as
\begin{equation} \elabel{pc-expansion}
  \oPC{\vars}{\pcorder}{\vP_k(\o)} = \sum_{i = 1}^{\pcterms} \pcc{\vP}_{ki} \; \pcb_i(\vZ(\o))
\end{equation}
where $\{ \pcb_i(\vZ(\o)) \}_{i = 1}^{\pcterms}$ is the truncated basis with $\pcterms$ polynomial terms of $\vars$ variables, and $\pcc{\vP}_{ki} \in \real^\cores$ are the coefficients of the expansion. The latter are computed using spectral projections as it is explained in the appendix, \aref{spectral-projection}. $\pcorder$ denotes the order of the expansion, which determines the maximal degree of the $\vars$-variate polynomials involved in the expansion; hence, $\pcorder$ also defines accuracy.

It can be seen in \eref{recurrence} that, due to the linearity of the operations involved in the recurrence, $\vX_k(\o)$ retains the same polynomial structure as $\vP_k(\o)$. Therefore, using \eref{pc-expansion}, \eref{recurrence} is rewritten as follows, for $k = 1, \dotsc, \steps$:
\begin{equation} \elabel{expanded-recurrence}
  \oPC{\vars}{\pcorder}{\vX_k(\o)} = \mCF_k \: \oPC{\vars}{\pcorder}{\vX_{k-1}(\o)} + \mCS_k \: \oPC{\vars}{\pcorder}{\vP_k(\o)}.
\end{equation}
Thus, there are two PC expansions for two concurrent stochastic processes with the same basis but different coefficients. As shown in \aref{spectral-projection}, \eref{expanded-recurrence} can be reduced to
\begin{equation} \elabel{pc-recurrence}
  \pcc{\vX}_{ki} = \mCF_k \: \pcc{\vX}_{(k - 1)i} + \mCS_k \: \pcc{\vP}_{ki}
\end{equation}
where $k = 1, \dotsc, \steps$ and $i = 1, \dotsc, \pcterms$. Finally, \eref{fourier-output} and \eref{pc-recurrence} are combined together to compute the coefficients of the PC expansion of the temperature vector $\vTO_k(\o)$.

To summarize, let us recall the stochastic recurrence in \eref{recurrence} where, in the presence of correlations, an arbitrary functional $\vP_k(\omega)$ of the uncertain parameters $\vU(\o)$ and random temperature $\vTO_k(\o)$ (see \sref{power-model}) needs to be evaluated and combined with another random vector, $\vX_k(\omega)$. Now, the recurrence in \eref{recurrence} has been replaced with a purely deterministic recurrence in \eref{pc-recurrence} that involves only linear operations. Moreover, the performed spectral decompositions have substituted the heavy thermal system in \eref{fourier-system} with a light polynomial surrogate defined by a set of basis functions $\{ \pcb_i(\vz) \}$ and the corresponding sets of coefficients, namely, $\{ \pcc{\vP}_{ki} \}$ for power and $\{ \pcc{\vTO}_{ki} \}$ for temperature, where $k$ traverses all the $\steps$ intervals $\dt_k$ of the considered time span. Consequently, the output of the proposed framework constitutes two stochastic profiles, the power $\profP{\o}$ and temperature $\profT{\o}$ profiles, that are ready to be analyzed from the UQ perspective.


\subsection{Correlated Random Variables}

\subsection{Multi-Dimensional Integration} \slabel{integration}
The family of Gaussian quadrature rules is superior when it comes to the one-dimensional integration of smooth functions, viz. those that are well-approximated by a polynomial \cite{press2007}. In this case, the integral of interest is replaced by a weighted sum over $n$ points, which necessary gives the \emph{exact} solution when the integrand is a $(2n - 1)$-degree (or less) polynomial. The rules differ by the choice of abscissae and weights for the summation. For instance, the abscissae of the Gauss-Hermite quadrature rule are the roots of the Hermite polynomials, and the weight function is $e^{-x^2}$, which makes the rule suitable for the integration with respect to the Gaussian measure.

In the multi-dimensional case, which we have in \eref{pc-coefficients}, one usually employs so-called cubature rules that are constructed from one-dimensional quadrature rules. The rules are characterized by the level of accuracy $\cblevel \in \natural{1}$. A $\cblevel$-level cubature rule for the $\vars$-dimensional integration is defined as
\[
  \oCub{\vars}{\cblevel}{f} \eqdef \sum_{i = 1}^{\cbpoints} f(\cbp{\vU}_i) w_i
\]
where $\cbpoints$ is the number of summation points, $\cbp{\vU}_i \in \real^\vars$ and $w_i \in \real$ are the prescribed abscissae and weights, respectively. The abscissae are $\vars$-dimensional vectors, which corresponds the number of uncertain parameters $\vU(\o)$.

In order to tackle dependent r.v.'s for a non-Gaussian distribution, the approach proposed in \cite{babuska2010} can be applied.
