Given the thermal specification $\system$ of the system at hand (see \sref{problem-formulation}), an equivalent thermal RC circuit with $\nnodes$ thermal nodes is constructed \cite{kreith2000}.
The structure of the circuit depends on the intended level of granularity and, therefore, impacts the resulting accuracy.
For clarity of presentation, we assume that each processing element is mapped onto one corresponding node, and the thermal package is represented as a set of additional nodes.
An example of such a circuit for a dual-core architecture can be found in \fref{circuit}.

The thermal behavior of the constructed circuit is modeled with the following system of differential-algebraic equations (see \aref{thermal-model} for a derivation):
\begin{subnumcases}{\elabel{fourier-system}}
  \frac{d\vX(\t, \o)}{d\t} = \mA \: \vX(\t, \o) + \mB \: \vP(\t, \o) \elabel{fourier-de} \\
  \vTO(\t, \o) = \mB^T \vX(\t, \o) + \vTO_\amb \elabel{fourier-output}
\end{subnumcases}
where $\vP(\t, \o)$ and $\vTO(\t, \o)$ are the input power and output temperature vectors of the processing elements, respectively, and $\vX \in \real^\nnodes$ is the vector of the internal state of the system.
In general, \eref{fourier-de} does not have a closed-form solution due to the power term, which is an arbitrary function.

Recall that the power and temperature profiles we work with are discrete-time representations of the power consumption and heat dissipation, respectively, which contain $\nsteps$ samples, or steps, covering a certain time span (see \sref{problem-formulation}).
As detailed in \aref{thermal-model}, we let the total power be constant between neighboring power steps and reduce the solution process of \eref{fourier-system} to the following recurrence, for $k = 1, \dotsc, \nsteps$,
\begin{equation} \elabel{recurrence}
  \vX_k(\o) = \mCF_k \: \vX_{k - 1}(\o) + \mCS_k \: \vP_k(\o)
\end{equation}
where $\vX_0(\o) = \vZero$.
In the deterministic case, \eref{recurrence} can be readily employed to perform deterministic transient PTA \cite{thiele2011, ukhov2012}.
In the stochastic case, however, the analysis of \eref{recurrence} is substantially different since $\vP_k(\o)$ and, consequently, $\vX_k(\o)$ and $\vTO_k(\o)$ are probabilistic quantities.
The situation is complicated by the fact that, at each step of the iterative process in \eref{recurrence}, (a) $\vP_k(\o)$ is an arbitrary transformation of the uncertain parameters $\vU(\o)$ and stochastic temperature $\vTO_k(\o)$ (see \sref{power-model}), which results in a multivariate random variable with a generally unknown probability distribution, and (b) $\vU(\o)$, $\vP_k(\o)$, $\vX_k(\o)$, and $\vTO_k(\o)$ are dependent random variables as the latter three are functions of $\vU(\o)$.
Therefore, the operations involved in the recurrence given by \eref{recurrence} are to be performed on dependent random variables with arbitrary probability distributions, which, in general, have no closed-form solutions.
To tackle this difficulty, we utilize PC expansions as we shall discuss in the next section.
