Given a thermal specification $\system$ of the platform (see \sref{architecture-model}), an equivalent thermal RC circuit with $\nodes$ thermal nodes is constructed \cite{kreith2000}. The structure of the circuit depends on the intended level of granularity, which defines the resulting accuracy. For clarity, we assume that each processing element is mapped onto one corresponding node, and the thermal package is represented as a set of additional nodes. In this section, we highlight the most important parts of the thermal model and refer the reader to \aref{thermal-model} for a thorough discussion.

The thermal behavior of the circuit is modeled with the following system of differential-algebraic equations (DAEs):
\begin{subnumcases}{\elabel{fourier-system}}
  \frac{d\vX(\t, \o)}{d\t} = \mA \: \vX(\t, \o) + \mB \: \vP(\t, \o) \elabel{fourier-de} \\
  \vTO(\t, \o) = \mB^T \vX(\t, \o) + \vTO_\amb \elabel{fourier-output}
\end{subnumcases}
where $\vP \in \real^\cores$ and $\vTO \in \real^\cores$ are the input power and output temperature of the processing elements, respectively, and $\vX \in \real^{\nodes}$ is the state vector of the system. As a preliminary step, we reduce the thermal modeling to the following recurrent expression, derived in \aref{thermal-model}, that traverses all the $\steps$ time intervals $\dt_k$ of the input power profile $\profPdyn$:
\begin{align}
  & \vX_k(\t, \o) = \mE(\t) \: \vX_k(0, \o) + \mD(\t) \: \vP_k(0, \o), \elabel{recurrence} \\
  & \vX_k(0, \o) = \vX_{k - 1}(\dt_{k - 1}, \o), \hspace{1em} \vX_0(0, \o) = \vZero. \nonumber
\end{align}
Note that, for the sake of convenience, $\t$ is assumed to start from zero at each interval, \ie, $\t \in [ 0, \dt_k ]$. In the deterministic case, the recurrence in \eref{recurrence} can be readily employed to perform the deterministic transient PTA \cite{thiele2011, ukhov2012}. In the stochastic case, however, the analysis of \eref{recurrence} is substantially different since $\vP_k(\t, \o)$ and, consequently, $\vX_k(\t, \o)$ and $\vTO_k(\t, \o)$ are probabilistic quantities. The situation is complicated by fact that, at each step of the iterative process, (a) $\vP_k(\t, \o)$ is an arbitrary transformation of the uncertain parameters $\vU(\o)$ and stochastic temperature $\vTO_k(\t, \o)$ (see \sref{power-model}), which results in a multivariate \rv\ with a generally unknown probability distribution; (b) $\vU(\o)$, $\vP_k(\t, \o)$, $\vX_k(\t, \o)$, and $\vTO_k(\t, \o)$ are dependent \rvs\ as the latter three are functions of $\vU(\o)$. Therefore, the operations involved in \eref{recurrence} are to be performed on dependent \rvs\ with arbitrary probability distributions, which, in general, have no closed-form solutions. To tackle this difficulty, we utilize the theory of polynomial chaos expansions mentioned in the introduction and further discussed in the next section.
