Given a thermal specification $\system$ of the platform (see \sref{architecture-model}), an equivalent thermal RC circuit with $\nodes$ thermal nodes is constructed \cite{kreith2000}. The structure of the circuit depends on the intended level of granularity and, therefore, defines the resulting accuracy. For clarity, we assume that each processing element is mapped onto one corresponding node, and the thermal package is represented as a set of additional nodes. An example of such a circuit is depicted in \fref{circuit} in the appendix. The appendix, \aref{thermal-model}, also contains a detail derivation of all the equations given below; here we state only the most important results necessary for understanding of the proposed framework.

The thermal behavior of the circuit is modeled with the following system of differential-algebraic equations (DAEs):
\begin{subnumcases}{\elabel{fourier-system}}
  \frac{d\vX(\t, \o)}{d\t} = \mA \: \vX(\t, \o) + \mB \: \vP(\t, \o) \elabel{fourier-de} \\
  \vTO(\t, \o) = \mB^T \vX(\t, \o) + \vTO_\amb \elabel{fourier-output}
\end{subnumcases}
where $\vP, \vTO \in \real^\cores$ are the input power and output temperature vectors of the processing elements, respectively, and $\vX \in \real^{\nodes}$ is the internal state vector of the system. Here we would like to remark that the system in \eref{fourier-system} is the one which the straightforward MC sampling employs to quantify the uncertainty. More precisely, relying on MC as means of UQ, one needs (a) to draw a large numer of samples, say $\mcsamples = 10^4$, from the distribution of $\vU(\o)$; (b) to solve \eref{fourier-system} for each of the samples using some numerical technique; (c) to process the collected sample traces of power and temperature in order to extract the desired statistics. The most time-consuming step is the repetitive solution of the system as the number of passes should be significantly large when the MC sampling is concerned; see, \eg, \cite{diaz-emparanza2002}.

Recall that the input dynamic power profile, $\profPdyn$, is a discrete-time representation of the power consumption, which contains $\steps$ power samples, or steps. As detailed in \aref{thermal-model}, we let the total power be constant between neighbor steps of $\profPdyn$ and reduce the solution process of \eref{fourier-system} to the following recurrence for $k = 1, \dotsc, \steps$:
\begin{equation} \elabel{recurrence}
  \vX_k(\o) = \mE_k \: \vX_{k - 1}(\o) + \mD_k \: \vP_k(\o)
\end{equation}
where $\vX_0(\o) = \vZero$. It is worth being mentioned that if $\profPdyn$ is evenly sampled, $\mE_k = \mE$ and $\mD_k = \mD$, \ie, the coefficients of the recurrence do not change. In the deterministic case, \eref{recurrence} can be readily employed to perform the deterministic transient PTA \cite{thiele2011, ukhov2012}. In the stochastic case, however, the analysis of \eref{recurrence} is substantially different since $\vP_k(\o)$ and, consequently, $\vX_k(\o)$ and $\vTO_k(\o)$ are probabilistic quantities. The situation is complicated by fact that, at each step of the iterative process, (a) $\vP_k(\o)$ is an arbitrary transformation of the uncertain parameters $\vU(\o)$ and stochastic temperature $\vTO_k(\o)$ (see \sref{power-model}), which results in a multivariate \rv\ with a generally unknown probability distribution; (b) $\vU(\o)$, $\vP_k(\o)$, $\vX_k(\o)$, and $\vTO_k(\o)$ are dependent \rvs\ as the latter three are functions of $\vU(\o)$. Therefore, the operations involved in \eref{recurrence} are to be performed on dependent \rvs\ with arbitrary probability distributions, which, in general, have no closed-form solutions. To tackle this difficulty, we utilize the theory of spectral expansions mentioned in the introduction and further discussed in the next section.
