As mentioned in \sref{problem-formulation}, the user of the framework is supposed to decide on the power model for the multiprocessor system under consideration.
Such a model can be generally expressed as the following $\nprocs$-dimensional functional $\oPower$ where $\nprocs$ is the number of processing elements in the system:
\begin{equation} \elabel{power-model}
  \vP(\t, \o) = \oPower{\vTO(\t, \o), \vU(\o)}
\end{equation}
where $\vP(\t, \o) \in \real^\nprocs$ and $\vTO(\t, \o) \in \real^\nprocs$ are vectors of power and temperature, respectively, at time $\t$ and a given outcome of the probability space $\o \in \outcomes$.
Note that, for brevity, we do not repeat the dependency on $\vU(\omega)$ and use $\omega$ instead.
Also, since $\vU(\o) = \oTransform{\vZ(\o)}$, we shall use $\vZ(\o)$ instead of $\vU(\o)$, which implies no loss of generality.
The user can choose any $\oPower$: it can be a closed-form formula, a piece of code, or the output of a power simulator that takes in, for some fixed $\o \in \outcomes$, the temperature vector $\vTO(\t, \o)$ and uncertain parameters $\vU(\o)$ and computes the corresponding total power $\vP(\t, \o)$.
The only assumption we make about $\oPower$ is that the function is smooth in $\vZ(\o)$ and has a finite variance, which is applicable to most physical systems \cite{xiu2010}.
Note also that the operation performed by $\oPower$ is purely deterministic.
In can be seen that the definition of $\oPower$ is flexible enough to account for such effects as the interdependency between leakage and temperature \cite{srivastava2010, liu2007}, which is discussed in \sref{ie-power-model}.
