As discussed in \sref{problem-formulation}, the user of our UQ framework is supposed to decide on the power model for the multiprocessor system under consideration. Such a model can be generally expressed as the following $\cores$-dimensional vector where $\cores$ is the number of processing elements in the system:
\begin{equation} \elabel{power-model}
  \vP(\t, \o) = \f(\vP_\dyn(\t), \vTO(\t, \o), \vU(\o)).
\end{equation}
The user is free to choose any $\f$: it can be a closed-form formula, (deterministic) piece of code, \etc\ that takes in, for some fixed $\t$ and $\o$, the nominal dynamic power $\vP_\dyn(\t)$, stochastic temperature $\vTO(\t, \o)$, and uncertain parameters $\vU(\o)$ and outputs the corresponding total power $\vP(\t, \o)$. It should be noted that the operation performed by $\f$ is purely deterministic. The only assumption we make about $\f$ is that the function is smooth in $\vZ(\o)$ and has a finite variance, which is applicable to most physical systems \cite{xiu2002}. In can be seen that the definition of $\f$ is flexible enough to account for such effects as the interdependency between leakage and temperature \cite{srivastava2010, liu2007}, which is further discussed in \sref{ie-power-model}. Note that, in \eref{power-model}, we do not write explicitly the dependency of $\vP$ and $\vTO$ on $\vU(\omega)$ and use $\omega$ instead; also, since $\vU(\o) = \oTransform{\vZ(\o)}$, in what follows, we shall use $\vZ(\o)$ instead of $\vU(\o)$, which has no loss of generality.
