The goal now is to transform the ``problematic'' term in \eref{recurrence}, \ie, the power term defined by \eref{power-model}, in such a way that the recurrence in \eref{recurrence} becomes computationally tractable.
Our solution is the construction of a surrogate model for the power model in \eref{power-model}, which we further propagate through \eref{recurrence} to obtain an approximation for temperature.
To this end, we employ polynomial chaos (PC) \cite{xiu2010}, which decomposes stochastic quantities into infinite series of orthogonal polynomials of random variables.
Such series are especially attractive from the post-processing perspective as they are nothing more than polynomials; hence, PC expansions are easy to interpret and easy to evaluate.
An introduction to orthogonal polynomials, which we shall rely on in the derivation below, is given in \aref{polynomial-chaos}.

\subsubsection{Polynomial basis}
The first step towards a polynomial expansion is the choice of a suitable polynomial basis $\{ \pcb_i(\vz) \}_{i = 1}^\infty$, which is typically made based on the Askey scheme of orthogonal polynomials \cite{xiu2010}.
The step is crucial as the rate of convergence of PC expansions closely depends on it.
Although there are no strict rules that guarantee the optimal choice \cite{maitre2010, knio2006}, there are best practices saying that one should be guided by the probability distributions of the (independent) random variables that drive the stochastic system at hand (see \aref{polynomial-chaos}).
In our case, these variables are $\vZ(\o)$.
For instance, when a random variable follows a beta distribution, the Jacobi basis is worth being tried first; on the other hand, the Hermite basis is preferable for Gaussian distributions.
In multiple dimensions, which is the case with the $\nvars$-dimensional random variable $\vZ(\o)$, several (possibly different) univariate bases are to be combined together to produce a single $\nvars$-variate polynomial basis $\{ \pcb_i(\vz) \}_{i = 1}^\infty$, $\vz \in \real^\nvars$ (see \cite{xiu2010, maitre2010}).

\subsubsection{Recurrence of polynomial expansions} \slabel{pc-recurrence}
Having chosen an appropriate basis, we apply the PC expansion formalism to the power term in \eref{recurrence} and truncate the resulting infinite series in order to make it feasible for practical implementations.
Such an expansion is defined as
\begin{equation} \elabel{pc-expansion}
  \oPC{\nvars}{\pcorder}{\vP_k(\o)} := \sum_{i = 1}^{\pcterms} \pcc{\vP}_{ki} \; \pcb_i(\vZ(\o))
\end{equation}
where $\{ \pcb_i(\vz) \}_{i = 1}^{\pcterms}$, $\vz \in \real^\nvars$, is the truncated basis with $\pcterms$ polynomial terms of $\nvars$ variables, and $\pcc{\vP}_{ki} \in \real^\nprocs$ are the coefficients of the expansion.
The latter can be computed using spectral projections as it is described in \sref{pc-coefficients}.
$\pcorder$ denotes the order of the expansion, which determines the maximal degree of the $\nvars$-variate polynomials involved in the expansion; hence, $\pcorder$ also determines the resulting accuracy.
The total number of the PC coefficients, $\pcterms$, is given by the following expression, which corresponds to the total-order polynomial space \cite{eldred2008, beck2011}:
\begin{equation} \elabel{pc-terms}
  \pcterms = { \pcorder + \nvars \choose \nvars } = \frac{(\pcorder + \nvars)!}{\pcorder! \, \nvars!}.
\end{equation}

It can be seen in \eref{recurrence} that, due to the linearity of the operations involved in the recurrence, $\vX_k(\o)$ retains the same polynomial structure as $\vP_k(\o)$. Therefore, using \eref{pc-expansion}, \eref{recurrence} is rewritten as follows, for $k = 1, \dotsc, \nsteps$:
\begin{equation} \elabel{expanded-recurrence}
  \oPC{\nvars}{\pcorder}{\vX_k(\o)} = \mCF_k \: \oPC{\nvars}{\pcorder}{\vX_{k-1}(\o)} + \mCS_k \: \oPC{\nvars}{\pcorder}{\vP_k(\o)}.
\end{equation}
Thus, there are two PC expansions for two concurrent stochastic processes with the same basis but different coefficients.

Using the definition in \eref{pc-expansion}, \eref{expanded-recurrence} can be explicitly written as follows:
\[
  \sum_{i = 1}^{\pcterms} \pcc{\vX}_{ki} \: \pcb_i(\vZ(\o)) = \sum_{i = 1}^{\pcterms} \left( \mCF_k \: \pcc{\vX}_{(k - 1)i} + \mCS_k \: \pcc{\vP}_{ki} \right) \pcb_i(\vZ(\o)).
\]
Multiplying the above equation by each polynomial from the basis and making use of the orthogonality property (given in \eref{orthogonality} in \aref{polynomial-chaos}), we obtain the following recurrence:
\begin{equation} \elabel{pc-recurrence}
  \pcc{\vX}_{ki} = \mCF_k \: \pcc{\vX}_{(k - 1)i} + \mCS_k \: \pcc{\vP}_{ki}
\end{equation}
where $k = 1, \dotsc, \nsteps$ and $i = 1, \dotsc, \pcterms$. Finally, \eref{fourier-output} and \eref{pc-recurrence} are combined together to compute the coefficients of the PC expansion of the temperature vector $\vTO_k(\o)$.

\subsubsection{Expansion coefficients} \slabel{pc-coefficients}
The general formula of a truncated PC expansion applied to the power term in \eref{recurrence} is given in \eref{pc-expansion}.
Let us now find the coefficients of this expansion, $\pcc{\vP}_{ki}$, which will be propagated to temperature (using \eref{pc-recurrence} and \eref{fourier-output}).
To this end, a spectral projection of the stochastic quantity being expanded, \ie, $\vP_k(\o)$, is to be performed onto the space spanned by $\{ \pcb_i(\vZ(\o)) \}_{i = 1}^{\pcterms}$, where $\pcterms$ is the number of polynomials in the truncated basis.
This means that we need to compute inner products of \eref{power-model} with each polynomial from the basis as
\[
  \oInner{\vP_k(\o)}{\pcb_i(\vZ(\o))} = \oInner{\sum_{j=1}^{\pcterms} \pcc{\vP}_{kj} \: \pcb_j(\vZ(\o))}{\pcb_i(\vZ(\o))}
\]
where $\oInner{\cdot}{\cdot}$ stands for the inner product (see \aref{polynomial-chaos} for a definition), $i = 1, \dotsc, \pcterms$, and $k = 1, \dotsc, \nsteps$. Making use of the orthogonality property of the basis, we obtain
\begin{equation} \elabel{pc-coefficients}
  \pcc{\vP}_{ki} = \frac{1}{\pcn_i} \oInner{\vP_k(\o)}{\pcb_i(\vZ(\o))}.
\end{equation}
In general, the inner product in \eref{pc-coefficients}, given in \eref{inner-product} in \aref{polynomial-chaos}, should be evaluated numerically.
\updated{This task is typically accomplished by virtue of a quadrature rule \cite{press2007}, which, loosely speaking, is a weighted summation over the integrand values computed at a set of prescribed points.
These points along with the corresponding weights are generally precomputed and tabulated since they do not depend the quantity being integrated; see \aref{gauss-quadrature}.
The only thing, which is essential for the development in this section, is the number of such points denoted by $\qdorder$.
The rest of the integration procedure is discussed separately in \aref{gauss-quadrature}.
However, it is worth noting here that, since $\vP(\o)$ depends on temperature as shown in \eref{power-model}, at each step of the iterative process in \eref{pc-recurrence}, the computation of $\pcc{\vP}_{ki}$ should be done with respect to the PC expansion of the temperature vector $\vTO_{k - 1}(\o)$.}

\subsubsection{Computational challenges} \slabel{computational-challenges}
$ $\updated{The construction process of the stochastic power and temperature profiles, implemented inside our prototype of the proposed framework, is estimated to have the following time complexity:}
\[
  \updated{\O(\nsteps \, \nprocs \, \qdorder \, \pcterms + \nsteps \, \qdorder \, \oPower{\nprocs})}
\]
\updated{where $\oPower{\nprocs}$ denotes the complexity of the computations associated with the power model in \eref{power-model}.
The expression can be detailed further by expanding $\pcterms$ and $\qdorder$.
The exact formula for $\pcterms$ is given in \eref{pc-terms}, and the limiting behavior of $\pcterms$ with respect to $\nvars$ is $\O(\nvars^\pcorder / \pcorder!)$.
For brute-force quadrature rules, $\log(\qdorder)$ is $\O(\nvars)$, meaning that the dependency of $\qdorder$ on $\nvars$ is exponential.
It can be seen that the theory of PC expansions suffers from the so-called curse of dimensionality \cite{xiu2010, maitre2010, eldred2008}.
More precisely, when $\nvars$ increases, the number of polynomial terms as well as the complexity of the corresponding coefficients exhibit a growth, which is exponential without special treatments.}
The problem does not have a general solution and is one of the central topics of many ongoing studies.
In this paper, we mitigate this issue by: (a) keeping the number of stochastic dimensions low using the KL decomposition as we shall see in \sref{ie-uncertain-parameters} and (b) utilizing efficient integration techniques as discussed in \aref{gauss-quadrature}.
\updated{In particular, for sparse integration grids based on Gaussian quadratures, $\log(\qdorder)$ is $\O(\log(\nvars))$, meaning that the dependency of $\qdorder$ on $\nvars$ is only polynomial \cite{heiss2008}.}

To summarize, let us recall the stochastic recurrence in \eref{recurrence} where, in the presence of correlations, an arbitrary functional $\vP_k(\omega)$ of the uncertain parameters $\vU(\o)$ and random temperature $\vTO_k(\o)$ (see \sref{power-model}) needs to be evaluated and combined with another random vector, $\vX_k(\omega)$.
Now the recurrence in \eref{recurrence} has been replaced with a purely deterministic recurrence in \eref{pc-recurrence} that involves only linear operations.
Thus, the performed spectral decompositions have effectively substituted the heavy thermal system in \eref{fourier-system} with a light polynomial surrogate defined by a set of basis functions $\{ \pcb_i(\vz) \}_{i = 1}^\pcterms$ and the corresponding sets of coefficients, namely, $\{ \pcc{\vP}_{ki} \}_{i = 1}^\pcterms$ for power and $\{ \pcc{\vTO}_{ki} \}_{i = 1}^\pcterms$ for temperature, where $k$ traverses the $\nsteps$ intervals of the considered time span.
Consequently, the output of the proposed PTA framework constitutes two stochastic profiles: the power and temperature profiles denoted by $\profileP{\o}$ and $\profileT{\o}$, respectively, which are ready to be analyzed.

\updated{Finally, note the easy and generality of taking the uncertainty into consideration using the proposed approach: the above derivation is delivered from any explicit formula for any particular uncertain parameter.
In contrast, a typical solution from the literature related to process variation is saturated with ad hoc expressions and should be tailored by the user for each new parameter individually; see, \eg, \cite{bhardwaj2008, ghanta2006}.
Our framework provides a great flexibility in this regard.}
