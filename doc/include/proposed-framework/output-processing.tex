Due to the properties of PC expansions, specifically due to the basis functions and their pairwise orthogonality (see \aref{orthogonal-polynomials}), the obtained polynomial traces allow for various prospective analyses to be performed with no effort. For instance, consider the PC expansion of temperature at the $k$th time interval given as
\vspace{-0.2em}%
\begin{equation} \elabel{pc-k}
  \oPC{\nvars}{\pcorder}{\vTO_k(\o)} = \sum_{i = 1}^{\pcterms} \pcc{\vTO}_{ki} \pcb_i(\vZ(\o))
\end{equation}
\vspace{-0.2em}%
where $\pcc{\vTO}_{ki}$ are computed using \eref{fourier-output} and \eref{pc-recurrence}. Let us, for example, find the expectation and variance of the expansion. As shown in \aref{spectral-projection}, these values have the following simple expressions solely based on the coefficients:
\vspace{-0.2em}%
\begin{equation} \elabel{pc-moments}
  \oExp{\vTO_k(\o)} = \pcc{\vTO}_{k1} \hspace{1em} \text{and} \hspace{1em} \oVar{\vTO_k(\o)} = \sum_{i = 2}^{\pcterms} \pcn_i \: \pcc{\vTO}_{ki}^2
\end{equation}
\vspace{-0.2em}%
for expectation and variance, respectively, where the squaring is element-wise, and $\pcn_i$ are normalization constants with analytical expressions. Such quantities as \cdfs\ and \pdfs\ can be estimated by sampling \eref{pc-k}; each sample is a trivial evaluation of a polynomial. Furthermore, global and local sensitivity analyses of deterministic and non-deterministic quantities can be readily conducted on \eref{pc-k}.
