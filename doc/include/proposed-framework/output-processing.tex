Due to the properties of PC expansions, specifically, due to the basis functions and thier pairwise orthogonality (see \aref{orthogonal-polynomials}), the obtained polynomial traces are trivial from the prospective UQ analyses. For instance, consider the PC expansion of temperature at the $k$th time interval given as
\begin{equation} \elabel{pc-k}
  \oPC{\vars}{\pcorder}{\vTO_k(\o)} = \sum_{i = 1}^{\pcterms} \pcc{\vTO}_{ki} \pcb_i(\vZ(\o))
\end{equation}
where $\pcc{\vTO}_{ki}$ are computed using \eref{fourier-output} and \eref{pc-recurrence}. Let us, for example, find the expectation and covariance of the expansion. As shown in \aref{spectral-projection}, these statistics have the following simple expressions solely based on the coefficients:
\begin{align*}
  & \oExp{\vTO_k(\o)} = \pcc{\vTO}_{k1}, & \oCov{\vTO_k(\o)} = \sum_{i = 2}^{\pcterms} \pcn_i \: \pcc{\vTO}_{ki} \pcc{\vTO}_{ki}^T
\end{align*}
where $\pcn_i$ are normalization constants. Furthermore, global and local sensitivity analyses of deterministic and non-deterministic quantities can be readily conducted on \eref{pc-k}; see, \eg, \cite{eldred2009, maitre2010}. The \cdf\ as well as the \pdf\ can be estimated by sampling \eref{pc-k}. Each sample is a trivial evaluation of a multivariate polynomial; on the contrary, when a MC-based technique is employed to perform UQ of \eref{fourier-system}, a sample is a complete simulation of $\profPdyn$. Since the number of such samples should be considerably large, \eg, in the order of $10^4$ \cite{xiu2010, diaz-emparanza2002}, to obtain reliable results, MC-based approaches become highly time-consuming and often infeasible in practice.
