Independence of the parameters is a prerequisite of PC expansions.
In general, however, $\vU(\o)$ can be correlated and, therefore, should be preprocessed in order to fulfill the requirement.
To this end, an adequate probability transformation should be undertaken \cite{eldred2008}.
Denote such a transformation by $\vU(\o) = \oTransform{\vZ(\o)}$, which relates the correlated uncertain parameters $\vU(\o)$ with $\nvars$ independent random variables $\vZ(\o)$.

Correlated random variables can be transformed into uncorrelated ones \via\ a linear mapping based on a factorization procedure of the covariance matrix or covariance function of $\vU(\o)$; the procedure is known as the Karhunen-Lo\`{e}ve (KL) decomposition \cite{ghanem1991}.
If, in addition, the correlated variables form a Gaussian vector then the uncorrelated ones are also mutually independent.
In the general case (non-Gaussian), the most prominent solutions to attain independence are the Rosenblatt \cite{rosenblatt1952} and Nataf transformations \cite{li2008}.\footnote{Only a few alternatives are listed here, and such techniques as independent component analysis (ICA) are left outside of the scope of the paper.}
Rosenblatt's approach is suitable when the joint probability distribution function of the uncertain parameters $\vU(\o)$ is known; however, such information is rarely available.
The marginal probability distributions and correlation matrix of $\vU(\o)$ are more likely to be given, which are already sufficient for perform the Nataf transformation.\footnote{The transformation is an approximation, which operates under the assumption that the copula of the distribution is elliptical.}
The Nataf transformation produces correlated Gaussian variables, which are then turned into independent ones by virtue of the KL decomposition mentioned earlier.

Apart from the extraction of the independent parameters $\vZ(\o)$, an essential operation at this stage is model order reduction since the number of stochastic dimensions of the problem directly impacts the complexity of the rest of the computations.
The intuition is that, due to the correlations possessed by the random variables in $\vU(\o)$, some of them can be harmlessly replaced by combinations of the rest, leading to a smaller number of the variables in $\vZ(\o)$.
This operation is often treaded as a part of the KL decomposition.

In \sref{ie-uncertain-parameters}, we shall demonstrate the Nataf transformation together with the KL decomposition.
A description of the latter can also be found in \aref{karhunen-loeve}.
