To be legitimate in the context of probability theory, we need to start from the formal definition of a probability space that we shall reside in. A probability space is defined as a triple $(\outcomes, \sAlgebra, \pMeasure)$ where $\outcomes$ is a set of outcomes, $\sAlgebra \subseteq 2^\outcomes$ is a $\sigma$-algebra on $\outcomes$, and $\pMeasure: \sAlgebra \to [0, 1]$ is a probability measure \cite{durrett2010}. Loosely speaking, an $n$-dimensional \rv\ is then a mapping $\v{X}: \o \in \outcomes \mapsto \v{X}(\o) \in \real^n$. In what follows, the space $(\outcomes, \sAlgebra, \pMeasure)$ is always implied.

As stated earlier, in this work, we adapt the theory of PC expansions in order to construct an approximation to the initial problem. Mutual independence and finiteness of the set of uncertain parameters $\vU(\o)$ are essential prerequisites of PC. In general, however, $\vU(\o)$ can be given as (a) a stochastic process, \ie, an infinite collection of \rvs, with a prescribed covariance kernel; or (b) as a finite but correlated set of \rvs. In these cases, $\vU(\o)$ should be preprocessed in order to fulfill the conditions. To this end, an adequate probability transformation should be undertaken. Denote such a transformation by $\vU(\o) \approx \oTransform{\vZ(\o)}$, which relates the initial uncertain parameters $\vU(\o)$ with a set of $\vars$ mutually independent \rvs\ $\vZ(\o)$. Without loss of generality, $\vZ(\o)$ are assumed to be centered and normalized, \ie, $\oExp{\vZ(\o)} = \mZero$ and $\oCov{\vZ(\o)} = \mI$. Note that the transformation, in general, is an approximation. In the following paragraph, we outline a possible strategy of choosing a suitable transformation and refer the reader to \cite{xiu2010, eldred2009} for in-depth discussions.

A stochastic process with a given covariance function is commonly addressed by the KL expansion \cite{xiu2010, maitre2010, ghanem1991}. KL decomposes the covariance function into a set of eigenfunctions and the corresponding eigenvalues and, sequentially, expands the process into a Fourier-like series, which is further being truncated. The result is a finite set of orthogonal \rvs\ that are typically assumed to be independent; however, the assumption formally holds only for Gaussian processes. In the finite-dimensional case, \ie, $\vU(\o)$ is a vector of \rvs, KL reduces to the well-known principal component analysis (PCA), singular value decomposition (SVD), or eigenvalue decomposition---they are all the same---of the covariance matrix of $\vU(\o)$; the technique is demonstrated in \sref{ie-uncertain-parameters}. When a finite set of \rvs\ does not follow a Gaussian distribution, it makes sense to look at alternative transformations. The most prominent examples are the Rosenblatt and Nataf transformations \cite{eldred2009, li2008}. Rosenblatt's approach is suitable when the joint \pdf\ of $\vU(\o)$ is known; however, such information is rarely available. The marginal distributions and correlation matrix of $\vU(\o)$ are more likely to be given, which are sufficient to perform the Nataf transformation. Besides, the latter is rather straightforward for implementation; see \cite{li2008}.
