For each time interval of the input dynamic power profile $\prof{\mP_\dyn}$, \eref{pc-recurrence} provides the coefficients of the PC expansion for temperature $\vTO(\t, \o)$ based on the coefficients of the PC expansion for power $\vP(\t, \o)$. Consequently, the output of the proposed framework constitutes two stochastic system profiles: the power $\prof{\mP(\o)}$ and temperature $\prof{\mTO(\o)}$ profiles. Due to orthogonality of the components of a PC expansion, the obtained traces can easily be analyzed from different perspectives \cite{eldred2009, maitre2010}. For instance, consider the PC expansion of temperature for the $k$th time interval
\begin{equation} \elabel{pc-k}
  \oPC{\vars}{\pcorder}{\vTO_k(\t, \o)} = \sum_{i = 1}^{\pcterms} \pcc{\vTO}_{ki}(\t) \pcb_i(\vZ(\o))
\end{equation}
where $\t \in [0, \dt_k]$, and $\pcc{\vTO}_{ki}(\t)$ are found using \eref{pc-recurrence} and \eref{fourier-output}. Let us, for example, find the expectation and covariance of the expansion. By definition \cite{xiu2002}, the first polynomial $\pcb_1(\vZ(\o))$ in a polynomial basis is 1, hence, $\oExp{\pcb_1(\vZ(\o))} = 1$. Therefore, using \eref{orthogonality}, we conclude that $\oExp{\pcb_i(\vZ(\o))} = 0$ for $i \in \{ 2, \dotsc, \pcterms \}$. Consequently, the expectation is
\[
  \oExp{\vTO_k(\t, \o)} = \pcc{\vTO}_{k1}(\t)
\]
and the covariance is
\[
  \oCov{\vTO_k(\t, \o)} = \sum_{i = 2}^{\pcterms} \pcc{\vTO}_{ki}(\t) \pcc{\vTO}_{ki}(\t)^T \pcn_i
\]
The probability density (or mass) function as well as the cumulative distribution function can be estimated by sampling \eref{pc-k}. Each sample is a trivial evaluation of a multivariate polynomial. On the contrary, when a MC-based technique is employed to quantify \eref{fourier-system}, a sample is a complete simulation of $\prof{\mP_\dyn}$. Since the number of such samples should be considerably large to obtain reliable results \cite{xiu2009}, MC-based approaches become highly time consuming.
