For each time interval of the input dynamic power profile $\profPdyn$, \eref{pc-coefficients} and \eref{pc-recurrence} yield the coefficients of the PC expansions for the power $\vP(\t, \o)$ and temperature $\vTO(\t, \o)$ vectors. Consequently, the output of the proposed framework constitutes two stochastic system profiles: the power $\profP{\o}$ and temperature $\profT{\o}$ profiles. Due to orthogonality of the components of a PC expansion, the obtained traces can easily be analyzed from different perspectives. For instance, consider the PC expansion of temperature for the $k$th time interval given as
\begin{equation} \elabel{pc-k}
  \oPC{\vars}{\pcorder}{\vTO_k(\t, \o)} = \sum_{i = 1}^{\pcterms} \pcc{\vTO}_{ki}(\t) \pcb_i(\vZ(\o))
\end{equation}
where $\t \in [0, \dt_k]$ (recall that we restart $\t$ at each time interval), and $\pcc{\vTO}_{ki}(\t)$ are found using \eref{pc-recurrence} and \eref{fourier-output}. Let us, for example, find the expectation and covariance of the expansion. By definition \cite{xiu2010}, the first polynomial $\pcb_1(\vZ(\o))$ in a orthogonal polynomial basis is 1, hence, $\oExp{\pcb_1(\vZ(\o))} = 1$. Therefore, using \eref{orthogonality}, we conclude that $\oExp{\pcb_i(\vZ(\o))} = 0$ for $i \in \{ 2, \dotsc, \pcterms \}$. Consequently, the expectation and covariance are
\begin{align*}
  & \oExp{\vTO_k(\t, \o)} = \pcc{\vTO}_{k1}(\t) \\
  & \oCov{\vTO_k(\t, \o)} = \sum_{i = 2}^{\pcterms} \pcc{\vTO}_{ki}(\t) \pcc{\vTO}_{ki}(\t)^T \pcn_i
\end{align*}
Further, global and local sensitivity analysis of deterministic and non-deterministic quantities can be readily conducted; see, e.g., \cite{eldred2009, maitre2010}. The probability density (or mass) function as well as the cumulative distribution function can be estimated by sampling \eref{pc-k}. Each sample is a trivial evaluation of a multivariate polynomial. On the contrary, when a MC-based technique is employed to quantify \eref{fourier-system}, a sample is a complete simulation of $\profPdyn$. Since the number of such samples should be considerably large, e.g., in the order of $10^4$ \cite{xiu2010}, to obtain reliable results, MC-based approaches become highly time consuming and often infeasible.
