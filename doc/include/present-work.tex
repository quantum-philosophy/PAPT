Our work makes the following main contribution.
We develop, for the first time, a probabilistic framework for the analysis of the transient power and temperature profiles of multiprocessor systems subject to the uncertainty due to process variation.
The proposed technique is flexible in modeling diverse probability distributions, specified by the user, of the uncertain parameters, such as the effective channel length and gate oxide thickness, and there are no assumptions on the distributions of the resulting power and temperature traces, as these distributions are unlikely to be known \apriori.
Furthermore, our technique is capable of capturing arbitrary joint effects of the uncertain parameters on the system since the parameters are injected into the framework as a ``black box'', which is also defined by the user.
Our approach is founded on the basis of PC expansions, mentioned in \sref{prior-work}, which constitutes an attractive alternative to MC sampling as it possess much faster convergence properties and provides analytical, intuitive for the subsequent interpretation and analysis, representations of system responses to stochastic inputs.
We also illustrate the framework considering one of the most important parameters affected by process variation: the effective channel length, which we model using beta distributions, opposed to the commonly found Gaussian model (demonstrated in \sref{introduction} and \sref{prior-work}).
Note, however, that our approach is not bounded to any particular source of variability.
