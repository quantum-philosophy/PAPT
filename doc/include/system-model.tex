\subsection{Power Model} \slabel{power-model}
Without loss of generality, we focus on the leakage power as it is motivated in \sref{introduction}. The power dissipation of the system with $\cores$ processing elements at time $\t$ is modeled as
\[
  \vP(\t, \vTO(\t, \o), \o) = \vP_\dyn(\t) + \vP_\leak(\vTO(\t, \o), \o)
\]
where $\vP_\dyn, \vP_\leak \in \real^\cores$ are the dynamic and leakage power, respectively, and $\vTO \in \real^\cores$ is temperature. The leakage power is modeled as a function of temperature due to the well-known, strong interdependency between them.

\subsection{Thermal Model} \slabel{thermal-model}
Given the system description $\system$, an equivalent thermal RC circuit with $\nodes$ \definition{thermal nodes} is constructed \cite{kreith2000}. The structure of the circuit depends on the intended level of details defining the accuracy of the final model.

The thermal behaviour of the circuit is modeled with the following system of differential-algebraic equations (DAEs) with the state-space dimension equal to $\nodes$:
\begin{align*}
  & \mC \frac{d\vTI(\t, \o)}{d\t} + \mG \vTI(\t, \o) = \mM \vP(\t, \vTO(\t, \o), \o) \\
  & \vTO(\t, \o) = \mM^T \vTI(\t, \o) + \vTO_\amb
\end{align*}
$\mC, \mG \in \real^{\nodes \times \nodes}$ are the matrices of the thermal capacitance and conductance, respectively. $\mC$ is a diagonal matrix with positive elements, and $\mG$ is a symmetric, positive-definite matrix. $\vTI \in \real^\nodes$ is the state vector of the system, which corresponds to the difference between the temperature of the thermal nodes and the ambient temperature. $\vP \in \real^\cores$ is the input vector of the power dissipation of the processing elements, and $\mM \in \real^{\nodes \times \cores}$ is its mapping matrix to the thermal nodes. $\vTO \in \real^\cores$ is the output vector of the system, which is the temperature of the processing elements. Finally, $\vTO_\amb \in \real^\cores$ is the vector of the ambient temperature, which, for clarity of presentation and without loss of generality, we model as a deterministic variable.

For convenience, we perform an auxiliary transformation of the original system proposed in \cite{ukhov2012}. Let $\vX = \mC^{1/2} \vTI$, $\mA = \mC^{1/2} \mG \mC^{1/2}$, and $\mB = \mC^{1/2} \mM$, then
\begin{align}
  & \frac{d\vX(\t, \o)}{d\t} = \mA \vX(\t, \o) + \mB \vP(\t, \vTO(\t, \o), \o) \elabel{fourier} \\
  & \vTO(\t, \o) = \mB^T \vX(\t, \o) + \vTO_\amb \nonumber
\end{align}
where $\vX$ is the new state vector. It can be seen that the coefficient matrix $\mA$ preserves the properties of $\mG$, i.e., it is symmetric and positive-definite.
