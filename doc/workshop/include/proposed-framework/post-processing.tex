Due to the properties of PC expansions the obtained polynomial traces allow for
various prospective analyses to be performed with no effort. For instance,
consider the PC expansion of temperature at the $k$th moment of time given by
\begin{equation} \elabel{pc-k}
  \oPC{\nvars}{\pcorder}{\vTO_k} = \sum_{i = 1}^{\pcterms} \pcc{\vTO}_{ki} \pcb_i(\vZ)
\end{equation}
where $\pcc{\vTO}_{ki}$ are computed using \eref{fourier-output} and
\eref{pc-recurrence}. The expected value and variance have the following simple
expressions solely based on the coefficients:
\begin{equation} \elabel{pc-moments}
  \oExp{\vTO_k} = \pcc{\vTO}_{k1} \hspace{1em} \text{and} \hspace{1em} \oVar{\vTO_k} = \sum_{i = 2}^{\pcterms} \pcn_i \: \pcc{\vTO}_{ki}^2
\end{equation}
where the squaring should be understood elementwise. Such quantities as \cdfs,
\pdfs, probabilities of certain events, \etc\ can be estimated by sampling
\eref{pc-k}; each sample is a trivial evaluation of a polynomial.
