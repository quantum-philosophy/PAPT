The thermal behavior of the system is modeled with the following system of
differential-algebraic equations:
\begin{subnumcases}{\elabel{fourier-system}}
  \frac{d\,\vS(\t, \vU)}{d\t} = \mA \: \vS(\t, \vU) + \mB \: \vP(\t, \vU) \elabel{fourier-de} \\
  \vTO(\t, \vU) = \mB^T \vS(\t, \vU) + \vTO_\amb \elabel{fourier-output}
\end{subnumcases}
where $\vP(\t, \vU)$ and $\vTO(\t, \vU)$ are the input power and output
temperature vectors of the processing elements, respectively, and $\vS(\t, \vU)
\in \real^\nnodes$ is the vector of the internal state of the system. In
general, the system in \eref{fourier-de} is nonlinear and does not have a
closed-form solution. A detailed description of all the quantities involved in
\eref{fourier-system} is can be found in \cite{ukhov2014}.

We let the total power be constant between neighboring power steps and reduce
the solution process of \eref{fourier-system} to the following recurrence, for
$k = 1, \dotsc, \nsteps$,
\begin{equation} \elabel{recurrence}
  \vS_k = \mCF_k \: \vS_{k - 1} + \mCS_k \: \vP_k
\end{equation}
where $\vS_0 = \vZero$. The coefficient matrices $\mCF_k$ and $\mCS_k$ are
irrelevant for the purpose of this paper and can be found in \cite{ukhov2014}.
In the deterministic case, \eref{recurrence} can be readily employed to perform
deterministic transient PTA \cite{ukhov2012}. In the stochastic case, however,
the analysis of \eref{recurrence} is substantially different since $\vP_k$,
$\vS_k$, and $\vTO_k$ are probabilistic quantities. To tackle this difficulty,
we utilize PC expansions as follows.
