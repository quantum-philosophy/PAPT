The goal now is to transform the ``problematic'' term in \eref{recurrence}, \ie,
the power term defined by \eref{power-model}, in such a way that the recurrence
in \eref{recurrence} becomes computationally tractable. Our solution is the
construction of a surrogate model for the power model in \eref{power-model},
which we further propagate through \eref{recurrence} to obtain an approximation
for temperature. To this end, we employ polynomial chaos (PC) \cite{xiu2010},
which decomposes stochastic quantities into infinite series of orthogonal
polynomials of random variables. An introduction to orthogonal polynomials,
which we rely on in what follows, can be found in \cite{ukhov2014}.

\subsubsection{Polynomial basis} \slabel{pc-basis}
The first step towards a polynomial expansion is the choice of a suitable
polynomial basis, which is typically made based on the Askey scheme of
orthogonal polynomials \cite{xiu2010}. The step is crucial as the rate of
convergence of PC expansions closely depends on it. In multiple dimensions,
which is the case with the $\nvars$-dimensional random variable $\vZ$, several
(possibly different) univariate bases are to be combined together to produce a
single $\nvars$-variate polynomial basis, which we denote by $\{ \pcb_i:
  \real^\nvars \to \real \}_{i = 1}^\infty$; see \cite{xiu2010}.

\subsubsection{Recurrence of polynomial expansions} \slabel{pc-recurrence}
Having chosen an appropriate basis, we apply the PC expansion formalism to the
power term in \eref{recurrence} and truncate the resulting infinite series in
order to make it feasible for practical implementations. Such an expansion is
defined as follows:
\begin{equation} \elabel{pc-expansion}
  \oPC{\nvars}{\pcorder}{\vP_k} := \sum_{i = 1}^{\pcterms} \pcc{\vP}_{ki} \, \pcb_i(\vZ)
\end{equation}
where $\{ \pcb_i: \real^\nvars \to \real \}_{i = 1}^{\pcterms}$ is the truncated
basis with $\pcterms$ polynomials in $\nvars$ variables, and $\pcc{\vP}_{ki} \in
\real^\nprocs$ are the coefficients of the expansion, which are deterministic.
The latter can be computed using spectral projections as it is described in
\sref{pc-coefficients}. $\pcorder$ denotes the order of the expansion, which
determines the maximal degree of the $\nvars$-variate polynomials involved in
the expansion; hence, $\pcorder$ also determines the resulting accuracy. The
total number of the PC coefficients $\pcterms$ is given by the following
expression, which corresponds to the total-order polynomial space
\cite{eldred2008, beck2011}:
\begin{equation} \elabel{pc-terms}
  \pcterms = { \pcorder + \nvars \choose \nvars }.
\end{equation}

Due to the linearity of the operations involved in \eref{recurrence}, $\vS_k$
retains the same polynomial structure as $\vP_k$. It can be shown that, using
\eref{pc-expansion}, \eref{recurrence} can be rewritten as follows:
\begin{equation} \elabel{pc-recurrence}
  \pcc{\vS}_{ki} = \mCF_k \: \pcc{\vS}_{(k - 1)i} + \mCS_k \: \pcc{\vP}_{ki}
\end{equation}
where $k = 1, \dotsc, \nsteps$ and $i = 1, \dotsc, \pcterms$. Finally,
\eref{fourier-output} and \eref{pc-recurrence} are combined together to compute
the coefficients of the PC expansion of the temperature vector $\vTO_k$.

\subsubsection{Expansion coefficients} \slabel{pc-coefficients}
The general formula of a truncated PC expansion applied to the power term in
\eref{recurrence} is given in \eref{pc-expansion}. Let us now find the
coefficients $\{ \pcc{\vP}_{ki} \}$ of this expansion, which will be propagated
to temperature (using \eref{pc-recurrence} and \eref{fourier-output}). To this
end, a spectral projection of the stochastic quantity being expanded---that is,
of $\vP_k$ as a function of $\vZ$ \via\ $\vU = \oTransform{\vZ}$ discussed in
\sref{uncertain-parameters}---is to be performed onto the space spanned by the
$\nvars$-variate polynomials $\{ \pcb_i \}_{i = 1}^{\pcterms}$, where $\pcterms$
is the number of polynomials in the truncated basis. This means that we need to
compute the inner product of \eref{power-model} with each polynomial from the
basis as follows:
\[
  \oInner{\vP_k}{\pcb_i} = \oInner{\sum_{j=1}^{\pcterms} \pcc{\vP}_{kj} \: \pcb_j}{\pcb_i}
\]
where $i = 1, \dotsc, \pcterms$, $k = 1, \dotsc, \nsteps$, and
$\oInner{\cdot}{\cdot}$ stands for the inner product, which should be understood
elementwise. From the above equation, we obtain
\begin{equation} \elabel{pc-coefficients}
  \pcc{\vP}_{ki} = \frac{1}{\pcn_i} \oInner{\vP_k}{\pcb_i}
\end{equation}
where $\{ \pcn_i = \oInner{\pcb_i}{\pcb_i} \}_{i = 1}^\pcterms$ are
normalization constants.

The inner product in \eref{pc-coefficients} should be evaluated numerically.
This task is accomplished by virtue of a quadrature rule. Denote such a
quadrature-based approximation of \eref{pc-coefficients} by
\begin{equation} \elabel{pc-coefficients-quadrature}
  \pcc{\vP}_{ki} = \frac{1}{\pcn_i} \oQuad{\nvars}{\qdlevel}{\vP_k \, \pcb_i}
\end{equation}
where $\qdlevel$ is the level of the quadrature utilized. The procedure is
detailed in \cite{ukhov2014}; here we only note that $\nvars$ and $\qdlevel$
dictate the number of quadrature points, which we shall denote by $\qdorder$.
Also, it is worth emphasizing that, since power depends on temperature as shown
in \eref{power-model}, at each step of the recurrence in \eref{pc-recurrence},
the computation of $\pcc{\vP}_{ki}$ should be done with respect to the PC
expansion of the temperature vector $\vTO_{k - 1}$.

To summarize, the recurrence in \eref{recurrence} has been replaced with a
purely deterministic recurrence in \eref{pc-recurrence}. More importantly, the
heavy thermal system in \eref{fourier-system} has been substituted with a light
polynomial surrogate defined by a set of basis functions $\{ \pcb_i \}_{i =
1}^\pcterms$ and the corresponding sets of coefficients, namely, $\{
\pcc{\vP}_{ki} \}_{i = 1}^\pcterms$ for power and $\{ \pcc{\vTO}_{ki} \}_{i =
1}^\pcterms$ for temperature, where $k$ traverses the $\nsteps$ intervals of the
considered time span. Consequently, the output of the proposed PTA framework
constitutes two stochastic profiles: the power and temperature profiles denoted
by $\profileP$ and $\profileT$, respectively, which are ready to be analyzed.
