Process variation constitutes one of the major concerns of electronic system
designs \cite{srivastava2010}. A crucial implication of process variation is
that it renders the key parameters of a technological process, \eg, the
effective channel length, gate oxide thickness, and threshold voltage, as random
quantities. Therefore, the same workload applied to two ``identical'' dies can
lead to two different power and, thus, temperature profiles since the
dissipation of power and heat essentially depends on the aforementioned
stochastic parameters. This concern is especially urgent due to the
interdependence between the leakage power and temperature. Consequently, process
variation leads to performance degradation in the best case and to severe faults
or burnt silicon in the worst scenario. Under these circumstances, uncertainty
quantification \cite{xiu2010} has evolved into an indispensable asset of
electronic system design workflows in order to provide them with guaranties on
the efficiency and robustness of products.

In this work, we present a probabilistic framework for the analysis of the
transient power and temperature profiles of electronic systems subject to the
uncertainty due to process variation. The proposed technique is flexible in
modeling diverse probability distributions, specified by the user, of the
uncertain parameters, such as the effective channel length and gate oxide
thickness. Moreover, there are no assumptions on the distributions of the
resulting power and temperature traces as these distributions are unlikely to be
known \apriori. The proposed technique is capable of capturing arbitrary joint
effects of the uncertain parameters on the system since the impact of these
parameters is introduced into the framework as a ``black box,'' which is also
defined by the user. In particular, it allows for the leakage-temperature
interdependence to be taken into account with no effort. Our approach is founded
on the basis of polynomial chaos (PC) expansions, which constitute an attractive
alternative to Monte Carlo (MC) sampling.
